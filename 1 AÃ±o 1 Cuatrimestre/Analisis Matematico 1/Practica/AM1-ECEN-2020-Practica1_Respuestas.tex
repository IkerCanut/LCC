\documentclass[10pt]{article}
\usepackage{hyperref}
\hypersetup{
    colorlinks=true,
    linkcolor=black,
    filecolor=magenta,      
    urlcolor=cyan,
}

\usepackage{import}
\usepackage{siunitx}
\usepackage{esvect}
\usepackage{fourier}
\usepackage{amssymb}
\usepackage{amsmath}
\usepackage{multicol}
\usepackage{geometry}
\usepackage[framemethod=TikZ]{mdframed}
\geometry{
 a4paper,
 total={160mm,237mm},
 left=30mm,
 top=30mm,
}

\usepackage{tikz}
\usetikzlibrary{automata, positioning, calc, through, angles, quotes, intersections}
\usepackage{multirow}

\subimport{.}{environment}

\author{Iker M. Canut}
\begin{document}
\title{Introducción a la Matemática}
\maketitle
\date
\newpage

\tableofcontents
\newpage

\section{Unidad 1: Numeros Reales}

\begin{prf}[Propiedad Cancelativa de la Suma]{}
Sea $d=a+b$, y por ende, $d=b+c$, por el Axioma 5, existe $y$ que es opuesto  a $a$, entonces:
$$y+d=y+(a+b) \overset{A2}{=} (y+a)+b = 0 + b \overset{A4}{=}b$$
$$y+d=y+(a+c) \overset{A2}{=} (y+a)+c = 0 + c \overset{A4}{=}c$$
$$b=c$$
\end{prf}

\begin{prf}[Unicidad del Elemento Neutro de la suma]{}
Supongamos que 0' es un numero que tambien funciona como neutro de la suma, entonces
$$a+0=a \land a+0'=a$$
$$a+0=a+0'$$
Y por propiedad cancelativa de la suma
$$0=0'$$
\end{prf}

\begin{prf}[Unicidad del Elemento Opuesto]{}
La existencia de un numero b esta dada por el axioma 5, hay que demostrar que es unico. Suponiendo que existe $b'$ / $a+b'=b'+a=0$, tenemos que
$$a+b=0 \land a+b'=0$$
$$a+b = a+b'$$
Y por propiedad cancelativa de la suma
$$b=b'$$
\end{prf}

\begin{prf}[$-(-a) = a$]{}
Sea $b$ el opuesto de $a$, se puede concluir que $a+b=0 \land b=(-a) \land a=(-b)$\hfill$(1)\land(2)\land(3)$
$$-(-a)\equals{(2)}-b\equals{(3)}a$$
\end{prf}

\begin{prf}[$-0 = 0$]{}
Por el axioma 5, todo numero real tiene su opuesto. Llamemos 0' al opuesto de 0, siendo $0+0'=0$ y\\
Del axioma 3 se concluye que $0+0=0$
$$si\ 0+0'=0 \land 0+0=0 \Rightarrow 0'=0$$
\end{prf}

\begin{prf}[$0.a = 0$]{}
	$$a.0 \equals{A4} a.0+0 \equals{A5} a.0+(a+(-a)) \equals{A2} (a.0+a)+(-a) \equals{A4} (a.0+a.1)+(-a) \equals{A3}$$
	$$a(0+1)+(-a) \equals{A4} a.1+(-a) \equals{A4} a+(-a) \equals{A5} 0$$
\end{prf}

\begin{prf}[$a(-b)=-(ab)=(-a)b$]{}
	$$
	a(-b) \equals{A4}
	a(-b)+0 \equals{A5}
	a(-b)+(ab+-(ab)) \equals{A2}
	(a(-b)+ab)+-(ab) \equals{A3}$$$$
	a((-b)+b)+-(ab) \equals{A5}
	a.0+-(ab) \equals{T2.3}
	0+-(ab) \equals{A4}
	-(ab)
	$$

	Y análogamente

	$$
	(-a)b \equals{A4}
	(-a)b+0 \equals{A5}
	(-a)b+(ab+-(ab)) \equals{A2}
	((-a)b+ab)+-(ab) \equals{A3}$$$$
	b((-a)+a)+-(ab) \equals{A5}
	b.0+-(ab) \equals{T2.3}
	0+-(ab) \equals{A4}
	-(ab)
	$$

	Reescribiendo

	$$a(-b)=-(ab)=(-a)b$$
\end{prf}

\begin{prf}[$(-a)(-b)=ab$]{}
	Analizamos la expresión $(-a)(-b)$, llamemos $c = -b$.
	Por el teorema anterior obtenemos:
	$$(-a)c = -(ac)$$

	Pero reemplzando por nuestra definición de $c = -b$, queda:
	$$-(ac) = -(a(-b))$$

	Que por la aplicación del mismo teorema nos da:
	$$-(a(-b)) = -(-(ab))$$

	Y finalmente por el teorema $-(-a) = a$:
	$$-(-(ab)) = ab$$

	Reescribiendo:
	$$(-a)(-b) = ab$$

\end{prf}

\begin{prf}[$a(b-c)=ab-ac$]{}
	Por la definicion de diferencia, se puede reescribir como:
	$$
	a(b+(-c)) \equals{A3}
	ab+a(-c) \equals{T2.4}
	ab+-(ac)
	$$
	Que por la definicion de diferencia, se puede reescribir como: $ab-ac$
\end{prf}

\begin{prf}[Propiedad cancelativa del producto]{}
	Analicemos $ab = ac$, llamemos $d = ab = ba$ y además $d = ac = ca$:
	$$
	da^{-1} \equals{Def}
	(ba)a^{-1} \equals{Asoc}
	b(aa^{-1}) \equals{Asoc}
	b.1 \equals{Neutro}
	b
	$$

	Y análogamente:
	$$
	da^{-1} \equals{Def}
	(ca)a^{-1} \equals{Asoc}
	c(aa^{-1}) \equals{Recip}
	c.1 \equals{Neutro}
	c
	$$

	Reescribiendo:
	$$b = c$$
\end{prf}

\begin{prf}[Unidad del elemento neutro del producto]{}
	Sabemos que existe $1$, tal que $\forall a, a.1 = a$, supongamos que existe $1'$ que cumple lo mismo, entonces:
	$$a.1 = a \land a.1' = a$$
	Entonces:
	$$a.1 = a.1'$$
	Y por el teorema anterior:
	$$1 = 1'$$
\end{prf}

\begin{prf}[Unidad del elemento recíproco]{}
	Sabemos que $\forall a \exists b \in \R / ab = 1$, supongamos que existe $b'$ que cumple lo mismo, entonces:
	$$a.b = 1 \land a.b' = 1$$
	Entonces:
	$$a.b = a.b'$$
	Y por el teorema anterior:
	$$b = b'$$
\end{prf}

\begin{prf}[$\nexists 0^{-1}$]{}
	Asumimos $\exists 0^{-1} \in \R$ tal que
	$$0.0^{-1} = 1$$

	Pero por $a.0 = 0.a = 0, \forall a \in \R$
	$$0.0^{-1} = 0$$

	Esto es una contradicción a lo supuesto.
	$$\therefore \nexists 0^{-1}$$
\end{prf}

\begin{prf}[$1^{-1} = 1$]{}

		\lreqn{1^{-1} =}{<Existencia del elemento neutro del producto>}
		\lreqn{1.1^{-1} =}{<Existencia del elemento recíproco>}
		\lreqn{1}{}
\end{prf}

\begin{prf}[$\frac{a}{1} = a; a \neq 0, \frac{1}{a} = a^{-1}$]{}
	Analizamos $\frac{a}{1}$\\

	\lreqn{\frac{a}{1} =}{<Definición de cociente>}
	\lreqn{a.1^{-1} =}{<$1^{-1} = 1$>}
	\lreqn{a.1 =}{<Elemento neutro del producto>}
	\lreqn{a}
	Analizando $\frac{1}{a}$ cuando $a \neq 0$\\

	\lreqn{\frac{1}{a} =}{<Definición de cociente>}
	\lreqn{1.a^{-1} =}{<Elemento neutro del producto>}
	\lreqn{a^{-1}}{}
\end{prf}

\begin{prf}[$ab = 0 \Rightarrow a = 0 \lor b = 0$]{}
	Hay dos casos posibles para la expresión $ab = 0$:\\

	\lreqn{ab = 0 \Rightarrow}{<$a = b \Rightarrow ac = bc$>}
	\lreqn{ab.b^{-1} = 0b^{-1} \Rightarrow}{<$a.0 = 0$>}
	\lreqn{ab.b^{-1} = 0 \Rightarrow}{<Propiedad asociativa>}
	\lreqn{a.(bb^{-1}) = 0 \Rightarrow}{<Existencia del elemento recíproco>}
	\lreqn{a.1 = 0 \Rightarrow}{<Existencia del elemento neutro del producto>}
	\lreqn{a = 0}{}

	\lreqn{ab = 0 \Rightarrow}{<$a = b \Rightarrow ca = cb$>}
	\lreqn{a^{-1}.ab = b^{-1}0 \Rightarrow}{<$0.a = 0$>}
	\lreqn{a^{-1}.ab = 0 \Rightarrow}{<Propiedad asociativa>}
	\lreqn{(a^{-1}a).b = 0 \Rightarrow}{<Existencia del elemento recíproco>}
	\lreqn{1.b = 0 \Rightarrow}{<Existencia del elemento neutro del producto>}
	\lreqn{b = 0}{}
	Como las dos afirmaciones son válidas:
	$$ab = 0 \Rightarrow a = 0 \lor b = 0$$
\end{prf}

\begin{prf}[$b \neq 0 \land d \neq 0 \Rightarrow (bd)^{-1} = b^{-1}.d^{-1}$]{}
	Analizamos la expresión $1 = 1$ que, por existencia del elemento neutro del producto, resulta ser equivalente a:
	$$1 = 1.1$$
	Observamos 3 cosas, por existencia y unicidad del elemento recíproco:
	$$bc.(bc)^{-1} = 1$$
	$$b.b^{-1} = 1$$
	$$c.c^{-1} = 1$$
	Y reemplazando en la expresión $1 = 1.1$:
	$$bc.(bc)^{-1} = (b.b^{-1}).(c.c^{-1})$$
	Que por propiedad asociativa y conmutativa del producto, reescribimos como:
	$$bc.(bc)^{-1} = bc.(b^{-1}.c^{-1})$$
	Y finalmente, por cancelativa del producto, obtenemos:
	$$(bc)^{-1} = b^{-1}.c^{-1}$$
\end{prf}

\begin{prf}[$b \neq 0 \land d \neq 0 \Rightarrow \frac{a}{b}+\frac{c}{d} = \frac{ad+cb}{bd}$]{}
	
	\lreqn{\frac{a}{b}+\frac{c}{d} =}{<Definición de cociente>}
	\lreqn{ab^{-1}+cd^{-1} =}{<Existencia del elemento neutro del producto>}
	\lreqn{1.ab^{-1}+1.cd^{-1} =}{<Existencia del elemento recíproco>}
	\lreqn{(dd^{-1}).(ab^{-1})+(bb^{-1}).(cd^{-1}) =}{<Reescribiendo con propiedad conmutativa y asociativa>}
	\lreqn{(ad).(b^{-1}d^{-1})+(cb).(b^{-1}d^{-1}) =}{<$(ab)^{-1} = a^{-1}b^{-1}$>}
	\lreqn{(ad).(bd)^{-1}+(cb).(bd)^{-1} =}{<Propiedad distributiva>}
	\lreqn{(ad+cb).(bd)^{-1} =}{<Definición de cociente>}
	\lreqn{\frac{ad+cb}{bd}}{}
	$$\therefore \frac{a}{b}+\frac{c}{d} = \frac{ad+cb}{bd}$$
\end{prf}

\begin{prf}[$b \neq 0 \land d \neq 0 \Rightarrow \frac{a}{b}.\frac{c}{d} = \frac{ac}{bd}$]{}
	
	\lreqn{\frac{a}{b}.\frac{c}{d} =}{<Definición de cociente>}
	\lreqn{(ab^{-1}).(cd^{-1}) =}{<Reescribniendo con propiedad conmutativa y asociativa>}
	\lreqn{(ac).(b^{-1}d^{-1}) =}{<$(ab)^{-1} = a^{-1}b^{-1}$>}
	\lreqn{(ac).(bd)^{-1} =}{<Definición de cociente>}
	\lreqn{\frac{ac}{bd}}{}
	$$\therefore \frac{a}{b}.\frac{c}{d} = \frac{ac}{bd}$$
\end{prf}

\begin{prf}[$a \neq 0 \land b \neq 0 \Rightarrow (\frac{a}{b})^{-1} = \frac{a^{-1}}{b^{-1}}$]{}

	\lreqn{\left(\frac{a}{b}\right)^{-1}}{<Definición de cociente>}
	\lreqn{\left(ab^{-1}\right)^{-1}}{<$(ab)^{-1} = a^{-1}.b^{-1}$>}
	\lreqn{a^{-1}\left(b^{-1}\right)^{-1}}{<Definición de cociente>}
	\lreqn{\frac{a^{-1}}{b^{-1}}}{}
\end{prf}


\begin{prf}[$\left(-1\right).a = -a$]{}

	\lreqn{-1.a =}{<Existencia del elemento neutro de la suma>}
	\lreqn{-1.a + 0 =}{<Existencia del elemento opuesto>}
	\lreqn{-1.a + (a + -a) =}{<Propiedad asociativa de la suma>}
	\lreqn{(-1.a + a) + -a =}{<Existencia del elemento neutro de la multiplicacion>}
	\lreqn{(-1.a + 1.a) + -a =}{<Propiedad distrubutiva>}
	\lreqn{a.(-1 + 1) + -a =}{<Existencia del elemento opuesto>}
	\lreqn{a.0 + -a =}{<$a.0 = 0$>}
	\lreqn{0 + -a =}{<Existencia del elemento neutro de la suma>}
	\lreqn{-a}{}
\end{prf}

Suponemos $\R^+ \subset \R$, tal que cumple:
\begin{itemize}
	\item $a, b \in \R^+ \Rightarrow a+b, ab \in \R^+$
	\item $\forall a \neq 0 \in \R, a \in \R^+ \veebar -a \in \R^+$
	\item $0 \notin \R^+$
\end{itemize}
Llamamos a estos números "positivos".
Definimos $<, >, \geq, \leq$ de la forma que está en el apunte.

\begin{prf}[$a > 0 \Leftrightarrow a \in \R^+$]{}
	Sea $a > 0, a \in \R$:\\

	\lreqn{a > 0}{<Definición de $<$>}
	\lreqn{a - 0 \in \R^+}{<Definición de resta>}
	\lreqn{a + (-0) \in \R^+}{<$0 = -0$>}
	\lreqn{a + 0 \in \R^+}{<Elemento neutro de la suma>}
	\lreqn{a \in \R^+}{<Elemento neutro de la suma>}
	Sea $a \in \R^+$:\\

	\lreqn{a \in \R^+}{<Elemento neutro de la suma>}
	\lreqn{a + 0 \in \R^+}{<$0 = -0$>}
	\lreqn{a + (-0) \in \R^+}{<Definición de resta>}
	\lreqn{a - 0 \in \R^+}{<Definición de $<$>}
	\lreqn{a > 0}{}

	$$\therefore a > 0 \Leftrightarrow a \in \R^+$$
\end{prf}

\begin{prf}[$a > 0 \Leftrightarrow -a < 0$]{}
	Sea $a > 0, a \in \R$:\\

	\lreqn{a > 0}{<$a > 0 \Leftrightarrow a \in \R^+$>}
	\lreqn{a \in \R^+}{<$a = -(-a)$>}
	\lreqn{-(-a) \in \R^+}{<Elemento neutro de la suma>}
	\lreqn{0 + -(-a) \in \R^+}{<Definición de resta>}
	\lreqn{0 - (-a) \in \R^+}{<Definición de $<$>}
	\lreqn{-a < 0}{<Definición de $<$>}
\end{prf}

\begin{prf}[$a < 0 \Leftrightarrow -a > 0$]{}
	Sea $a < 0, a \in \R$:\\

	\lreqn{a < 0}{<Definición de $<$>}
	\lreqn{0 - a \in \R^+}{<Definición de resta>}
	\lreqn{0 + (-a) \in \R^+}{<Elemento neutro de la suma>}
	\lreqn{-a \in \R^+}{<$a \in \R^+ \Leftrightarrow a > 0$>}
	\lreqn{-a > 0}{}
\end{prf}

Como llamamos a los números en $\R^+$ positivos, a sus opuestos los llamaremos "negativos". Además Si $a \geq 0$, es "no negativo".

\begin{prf}[Propiedad de Tricotomía]{}
	Para demostrar proposiciones mutuamente excluyentes, optaremos por probar que la ocurrencia de una implica la no ocurrencia de las otras, para todo posible caso.\\
	Sean $a,b \in \R$:\\

	\begin{itemize}
		\item Caso 1, $a < b$ o sea $b - a \in \R^+$:\\
		Supongamos que además $a = b$, entonces
		$$b - a = 0$$
		pero por axioma
		$$0 \notin \R^+$$
		Contradicción.\\
		Supongamos que además $a > b$, entonces
		$$a - b \in \R^+$$
		pero
		$$b - a = -(a - b)$$
		entonces por axioma
		$$-(a - b) \notin \R^+$$
		Contradicción.\\
		\item Caso 2, $a = b$ o sea $b - a = 0$:\\
		Supongamos que además $a < b$, entonces
		$$b - a \in \R^+$$
		pero por axioma
		$$0 \notin \R^+$$
		Contradicción.\\
		Supongamos que además $a > b$, entonces
		$$a - b \in \R^+$$
		pero
		$$b - a = -(a - b)$$
		y por axioma
		$$0 = -0 \notin \R^+$$
		Contradicción.\\
		\item Caso 3, $a > b$ o sea $a - b \in R^+$:\\
		Supongamos que además $a = b$, entonces
		$$a - b = 0$$
		pero por axioma
		$$0 \notin \R^+$$
		Contradicción.\\
		Supongamos que además $a < b$, entonces
		$$b - a \in \R^+$$
		pero
		$$a - b = -(b - a)$$
		entonces por axioma
		$$-(b - a) \notin \R^+$$
		Contradicción.\\
	\end{itemize}
	$$\therefore a < b\hspace{2mm}\veebar\hspace{2mm} a = b \hspace{2mm}\veebar\hspace{2mm} a > b$$
\end{prf}

\begin{prf}[Propiedad Transitiva del Menor]{}
	Sean $a,b,c \in \R$, tal que $a < b$ y $b < c$:\\
	Observamos que $b - a, c - b \in \R^+$\\

	\lreqn{b - a \in \R^+}{<Axioma $a, b \in \R^+ \Rightarrow a+b \in \R^+$}
	\lreqn{(b - a) + (c - b) \in \R^+}{<Reescribiendo>}
	\lreqn{(c - a) + (b - b) \in \R^+}{<Existencia del opuesto>}
	\lreqn{(c - a) + 0 \in \R^+}{<Elemento neutro de la suma>}
	\lreqn{c - a \in \R^+}{<Definición de $<$>}
	\lreqn{a < c}{<Definición de $<$>}
\end{prf}

\begin{prf}[$a<b \Rightarrow a+c<b+c$]{}
	Sean $a,b,c \in \R$:\\

	\lreqn{a < b}{Def <}
	\lreqn{b - a \in \R^+}{<Elemento neutro de la suma>}
	\lreqn{(b - a) + 0 \in \R^+}{<Existencia del opuesto>}
	\lreqn{(b - a) + (c + -c) \in \R^+}{<Definición de $<$>}
	\lreqn{(b + -a) + (c + -c) \in \R^+}{<Reescribiendo>}
	\lreqn{(b + c) + (-a + -c) \in \R^+}{<$-(a+b)=(-a)+(-b)$}
	\lreqn{(b + c) + -(a + c) \in \R^+}{<Definición de resta>}
	\lreqn{(b + c) - (a + c) \in \R^+}{Definición de $<$>}
	\lreqn{a + c < b + c}{}
\end{prf}

\begin{prf}[$a<b \land c<d \Rightarrow a+c<b+d$]{}
	Sean $a,b,c,d \in \R$\\
	$a<b$ o sea $b-a \in \R^+$, $c<d$ o sea $d-c \in \R^+$:\\

	\lreqn{a < b}{<Definición de $<$>}
	\lreqn{b - a \in \R^+}{<Axioma $a,b \in \R^+ \Rightarrow a+b \in \R^+$}
	\lreqn{(b - a) + (d - c) \in \R^+}{<Reescribiendo>}
	\lreqn{(b+d) - a - c \in \R^+}{<$-a-b = -(a+b)$>}
	\lreqn{(b+d) - (a+c) \in \R^+}{<Definición de $<$>}
	\lreqn{a+c < b+d}{}
\end{prf}

\begin{prf}[$a<b \land c>0 \Rightarrow ac<bc$]{}
	Sean $a,b,c \in \R, c>0$:\\
	Observamos $c>0 \Rightarrow c \in \R^+$\\

	\lreqn{a < b}{<Definición de $<$>}
	\lreqn{b - a \in \R^+}{<Axioma $a,b \in \R^+ \Rightarrow ab \in \R^+$}
	\lreqn{(b - a)c \in \R^+}{<Propiedad distributiva>}
	\lreqn{bc - ac \in \R^+}{<Definición de $<$>}
	\lreqn{ac < bc}{}
\end{prf}

\begin{prf}[$a<b \land c<0 \Rightarrow ac>bc$]{}
	Sean $a,b,c \in \R, c<0$:\\
	Observamos $c<0 \Rightarrow -c \in \R^+$\\

	\lreqn{a < b}{<Definición de $<$>}
	\lreqn{b - a \in \R^+}{<Axioma $a,b \in \R^+ \Rightarrow ab \in \R^+$}
	\lreqn{(b - a)(-c) \in \R^+}{<$a = (-1).a$>}
	\lreqn{(b - a)c.(-1) \in \R^+}{<Propiedad distributiva>}
	\lreqn{(bc - ac).-1 \in \R^+}{<$a = (-1).a$>}
	\lreqn{-(bc - ac) \in \R^+}{<$-(a-b) = b-a$>}
	\lreqn{ac - bc \in \R^+}{<Definición de $>$>}
	\lreqn{ac > bc}{}
\end{prf}

\begin{prf}[$a \neq 0 \Rightarrow a^2 > 0$]{}
	Sea $a \in \R, a \neq 0$
	Por la propiedad tricotómica, $a \neq 0\ \veebar\ a > 0\ \veebar\ a < 0$.
	Pero $a \neq 0$, entonces $a > 0\ \veebar\ a < 0$.\\

	\begin{itemize}
		\item Caso 1, $a > 0$ o sea $a \in \R^+$:\\
		Por axioma $a,b \in \R^+ \Rightarrow ab \in \R^+$
		$$aa \in \R^+ \Leftrightarrow a^2 \in \R^+$$
		\item Caso 2, $a < 0$ o sea $-a \in \R^+$:\\
		Por axioma $a,b \in \R^+ \Rightarrow ab \in \R^+$
		$$(-a)(-a) \in \R^+$$
		Que por $(-a)(-a) = aa$
		$$aa \in \R^+ \Leftrightarrow a^2 \in \R^+$$
	\end{itemize}
\end{prf}

\begin{prf}[$1 \in \R^+$]{}
	Existen neutros, 0 y 1, $0 \neq 1$\\
	Por axioma $1 \in \R^+\ \veebar\ -1 \in \R^+$.
	Supongo $-1 \in \R^+$, entonces $1 \notin \R^+$\\
	\lreqn{-1 \in \R^+}{<Axioma $a,b \in \R^+ \Leftarrow ab \in \R^+$}
	\lreqn{(-1).(-1) \in \R^+}{<$aa = (-a)(-a)$>}
	\lreqn{1 \in \R^+}{}
	Pero esto es una contradicción con lo supuesto.
\end{prf}

\begin{prf}[$a < b \Leftarrow -b < -a$]{}
	Sean $a,b \in \R, a < b$:\\

	\lreqn{a < b}{<Definición de $<$}
	\lreqn{b - a \in \R^+}{<Definición de resta>}
	\lreqn{b + (-a) \in \R^+}{<$a = -(-a)$>}
	\lreqn{-(-b) + (-a) \in \R^+}{<Conmutativa>}
	\lreqn{(-a) + -(-b) \in \R^+}{<Definición de resta>}
	\lreqn{(-a) - (-b) \in \R^+}{<Definición de $<$>}
	\lreqn{-b < -a}{}
\end{prf}

\begin{prf}[$ab > 0 \Leftrightarrow a,b \in \R^+ \lor -a,-b \in \R^+$]{}
	Sean $a,b \in \R, ab > 0$:\\
	Asumo $a \in \R^+, -b \in \R^+$. Por axioma
	$$ a(-b) \in \R^+$$
	$$ -ab \in \R^+$$
	Pero como $ab \in \R^+$ entonces no puede ser $-ab \in \R^+$.
	Esto es una contradicción con la suposición de que $a \in \R^+, -b \in \R^+$.\\
	Analogamente se puede demostrar para el caso $b \in \R^+, -a \in \R^+$. Por lo tanto

	$$ab > 0 \Rightarrow a,b \in \R^+ \lor -a,-b \in \R^+$$
	Sean $a,b \in \R^+$

	\lreqn{a,b \in \R^+}{<Producto cerrado en $\R^+$>}
	\lreqn{ab \in \R^+}{<$a > 0 \Leftrightarrow a \in \R^+$>}
	\lreqn{ab > 0}{}
	Sean $-a,-b \in \R^+$

	\lreqn{-a,-b \in \R^+}{<Producto cerrado en $\R^+$>}
	\lreqn{(-a)(-b) \in \R^+}{<$a > 0 \Leftrightarrow a \in \R^+$>}
	\lreqn{(-a)(-b) > 0}{<$ab = (-a)(-b)$}
	\lreqn{ab > 0}{}
	Por lo tanto
	$$a,b \in \R^+ \lor -a,-b \in \R^+ \Rightarrow ab > 0$$
	Finalmente
	$$\therefore ab > 0 \Leftrightarrow a,b \in \R^+ \lor -a,-b \in \R^+$$
\end{prf}

\begin{prf}[$ab < 0 \Leftrightarrow a,-b \in \R^+ \lor -a,b \in \R^+$]{}
	Sean $a,b \in \R, ab < 0$:\\
	$ab < 0$ implica que $-ab \in \R^+$\\
	Asumo $a \in \R^+, b \in \R^+$. Por axioma
	$$ab \in \R^+$$
	Pero como $-ab \in \R^+$ entonces no puede ser $ab \in \R^+$.
	Esto es una contradicción con la suposición de que $a \in \R^+, b \in \R^+$.\\
	Asumo $-a \in \R^+, -b \in \R^+$. Por axioma
	$$(-a)(-b) \in \R^+$$
	Por $ab = (-a)(-b)$
	$$ab \in \R^+$$
	Pero como $-ab \in \R^+$ entonces no puede ser $ab \in \R^+$.
	Esto es una contradicción con la suposición de que $-a \in \R^+, -b \in \R^+$.\\

	$$ab < 0 \Rightarrow a,-b \in \R^+ \lor -a,b \in \R^+$$
	Sean $a,-b \in \R^+$

	\lreqn{a,-b \in \R^+}{<Producto cerrado en $\R^+$>}
	\lreqn{a(-b) \in \R^+}{<$-ab = a(-b)$>}
	\lreqn{-ab \in \R^+}{<Elemento neutro de la suma>}
	\lreqn{0 + -ab \in \R^+}{<Definición de resta>}
	\lreqn{0 - ab \in \R^+}{<Definición de $<$>}
	\lreqn{ab < 0}{}
	Análogamente se demuestra para el caso $-a,b$, pues $(-a)b = a(-b)$.
	Por lo tanto
	$$a,-b \in \R^+ \lor a,-b \in \R^+ \Rightarrow ab < 0$$
	Finalmente
	$$\therefore ab < 0 \Leftrightarrow a,-b \in \R^+ \lor a,-b \in \R^+$$
\end{prf}

\begin{prf}[$a > 0 \Leftrightarrow \dfrac{1}{a} > 0$]{}
	Sea $a \in \R, a > 0$:\\
	Suponemos $\dfrac{1}{a} < 0$\\

	\lreqn{\dfrac{1}{a} < 0}{<Definición de $<$ y de cociente>}
	\lreqn{0 - a^{-1} \in \R^+}{<Definición de resta>}
	\lreqn{0 + -a^{-1} \in \R^+}{<Elemento neutro de la suma>}
	\lreqn{-a^{-1} \in \R^+}{<Producto cerrado en $\R^+$>}
	\lreqn{-a^{-1}a \in \R^+}{<$-a = (-1).a$>}
	\lreqn{(-1).a^{-1}a \in \R^+}{<Existencia del recíproco>}
	\lreqn{(-1).1 \in \R^+}{<Elemento neutro del producto>}
	\lreqn{-1 \in \R^+}{}
	Pero $1 \in \R^+$, entonces por axioma, no puede ser $-1 \in \R^+$.
	Esto es una contradicción.

	$$\therefore a > 0 \Rightarrow \dfrac{1}{a} > 0$$
	Sea $a \in \R, a \neq 0, \dfrac{1}{a} > 0$:\\
	Suponemos $a < 0$\\

	\lreqn{a < 0}{<$a < 0 \Rightarrow -a > 0$>}
	\lreqn{-a > 0}{<$a < b \land c > 0 \Rightarrow ac < bc$>}
	\lreqn{-a.\dfrac{1}{a} > 0.\dfrac{1}{a}}{<$a.0 = 0$ y Definición de cociente>}
	\lreqn{-aa^{-1} > 0}{<$-a = (-1).a$>}
	\lreqn{(-1).aa^{-1} > 0}{<Existencia del recíproco>}
	\lreqn{(-1).1 > 0}{<Elemento neutro del producto>}
	\lreqn{(-1) > 0}{<$a < 0 \Rightarrow -a > 0$>}
	\lreqn{1 < 0}{<Definición de $<$>}
	Pero $1 > 0$. Esto es una contradicción. Y por propiedad tricotómica y $a \neq 0$

	$$\therefore \dfrac{1}{a} > 0 \Rightarrow a > 0$$

	$$\therefore a > 0 \Leftrightarrow \dfrac{1}{a} > 0$$
\end{prf}

\begin{prf}[$0 < a < b \Rightarrow 0 < \dfrac{1}{b} < \dfrac{1}{a}$]{}
	Sean $a,b \in \R, 0 < a < b$:\\
	Por la demostración anterior
	$$a > 0 \Leftrightarrow \dfrac{1}{a} > 0$$
	Por propiedad transitiva de la relación de menor, $0 < b$, entonces
	$$b > 0 \Leftrightarrow \dfrac{1}{b} > 0$$
	Analicemos $a < b$\\
	\lreqn{a < b}{<Definición de $<$>}
	\lreqn{b - a \in \R^+}{<Producto cerrado en $\R^+$>}
	\lreqn{(b - a)b^{-1}a^{-1} \in \R^+}{<Distributiva>}
	\lreqn{bb^{-1}a^{-1} - ab^{-1}a^{-1} \in \R^+}{<Existencia del elemento recíproco>}
	\lreqn{1.a^{-1} - 1.b^{-1} \in \R^+}{<Elemento neutro del producto>}
	\lreqn{a^{-1} - b^{-1} \in \R^+}{<Definición de $<$>}
	\lreqn{b^{-1} < a^{-1}}{<Definición de cociente>}
	\lreqn{\dfrac{1}{b} < \dfrac{1}{a}}{}
	$$0 < a < b \Rightarrow 0 < \dfrac{1}{b} < \dfrac{1}{a}$$
\end{prf}

\begin{prf}[$|x| \geq 0, |x| = 0 \Leftrightarrow x = 0$]{}
	Por triconomía pruebo los 3 casos y me dan.

	La ida por contradicción, la vuelta directa.
\end{prf}

\begin{prf}[$|x| = |-x|$]{}
	Demostración directa
\end{prf}

\begin{prf}[$-|x| \leq x \leq |x|$]{}
	Sea $x \geq 0$:\\
	\lreqn{-x \leq x \leq x}{<Definición de $|x|$>}
	\lreqn{-|x| \leq x \leq |x|}{}
	Sea $x < 0$:\\
	\lreqn{x \leq x \leq -x}{<$a = -(-a)$>}
	\lreqn{-(-x) \leq x \leq -x}{<Definición de $|x|$>}
	\lreqn{-|x| \leq x \leq |x|}{}
\end{prf}

\begin{prf}[$|x| < a \Leftrightarrow -a < x < a$]{}
	Sea $|x| < a$:\\
	Caso $x \geq 0$:\\
	\lreqn{|x| < a}{}
	\lreqn{x < a}{}
	\lreqn{0 \leq x < a}{}
	\lreqn{-a < 0 \leq x < a}{}
	\lreqn{-a < x < a}{}
	Caso $x < 0$:\\
	\lreqn{|x| < a}{}
	\lreqn{-x < a}{}
	\lreqn{x > -a}{}
	\lreqn{-a < x}{}
	\lreqn{-a < x < 0}{}
	\lreqn{-a < x < 0 \leq a}{}
	\lreqn{-a < x < a}{}

	Sea $-a < x < a$:\\
	Caso $x \geq 0$:\\
	\lreqn{x < a}{}
	\lreqn{|x| < a}{}
	Caso $x < 0$:\\
	\lreqn{-a < x}{}
	\lreqn{-x < a}{}
	\lreqn{|x| < a}{}
\end{prf}

\begin{prf}[$|x| > a \Leftrightarrow x < -a \lor x > a$]{}
	Analogamente
\end{prf}

\begin{prf}[Desigualdad triangular]{}
	Se demuestra por casos
\end{prf}

\begin{prf}[producto y cociente]{}
	Se demuestra por casos? O contradiccion?
\end{prf}

\begin{prf}[$\forall y \in \R, \exists n \in \N : y < n$]{}
	
\end{prf}

\begin{prf}[$x > 0 \Rightarrow \exists n \in \N : \dfrac{1}{n} < x$]{}
	Por la propiedad arquimedeana, $\forall y \in \R, \exists x > 0$, tal que:
	$$y < nx$$
	Pero $1 \in \R$
	$$1 < nx$$
	$$\dfrac{1}{n} < x$$
\end{prf}

\begin{prf}[ndea]{}
	Supongo $x \neq y$:\\
	\lreqn{x < y < x + z/n}{<Sumo $-x$ y llamo $y - x = y'$>}
	\lreqn{0 < y' < z/n}{<Multiplico por $n$>}
	\lreqn{0 < ny' < z}{}
	\lreqn{ny' < z}{}
	Esto es una contradicción a la propiedad arquimedeana.
	$$\therefore x = y$$
\end{prf}
\end{document}
