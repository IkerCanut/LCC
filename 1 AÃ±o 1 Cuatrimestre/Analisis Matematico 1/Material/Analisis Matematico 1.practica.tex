\documentclass[10pt]{article}
\usepackage{hyperref}
\hypersetup{
    colorlinks=true,
    linkcolor=black,
    filecolor=magenta,      
    urlcolor=cyan,
}

\usepackage{import}
\usepackage{siunitx}
\usepackage{esvect}
\usepackage{fourier}
\usepackage{amssymb}
\usepackage{amsmath}
\usepackage{multicol}
\usepackage{geometry}
\usepackage[framemethod=TikZ]{mdframed}
\geometry{
 a4paper,
 total={160mm,237mm},
 left=30mm,
 top=30mm,
}

\usepackage{tikz}
\usetikzlibrary{automata, positioning, calc, through, angles, quotes, intersections}
\usepackage{multirow}

\subimport{.}{environment}

\author{Iker M. Canut}
\begin{document}
\title{Introducción a la Matemática}
\maketitle
\date
\newpage

\tableofcontents
\newpage

\section{Unidad 1: Numeros Reales}

\begin{prf}[Demostracion de la Propiedad Cancelativa de la Suma]{}
Sea $d=a+b$, y por ende, $d=b+c$, por el Axioma 5, existe $y$ que es opuesto  a $a$, entonces:
$$y+d=y+(a+b) \overset{A2}{=} (y+a)+b = 0 + b \overset{A4}{=}b$$
$$y+d=y+(a+c) \overset{A2}{=} (y+a)+c = 0 + c \overset{A4}{=}c$$
$$b=c$$
\end{prf}

\begin{prf}[Demostracion de la Unicidad del Elemento Neutro de la suma]{}
Supongamos que 0' es un numero que tambien funciona como neutro de la suma, entonces
$$a+0=a \land a+0'=a$$
$$a+0=a+0'$$
Y por propiedad cancelativa de la suma
$$0=0'$$
\end{prf}

\begin{prf}[Demostracion de la Unicidad del Elemento Opuesto]{}
La existencia de un numero b esta dada por el axioma 5, hay que demostrar que es unico. Suponiendo que existe $b'$ / $a+b'=b'+a=0$, tenemos que
$$a+b=0 \land a+b'=0$$
$$a+b = a+b'$$
Y por propiedad cancelativa de la suma
$$b=b'$$
\end{prf}

\begin{prf}[Demostracion de que el opuesto al opuesto de $a$ es $a$]{}
Sea $b$ el opuesto de $a$, se puede concluir que $a+b=0 \land b=(-a) \land a=(-b)$\hfill$(1)\land(2)\land(3)$
$$-(-a)\equals{(2)}-b\equals{(3)}a$$
\end{prf}

\begin{prf}[Demostracion de que el opuesto de 0 es 0]{}
Por el axioma 5, todo numero real tiene su opuesto. Llamemos 0' al opuesto de 0, siendo $0+0'=0$ y\\
Del axioma 3 se concluye que $0+0=0$
$$si\ 0+0'=0 \land 0+0=0 \Rightarrow 0'=0$$
\end{prf}

\begin{prf}[Demostracion de que el producto de 0 con cualquier otro numero es 0]{}
$$a.0 \equals{A4} a.0+0 \equals{A5} a.0+(a+(-a)) \equals{A2} (a.0+a)+(-a) \equals{A4} (a.0+a.1)+(-a) \equals{A3}$$
$$a(0+1)+(-a) \equals{A4} a.1+(-a) \equals{A4} a+(-a) \equals{A5} 0$$
\end{prf}

\begin{prf}[$a(-b)=-(ab)=(-a)b$]{}
$$
a(-b) \equals{A4}
a(-b)+0 \equals{A5}
a(-b)+(ab+-(ab)) \equals{A2}
(a(-b)+ab)+-(ab) \equals{A3}$$$$
a((-b)+b)+-(ab) \equals{A5}
a.0+-(ab) \equals{T2.3}
0+-(ab) \equals{A4}
-(ab)
$$

Y análogamente

$$
(-a)b \equals{A4}
(-a)b+0 \equals{A5}
(-a)b+(ab+-(ab)) \equals{A2}
((-a)b+ab)+-(ab) \equals{A3}$$$$
b((-a)+a)+-(ab) \equals{A5}
b.0+-(ab) \equals{T2.3}
0+-(ab) \equals{A4}
-(ab)
$$

Reescribiendo

$$a(-b)=-(ab)=(-a)b$$
\end{prf}

\begin{prf}[$(-a)(-b)=ab$]{}
Analizamos la expresión $(-a)(-b)$, llamemos $c = -b$.
Por el teorema anterior obtenemos:
$$(-a)c = -(ac)$$

Pero reemplzando por nuestra definición de $c = -b$, queda:
$$-(ac) = -(a(-b))$$

Que por la aplicación del mismo teorema nos da:
$$-(a(-b)) = -(-(ab))$$

Y finalmente por el teorema $-(-a) = a$:
$$-(-(ab)) = ab$$

Reescribiendo:
$$(-a)(-b) = ab$$

\end{prf}

\begin{prf}[$a(b-c)=ab-ac$]{}
Por la definicion de diferencia, se puede reescribir como:
$$
a(b+(-c)) \equals{A3}
ab+a(-c) \equals{T2.4}
ab+-(ac)
$$
Que por la definicion de diferencia, se puede reescribir como: $ab-ac$
\end{prf}

\begin{prf}[Propiedad cancelativa del producto]{}
Analicemos $ab = ac$, llamemos $d = ab = ba$ y además $d = ac = ca$:
$$
da^{-1} \equals{Def}
(ba)a^{-1} \equals{Asoc}
b(aa^{-1}) \equals{Asoc}
b.1 \equals{Neutro}
b
$$

Y análogamente:
$$
da^{-1} \equals{Def}
(ca)a^{-1} \equals{Asoc}
c(aa^{-1}) \equals{Recip}
c.1 \equals{Neutro}
c
$$

Reescribiendo:
$$b = c$$
\end{prf}

\begin{prf}[Unidad del elemento neutro del producto]{}
Sabemos que existe $1$, tal que $\forall a, a.1 = a$, supongamos que existe $1'$ que cumple lo mismo, entonces:
$$a.1 = a \land a.1' = a$$
Entonces:
$$a.1 = a.1'$$
Y por el teorema anterior:
$$1 = 1'$$
\end{prf}

\begin{prf}[Unidad del elemento recíproco]{}
Sabemos que $\forall a \exists b \in \mathbb{R} / ab = 1$, supongamos que existe $b'$ que cumple lo mismo, entonces:
$$a.b = 1 \land a.b' = 1$$
Entonces:
$$a.b = a.b'$$
Y por el teorema anterior:
$$b = b'$$
\end{prf}

\begin{prf}[$\nexists 0^{-1}$]{}
Asumimos $\exists 0^{-1} \in \mathbb{R}$ tal que
$$0.0^{-1} = 1$$

Pero por $a.0 = 0.a = 0, \forall a \in \mathbb{R}$
$$0.0^{-1} = 0$$

Esto es una contradicción a lo supuesto.
$$\therefore \nexists 0^{-1}$$
\end{prf}

\begin{prf}[$1^{-1} = 1$]{}

\lreqn{1^{-1} =}{<Existencia del elemento neutro del producto>}
\lreqn{1.1^{-1} =}{<Existencia del elemento recíproco>}
\lreqn{1}{}

\end{prf}

\begin{prf}[$\frac{a}{1} = a; a \neq 0, \frac{1}{a} = a^{-1}$]{}

Analizamos $\frac{a}{1}$

\lreqn{\frac{a}{1} =}{<Definición de cociente>}
\lreqn{a.1^{-1} =}{<$1^{-1} = 1$>}
\lreqn{a.1 =}{<Elemento neutro del producto>}
\lreqn{a}

Analizando $\frac{1}{a}$ cuando $a \neq 0$

\lreqn{\frac{1}{a} =}{<Definición de cociente>}
\lreqn{1.a^{-1} =}{<Elemento neutro del producto>}
\lreqn{a^{-1}}

\end{prf}

\begin{prf}[$ab = 0 \Rightarrow a = 0 \lor b = 0$]{}

Hay dos casos posibles para la expresión $ab = 0$:
\vspace{5mm}\\
\lreqn{ab = 0 \Rightarrow}{<$a = b \Rightarrow ac = bc$>}
\lreqn{ab.b^{-1} = 0b^{-1} \Rightarrow}{<$a.0 = 0$>}
\lreqn{ab.b^{-1} = 0 \Rightarrow}{<Propiedad asociativa>}
\lreqn{a.(bb^{-1}) = 0 \Rightarrow}{<Existencia del elemento recíproco>}
\lreqn{a.1 = 0 \Rightarrow}{<Existencia del elemento neutro del producto>}
\lreqn{a = 0}{}

\lreqn{ab = 0 \Rightarrow}{<$a = b \Rightarrow ca = cb$>}
\lreqn{a^{-1}.ab = b^{-1}0 \Rightarrow}{<$0.a = 0$>}
\lreqn{a^{-1}.ab = 0 \Rightarrow}{<Propiedad asociativa>}
\lreqn{(a^{-1}a).b = 0 \Rightarrow}{<Existencia del elemento recíproco>}
\lreqn{1.b = 0 \Rightarrow}{<Existencia del elemento neutro del producto>}
\lreqn{b = 0}{}

Como las dos afirmaciones son válidas:
$$ab = 0 \Rightarrow a = 0 \lor b = 0$$

\end{prf}

\begin{prf}[$b \neq 0 \land d \neq 0 \Rightarrow (bd)^{-1} = b^{-1}.d^{-1}$]{}

Analizamos la expresión $1 = 1$ que, por existencia del elemento neutro del producto, resulta ser equivalente a:
$$1 = 1.1$$
Observamos 3 cosas, por existencia y unicidad del elemento recíproco:
$$bc.(bc)^{-1} = 1$$
$$b.b^{-1} = 1$$
$$c.c^{-1} = 1$$
Y reemplazando en la expresión $1 = 1.1$:
$$bc.(bc)^{-1} = (b.b^{-1}).(c.c^{-1})$$
Que por propiedad asociativa y conmutativa del producto, reescribimos como:
$$bc.(bc)^{-1} = bc.(b^{-1}.c^{-1})$$
Y finalmente, por cancelativa del producto, obtenemos:
$$(bc)^{-1} = b^{-1}.c^{-1}$$
\end{prf}

\begin{prf}[$b \neq 0 \land d \neq 0 \Rightarrow \frac{a}{b}+\frac{c}{d} = \frac{ad+cb}{bd}$]{}
\vspace{5mm}
\lreqn{\frac{a}{b}+\frac{c}{d} =}{<Definición de cociente>}
\lreqn{ab^{-1}+cd^{-1} =}{<Existencia del elemento neutro del producto>}
\lreqn{1.ab^{-1}+1.cd^{-1} =}{<Existencia del elemento recíproco>}
\lreqn{(dd^{-1}).(ab^{-1})+(bb^{-1}).(cd^{-1}) =}{<Reescribiendo con propiedad conmutativa y asociativa>}
\lreqn{(ad).(b^{-1}d^{-1})+(cb).(b^{-1}d^{-1}) =}{<$(ab)^{-1} = a^{-1}b^{-1}$>}
\lreqn{(ad).(bd)^{-1}+(cb).(bd)^{-1} =}{<Propiedad distributiva>}
\lreqn{(ad+cb).(bd)^{-1} =}{<Definición de cociente>}
\lreqn{\frac{ad+cb}{bd}}{}
$$\therefore \frac{a}{b}+\frac{c}{d} = \frac{ad+cb}{bd}$$
\end{prf}

\begin{prf}[$b \neq 0 \land d \neq 0 \Rightarrow \frac{a}{b}.\frac{c}{d} = \frac{ac}{bd}$]{}
\vspace{5mm}
\lreqn{\frac{a}{b}.\frac{c}{d} =}{<Definición de cociente>}
\lreqn{(ab^{-1}).(cd^{-1}) =}{<Reescribniendo con propiedad conmutativa y asociativa>}
\lreqn{(ac).(b^{-1}d^{-1}) =}{<$(ab)^{-1} = a^{-1}b^{-1}$>}
\lreqn{(ac).(bd)^{-1} =}{<Definición de cociente>}
\lreqn{\frac{ac}{bd}}{}
$$\therefore \frac{a}{b}.\frac{c}{d} = \frac{ac}{bd}$$
\end{prf}

\begin{prf}[$a \neq 0 \land b \neq 0 \Rightarrow (\frac{a}{b})^{-1} = \frac{a^{-1}}{b^{-1}}$]{}

\lreqn{\left(\frac{a}{b}\right)^{-1}}{<Definición de cociente>}
\lreqn{\left(ab^{-1}\right)^{-1}}{<$(ab)^{-1} = a^{-1}.b^{-1}$>}
\lreqn{a^{-1}\left(b^{-1}\right)^{-1}}{<Definición de cociente>}
\lreqn{\frac{a^{-1}}{b^{-1}}}{}
\end{prf}


\begin{prf}[$\left(-1\right).a = -a$]{}

\lreqn{-1.a =}{<Existencia del elemento neutro de la suma>}
\lreqn{-1.a + 0 =}{<Existencia del elemento opuesto>}
\lreqn{-1.a + (a + -a) =}{<Propiedad asociativa de la suma>}
\lreqn{(-1.a + a) + -a =}{<Existencia del elemento neutro de la multiplicacion>}
\lreqn{(-1.a + 1.a) + -a =}{<Propiedad distrubutiva>}
\lreqn{a.(-1 + 1) + -a =}{<Existencia del elemento opuesto>}
\lreqn{a.0 + -a =}{<$a.0 = 0$>}
\lreqn{0 + -a =}{<Existencia del elemento neutro de la suma>}
\lreqn{-a}{}
\end{prf}

\begin{prf}[$a<b \Rightarrow a+c<b+c$]{}
$$
a < b \impliesbecause{Def <}
b - a \in \mathbb{R}^{+} \impliesbecause{A4}
(b - a) + 0 \in \mathbb{R}^{+} \impliesbecause{A5}
(b - a) + (c + -c) \in \mathbb{R}^{+} \impliesbecause{Def -}
(b + -a) + (c + -c) \in \mathbb{R}^{+}
$$

Reescribiendo usando $A1$ y $A2$:

$$
(b + c) + (-a + -c) \in \mathbb{R}^{+} \impliesbecause{T?}
(b + c) + -(a + c) \in \mathbb{R}^{+} \impliesbecause{Def -}
(b + c) - (a + c) \in \mathbb{R}^{+} \impliesbecause{Def <}
a + c < b + c
$$
\end{prf}

\begin{prf}[$a<b \land c>0 \Rightarrow ac<bc$]{}
Tenemos que:
$c > 0 \Rightarrow c \in \mathbb{R}^{+}$

Analizamos $a < b$:
$$
a < b \impliesbecause{Def <}
b - a \in \mathbb{R}^{+} \impliesbecause{A7 y c > 0}
(b - a)c \in \mathbb{R}^{+} \impliesbecause{A?}
bc - ac \in \mathbb{R}^{+} \impliesbecause{Def <}
ac < bc
$$
\end{prf}

\begin{prf}[$a \neq 0 \Rightarrow a^{2} > 0$]{}

Por la propiedad tricotómica, $a \neq 0 \Rightarrow a > 0 xor a < 0$. Analicemos los dos casos.

Analizemos a > 0:
$$
a > 0 \impliesbecause{Def >}
a - 0 \in \mathbb{R}^{+} \impliesbecause{Def -}
a + -0 \in \mathbb{R}^{+} \impliesbecause{T?}
a \in \mathbb{R}^{+} \impliesbecause{A?}
a.a \in \mathbb{R}^{+} \impliesbecause{Def > y Def x^{2}}
a^{2} > 0
$$
Analizamos a < 0:
$$
a < 0 \impliesbecause{Def <}
0 - a \in \mathbb{R}^{+} \impliesbecause{Def -}
0 + -a \in \mathbb{R}^{+} \impliesbecause{A?}
-a \in \mathbb{R}^{+} \impliesbecause{A?}
(-a).(-a) \in \mathbb{R}^{+} \impliesbecause{T?}
$$

$$
(-1.a) * (-1.a) \in \mathbb{R}^{+} \impliesbecause{A?}
(-1.-1).(aa) \in \mathbb{R}^{+} \impliesbecause{A?}
1.(aa) \in \mathbb{R}^{+} \impliesbecause{A?}
aa \in \mathbb{R}^{+} \impliesbecause{Def > y Def x^{2}}
a^{2} > 0 
$$
\end{prf}

\begin{prf}[$1 \in \mathbb{R}^{+}$]{}

Existen neutros, 0 y 1, $0 \neq 1$

Ax 8, $1 \in \mathbb{R^{+}} xor -1 \in \mathbb{R^{+}}$

Supongo $-1 \in \mathbb{R^{+}}$, entonces $1 \notin \mathbb{R^{+}}$

Por Ax 7 $-1 * -1 \in \mathbb{R^{+}}$

$-1 * -1 = 1 \in \mathbb{R^{+}}$, pero esto es una contradicción con lo supuesto
\end{prf}

\end{document}