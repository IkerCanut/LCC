\documentclass[10pt]{article}
\usepackage{hyperref}
\hypersetup{
    colorlinks=true,
    linkcolor=black,
    filecolor=magenta,      
    urlcolor=cyan,
}

\usepackage{import}
\usepackage{siunitx}
\usepackage{esvect}
\usepackage{fourier}
\usepackage{amssymb}
\usepackage{amsmath}
\usepackage{multicol}
\usepackage{geometry}
\usepackage[framemethod=TikZ]{mdframed}
\geometry{
 a4paper,
 total={160mm,237mm},
 left=30mm,
 top=30mm,
}

\usepackage{tikz}
\usetikzlibrary{automata, positioning, calc, through, angles, quotes, intersections}
\usepackage{multirow}

\subimport{.}{environment}

\author{Luciano N. Barletta \& Iker M. Canut}
\begin{document}
\title{Introducción a la Matemática}
\maketitle
\date
\newpage

\tableofcontents
\newpage

\section{Unidad 1: Numeros Reales}
Los números reales son elementos de un conjunto denominado $R$ entre los que existen dos operaciones que, por definición, satisfacen ciertas propiedades específicas llamadas axiomas.\\ Las operaciones son la suma y el producto. Si $a,b \in R$
\begin{itemize}
\item Y la operación suma les asigna el elemento $c \in R$, escribimos: $a+b=c$
\item Y la operación producto les asigna el elemento $d \in R$, escribimos $a.b=d$
\end{itemize}


\subsection{Axiomas de Cuerpo}
\begin{axiom}[Axioma de Cuerpo 1: Conmutativa]{}
$$a+b=b+a \land a.b=b.a$$
\end{axiom}

\begin{axiom}[Axioma de Cuerpo 2: Asociativa]{}
$$(a+b)+c=a+(b+c) \land (a.b).c=a.(b.c)$$
\end{axiom}

\begin{axiom}[Axioma de Cuerpo 3: Distributiva de la Multiplicacion respecto a la Suma]{}
$$a.(b+c)=a.b+a.c$$
\end{axiom}

\begin{axiom}[Axioma de Cuerpo 4: Existencia de Elementos Neutros]{}
Existen dos numeros reales, notados 0 y 1 $/\ \forall a \in R$
$$0+a = a+0 = a \land 1.a = a.1 = a$$
\end{axiom}

\begin{axiom}[Axioma de Cuerpo 5: Existencia de Elementos Opuestos]{}
$$\forall a \in R, \exists\ b \in R\ / a+b = b+a = 0$$
\end{axiom}

\begin{axiom}[Axioma de Cuerpo 6: Existencia de Elementos Reciprocos]{}
$$\forall a \in R-\{0\}, \exists\ b \in R\ / a.b = b.a = 1$$
\end{axiom}

\begin{theo}[Propiedad Cancelativa de la Suma]{}
$$a,b,c \in R, si\ a+b = a+c,\ entonces\ b=c $$
\end{theo}

\begin{data}{}
Junto con los axiomas, se presupone la validez de las siguientes propiedades de la igualdad:
\begin{itemize}
\item Propiedad de Reflexibidad: $\forall a, a=a$
\item Propiedad de Simetria: $si\ a=b \Rightarrow b=a$
\item Propiedad de Transitividad: $si\ a=b \land b=c \Rightarrow a=c$
\end{itemize}
\end{data}

\newpage

\begin{cor}[Unicidad del Elemento Neutro de la suma]{}
Si $0'$ es un numero que verifica que $a+0' = 0'+a = a$, $\forall a \in R$, entonces $0' = 0$
\end{cor}

\begin{cor}[Unicidad del Elemento Opuesto]{}
$\forall a \in R, \exists$ un unico numero $b$ / $a+b=b+a =0$
\end{cor}

\begin{data}{}
Para cualquier numero $a$, denotamos con $-a$ al unico elemento opuesto de $a$.

Llamamos \textit{diferencia} entre dos numeros reales $a$ y $b$, y lo denotamos como $a-b$, al numero dado por la suma de $a$ y el opuesto de $b$.
$$a-b = a+ (-b)$$
\end{data}

\begin{theo}[]{}
\begin{multicols}{2}
$$-(-a)=a$$
$$-0 = 0$$
$$0.a = 0$$
$$a(-b) = -(ab) = (-a)b$$
$$(-a)(-b)=ab$$
$$a(b-c)=ab-ac$$
$$-(a-b)=b-a$$
$$a=b \iff -a=-b$$
\end{multicols}
\end{theo}

\begin{property}[Propiedad]{}
Sean $a,b \in R$, $a-b=0 \iff a=b$
\end{property}

\begin{theo}[Propiedad Cancelativa del Producto]{}
$a,b,c \in R, a \not = 0$, si $ab=ac \Rightarrow b=c$
\end{theo}

\begin{cor}[Unicidad del Elemento Neutro del Producto]{}
Si $1'$ es un numero que verifica que $a.1' = 1'.a = a, \forall a \in R$, entonces $1'=1$
\end{cor}

\begin{cor}[Unicidad del Reciproco]{}
$\forall a \in R-\{0\}, \exists!\ b\ /\ ab = ba = 1$
\end{cor}

\begin{data}
Dado $a \in R-\{0\}$, el reciproco se nota como $a^{-1}$\\

Si $a,b \in R, b \not = 0$, llamamos cociente entre $a$ y $b$, y lo notamos $\dfrac{a}{b}$, al numero dado por el producto de $a$ y el reciproco de $b$. 
\end{data}

\newpage

\begin{theo}[]{}
\begin{multicols}{2}
\begin{enumerate}
\item $0$ no tiene reciproco
\item $1^{-1} = 1$
\item $\dfrac{a}{1}=a$ y si $a \not = 0 \Rightarrow \dfrac{1}{a}=a^{-1}$
\item Si $ab=0 \Rightarrow a=0 \lor b=0$
\item Si $b \not = 0 \land d \not = 0 \Rightarrow$
\begin{enumerate}
\item [i] $(bd)^{-1} = b^{-1}.d^{-1}$
\item [ii] $\dfrac{a}{b} + \dfrac{c}{d} = \dfrac{ad+bc}{bd}$
\item [iii] $\dfrac{a}{b} . \dfrac{c}{d} = \dfrac{ac}{bd}$
\end{enumerate}
\item Si $a \not = 0 \land b \not = 0 \Rightarrow \left(\dfrac{a}{b}\right)^{-1} = \dfrac{a^{-1}}{b^{-1}} = \dfrac{b}{a}$
\item $-a = -1. a$
\end{enumerate}
\end{multicols}
\end{theo}

\subsection{Axiomas de Orden}
Suponemos la existencia de un subconjunto de $R$, al que llamaremos conjunto de numeros positivos, y lo notaremos $R^+$, tal que satisface los siguientes axiomas.

\begin{axiom}[Axioma de Orden 7]{}
Si $a \in R^+ \land b \in R^+ \Rightarrow a+b \in R^+ \land a.b \in R^+$
\end{axiom}

\begin{axiom}[Axioma de Orden 8]{}
$\forall a \in R-\{0\}, a \in R^+ \lor (-a) \in R$
\end{axiom}

\begin{axiom}[Axioma de Orden 9]{}
$\emptyset \in R^+$
\end{axiom}

\begin{data}{}
\begin{itemize}
\begin{multicols}{2}
\item $a < b = b - a \in R^+$
\item $a > b = a - b \in R^+$
\item $a \leq b = b - a \in R^+ \lor b=a$
\item $a \geq b = a - b \in R^+ \lor b=a$
\item $a > 0 \iff a \in R^+$
\item $a > 0 \Rightarrow (-a) < 0$
\item $a < 0 \Rightarrow (-a) > 0$
\end{multicols}
Si $a<0$ se dice que a es negativo.
\end{itemize}
\end{data}

\begin{theo}[Propiedad de Tricotomia]{}
Dados dos numeros reales cualesquiera a y b, se verifica exactamente una de las siguientes afirmaciones:
\begin{multicols}{3}
\begin{enumerate}
\item [i] $a < b$
\item [ii] $ a = b$
\item [iii] $a > b$
\end{enumerate}
\end{multicols}
\end{theo}

\newpage

\begin{theo}[Propiedad Transitiva de a relacion menor]{}
Dados $a,b,c \in R$, si $a<b \land b<c \Rightarrow a<c$
\begin{multicols}{2}
\begin{enumerate}
\item Si $a<b \Rightarrow a+c < b+c$
\item Si $a<b \land c<d \Rightarrow a+c < b+d$
\item \begin{enumerate}
\item $a<b \land c>0 \Rightarrow ac < bc$
\item $a<b \land c<0 \Rightarrow ac > bc$
\end{enumerate}
\item $a \not = 0 \Rightarrow a^2 > 0$
\item $1 > 0$. Es decir, $1 \in R^+$
\item $a < b \Rightarrow -b < -a$
\item $ab > 0 \iff$ $a$ y $b$ son positivos o los dos son negativos
\item $ab < 0 \Rightarrow$ o $a$ es positivo y $b$ es negativo o viceversa.
\item $a>0 \iff \dfrac{1}{a} > 0$
\item $0<a<b \Rightarrow a<\dfrac{1}{b}<\dfrac{1}{a}$
\end{enumerate}
\end{multicols} 
\end{theo}

\section{Numeros Naturales, enteros y racionales e irracionales}
Sabemos de los axiomas de orden que $0<1$ y que para cualesquiera $a,b,c \in R$ se cumple que $a<b \Rightarrow a+c<b+c$\\
Luego, $0<1 \Rightarrow 0+1<1+1$ y se respeta el orden. Aplicando la propiedad transitiva, tenemos que $0<1<2$.













\newpage
\section{Demostraciones}



\begin{prf}[Demostracion de la Propiedad Cancelativa de la Suma]{}
Sean $a,b,c \in R$, $a+b=a+c \Rightarrow b=c$
Sea $d=a+b$, y por ende, $d=a+c$, por la existencia de elementos opuestos, existe $y$ que es opuesto  a $a$, entonces:

\begin{itemize}
\item $y+d$ \dotfill Hipotesis\\
$y+(a+b)$ \dotfill Asociativa\\
$(y+a)+b$ \dotfill Hipotesis\\
$0+b$ \dotfill Elemento neutro de la suma\\
$b$\\
\item $y+d$ \dotfill Hipotesis\\
$y+(a+c)$ \dotfill Asociativa\\
$(y+a)+c$ \dotfill Hipotesis\\
$0+c$ \dotfill Elemento neutro de la suma\\
$c$\\
\end{itemize}
Ergo, $b=c$
\end{prf}

\begin{prf}[Demostracion de la Unicidad del Elemento Neutro de la suma]{}
Supongamos que 0' es un numero que tambien funciona como neutro de la suma, entonces
$$a+0=a \land a+0'=a$$
$$a+0=a+0'$$
Y por propiedad cancelativa de la suma
$$0=0'$$
\end{prf}

\begin{prf}[Demostracion de la Unicidad del Elemento Opuesto]{}
La existencia de un numero b esta dada por el axioma 5, hay que demostrar que es unico. Suponiendo que existe $b'$ / $a+b'=b'+a=0$, tenemos que
$$a+b=0 \land a+b'=0$$
$$a+b = a+b'$$
Y por propiedad cancelativa de la suma
$$b=b'$$
\end{prf}

\begin{prf}[Demostracion de que el opuesto al opuesto de $a$ es $a$]{}
Sea $b$ el opuesto de $a$, se puede concluir que $a+b=0 \land b=(-a) \land a=(-b)$\hfill$(1)\land(2)\land(3)$
$$-(-a)\equals{(2)}-b\equals{(3)}a$$
\end{prf}

\begin{prf}[Demostracion de que el opuesto de 0 es 0]{}
Por el axioma 5, todo numero real tiene su opuesto. Llamemos 0' al opuesto de 0, siendo $0+0'=0$ y\\
Del axioma 3 se concluye que $0+0=0$
$$si\ 0+0'=0 \land 0+0=0 \Rightarrow 0'=0$$
\end{prf}

\begin{prf}[Demostracion de que el producto de 0 con cualquier otro numero es 0]{}
$$a.0 \equals{A4} a.0+0 \equals{A5} a.0+(a+(-a)) \equals{A2} (a.0+a)+(-a) \equals{A4} (a.0+a.1)+(-a) \equals{A3}$$
$$a(0+1)+(-a) \equals{A4} a.1+(-a) \equals{A4} a+(-a) \equals{A5} 0$$
\end{prf}

\begin{prf}[$a(-b)=-(ab)=(-a)b$]{}
$$
a(-b) \equals{A4}
a(-b)+0 \equals{A5}
a(-b)+(ab+-(ab)) \equals{A2}
(a(-b)+ab)+-(ab) \equals{A3}$$$$
(a((-b)+b)+-(ab)) \equals{A5}
a.0+-(ab) \equals{T2.3}
0+-(ab) \equals{A4}
-(ab)
$$
\end{prf}

\begin{prf}[$(-a)(-b)=ab$]{}
$$
(-a)(-b) \equals{T2.4}
-((-a)(-(-b))) \equals{T2.1}
-((-a)b) \equals{T2.4}
-(-(ab)) \equals{T2.1}
ab
$$
\end{prf}

\begin{prf}[$a(b-c)=ab-ac$]{}
Por la definicion de diferencia, se puede reescribir como:
$$
a(b+(-c)) \equals{A3}
ab+a(-c) \equals{T2.4}
ab+-(ac)
$$
Que por la definicion de diferencia, se puede reescribir como: $ab-ac$
\end{prf}
\end{document}