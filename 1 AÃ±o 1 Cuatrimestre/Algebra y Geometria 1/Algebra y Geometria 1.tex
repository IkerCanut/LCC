\documentclass[10pt]{article}
\usepackage{hyperref}
\hypersetup{
    colorlinks=true,
    linkcolor=black,
    filecolor=magenta,      
    urlcolor=cyan,
}

\usepackage{import}
\usepackage{siunitx}
\usepackage{esvect}
\usepackage{fourier}
\usepackage{amssymb}
\usepackage{amsmath}
\usepackage{multicol}
\usepackage{geometry}
\usepackage[framemethod=TikZ]{mdframed}
\geometry{
 a4paper,
 total={160mm,237mm},
 left=30mm,
 top=30mm,
}

\usepackage{tikz}
\usetikzlibrary{automata, positioning, calc, through, angles, quotes, intersections}
\usepackage{multirow}

\subimport{.}{environment}

\author{Luciano N. Barletta \& Iker M. Canut}
\begin{document}
\title{Práctica de Álgebra y Geometría 1}
\maketitle
\date
\newpage

\tableofcontents
\newpage

\section{Unidad 1: Números Complejos}

\subsection{Preámbulo}

Definimos el conjunto de los números complejos de la siguiente manera:
$$\C = \{(a,b) \mid a,b \in \R \}$$
O sea que $\C = \R \times \R$.\\
Dado un $z = (a,b) \in \C$, llamamos parte real de $z$ al número real $a$ y la notamos $\text{Re}(z) = a$. Análogamente llamamos parte imaginaria a $b$ y la notamos $\text{Im}(z) = b$.\\
Definimos para todo $z=(a,b) \in \C$ y $w=(c,d) \in \C$:
$$z=w \Leftrightarrow a=c \land b=d$$
$$z+w = (a+c,b+d)$$
$$zw = (ac-bd,bc+ad)$$
\subsection{Demostraciones}

\begin{prf}[Conmutatividad de la suma]{}
	Sean $z = (a,b), w = (c,d) \in \C$\\

	\lreqn{z + w =}{<Definicion de suma de complejos>}
	\lreqn{(a + c, b + d) =}{<Propiedad conmutativa de la suma de reales>}
	\lreqn{(c + a, d + b) =}{<Definicion de suma de complejos>}
	\lreqn{w + z}{}
\end{prf}

\begin{prf}[Conmutatividad del producto]{}
	Sean $z = (a,b), w = (c,d) \in \C$:\\

	\lreqn{zw =}{<Definicion de producto de complejos>}
	\lreqn{(ac - bd, ad + cb) =}{<Propiedad conmutativa de suma y producto de reales>}
	\lreqn{(ac - db, cb + da) =}{<Definicion de producto de complejos>}
	\lreqn{wz}{}
\end{prf}

\begin{prf}[Asociatividad de la suma]{}
	Sean $z = (a,b), u = (c,d), w = (e,f) \in \C$:\\

	\lreqn{z + (u + w) =}{<Definicion de suma de complejos>}
	\lreqn{z + (c + e, d + f) =}{<Definicion de suma de complejos>}
	\lreqn{(a + (c + e), b + (d + f)) =}{<Propiedad asociativa de la suma de reales>}
	\lreqn{((a + c) + e, (b + d) + f) =}{<Definicion de suma de complejos>}
	\lreqn{(a + c, b + d) + w =}{<Definicion de suma de complejos>}
	\lreqn{(z + u) + w}{}
\end{prf}

\begin{prf}[Propiedad distributiva de la multiplicación respecto de la suma]{}
	Por definción de igualdad de complejos, dados $z, w \in \C$
	$$z = w \Leftrightarrow \text{Re}(z) = \text{Re}(w) \land \text{Im}(z) = \text{Im}(w)$$
	llamemos $z = (a,b), u = (c,d), w = (e,f)$\\
	Demostramos $\text{Re}(z(uw)) = \text{Re}((zu)w)$\\
	\lreqn{\text{Re}(z(uw)) =}{<Def mult complejos>}
	\lreqn{(a\text{Re}(uw)) - (b\text{Im}(uw)) =}{<Def mult complejos>}
	\lreqn{(a(ce - df)) - (b(cf + de)) =}{<Distributiva>}
	\lreqn{(ace - adf) - (bcf + bde) =}{<Propiedad -(a+b) = -a - b>}
	\lreqn{ace - adf - bcf - bde =}{<Conmutativa>}
	\lreqn{ace - bde - adf - bcf =}{<Distributiva>}
	\lreqn{(ac - bd)e - (ad + bc)f =}{<Def mult complejos>}
	\lreqn{\text{Re}(zu)e - \text{Im}(zu)f =}{<Def mult complejos>}
	\lreqn{\text{Re}((zu)w)}{}
	Demostramos $\text{Im}(z(uw)) = \text{Im}((zu)w)$\\

	\lreqn{\text{Im}(z(uw)) =}{<Def mult complejos>}
	\lreqn{(b\text{Re}(uw)) + (a\text{Im}(uw)) =}{<Def mult comlpejos>}
	\lreqn{(b(ce - df)) + (a(cf + de)) =}{<Distributiva>}
	\lreqn{bce - bdf + acf + ade =}{<Conmutativa>}
	\lreqn{bce + ade + acf - bdf =}{<Distributiva>}
	\lreqn{(bc + ad)e + (ac - bd)f =}{<Def mult complejos>}
	\lreqn{\text{Im}(zu)e + \text{Re}(zu)f =}{<Def mult complejos>}
	\lreqn{\text{Im}((zu)w)}{}
	\text{Re}escribiendo:
	$$\text{Re}(z(uw)) = \text{Re}((zu)w) \land \text{Im}(z(uw)) = \text{Im}((zu)w)$$
	que por definición de igualdad de complejos implica
	$$z(uw) = (zu)w$$

\end{prf}

\begin{prf}[$\exists (0,0) \in \C / (0,0) + z = z$]{}
	Sea $z = (a,b) \in \C$:\\

	\lreqn{(0,0) + z =}{<Definicion de la suma de complejos>}
	\lreqn{(0 + a, 0 + b) =}{<Existencia del elemento neutro de suma de reales>}
	\lreqn{(a, b) =}{<Definicion de numero complejo>}
	\lreqn{z}{}
\end{prf}

\begin{prf}[$\exists (1,0) \in \C / (1,0)z = z$]{}
	Sea $z = (a,b) \in \C$:\\

	\lreqn{(1,0)z =}{<Definicion del producto de complejos>}
	\lreqn{(1a - 0b, 0a + 1b) =}{<Existencia del elemento neutro del producto de reales>}
	\lreqn{(a - 0b, 0a + b) =}{<a0 = 0>}
	\lreqn{(a - 0, 0 + b) =}{<Existencia del elemento neutro de la suma de reales>}
	\lreqn{(a, b) =}{<Definicion de numero complejo>}
	\lreqn{z}{}
\end{prf}

\begin{prf}[$\forall z = (a,b) \exists w = (-a,-b) / z + w = (0,0)$]{}
	Sea $z = (a,b), w = (-a,-b) \in \C$\\

	\lreqn{(0,0) =}{<Existencia del opuesto de la suma de reales>}
	\lreqn{(a + -a, b + -b) =}{<Definición de suma de complejos>}
	\lreqn{(a,b) + (-a,-b) =}{<sustituimos z = (a,b), w = (-a,-b)>}
	\lreqn{z + w}{}
	Por lo tanto:\\
	$$\forall z = (a,b) \in \Z \exists w = (-a,-b) / z + w = (0,0)$$
\end{prf}

%%%%%%%%%%%%%%%%%%%%%%%%%%%%%%%%%%%%%%%%%%%%%%%%%%%%%%%%%
% fucking asqueroso por dios no sé cómo se arregla esto %
%%%%%%%%%%%%%%%%%%%%%%%%%%%%%%%%%%%%%%%%%%%%%%%%%%%%%%%%%

\begin{prf}[$\forall z \neq (0,0) \exists w / zw = (1,0)$]{}
	Llamemos z = (a,b) y llamemos w = (c,d).\\
	Para que zw = (1,0), por definición de igualdad de complejos, tiene que ocurrir: 
	\begin{equation}
		\text{Re}(zw) = 1 \land \text{Im}(zw) = 0
	\end{equation}
	Que exista algún $w$ para todo $z$ implica entonces que podamos escribir $(c,d)$ en términos de $(a,b)$, basándonos en el siguiente sistema de ecuaciones:
	\begin{align}
		ac - bd &= 1\\
		bc + ad &= 0\\
		ac - bd &= 1\\
		bc &= -ad\\
		ac - bd &= 1\\
		c &= -\frac{ad}{b}
	\end{align}
	Continuamos con la ecuación de arriba
	\begin{align}
		a\left(\frac{-ad}{b}\right) - bd =& 1\\
		\left(\frac{-a^2}{b}\right)d - bd =& 1\\
		\left(\frac{-a^2}{b} - b\right)d =& 1\\
		\left(\frac{-a^2}{b} - \frac{b^2}{b}\right)d =& 1\\
		\frac{-a^2 - b^2}{b}d =& 1\\
		-\frac{a^2 + b^2}{b}d =& 1\\
		d = -\frac{b}{a^2 + b^2}&
	\end{align}
	Reemplazando en la de abajo
	\begin{align}
		c &= \left(-\frac{a}{b}\right)\left(-\frac{b}{a^2 + b^2}\right)\\
		c &= \frac{a}{b}\frac{b}{a^2 + b^2}\\
		c &= \frac{a}{a^2 + b^2}
	\end{align}
	La demostración queda entonces:
	$\forall z = (a,b) \neq (0,0)$, supongo un $w = \left(\dfrac{a}{a^2 + b^2}, -\dfrac{b}{a^2 + b^2}\right)$ y muestro que $zw = (1,0)$
	\begin{align}
		zw = \left(a\frac{a}{a^2 + b^2} - b\frac{-b}{a^2 + b^2}, b\frac{a}{a^2 + b^2} + a\frac{-b}{a^2 + b^2}\right)\\
		\left(\frac{a^2}{a^2 + b^2} + \frac{b^2}{a^2 + b^2}, \frac{ba}{a^2 + b^2} - \frac{ab}{a^2 + b^2}\right)\\
		\left(\frac{a^2 + b^2}{a^2 + b^2}, \frac{ba - ab}{a^2 + b^2}\right) = (1,0)
	\end{align}
	$$\therefore \forall z = (a,b) \neq (0,0), \exists w = (\frac{a}{a^2 + b^2}, -\frac{b}{a^2 + b^2}) / zw = (1,0)$$
\end{prf}

% la unicidad queda para el lector %

Llamamos $\C_0$ al conjunto $\left\{ z \mid z = (a,0) \in \C, \forall a \in \R \right\}$.\\
Definimos $z - w = z + (-w)$.\\
Definimos $\dfrac{z}{w} con w \neq (0,0), zw^{-1}$

\begin{prf}[Suma cerrada en $\C_0$]{}
	Sean $z = (a,0), w = (b,0) \in \C_0$\\

	\lreqn{z+w=}{<Definición de suma de complejos>}
	\lreqn{(a+b,0+0)=}{}
	\lreqn{(a+b,0)=}{}

	si llamamos $a+b=c$, entonces $(c,0) \in \C_0$ por definición
\end{prf}

\begin{prf}[Producto cerrado en $\C_0$]{}
	Sean $z = (a,0), w = (b,0) \in \C_0$\\

	\lreqn{zw=}{<Definición de producto de complejos>}
	\lreqn{(ab - 0.0, 0b + 0a)=}{}
	\lreqn{(ab,0+0)=}{}
	\lreqn{(ab,0)=}{}

	si llamamos $ab=c$, entonces $(c,0) \in \C_0$ por definición
\end{prf}

\begin{prf}[Producto cerrado en $\C_0$]{}
	Sean $z = (a,0), w = (b,0) \in \C_0$\\

	\lreqn{zw=}{<Definición de producto de complejos>}
	\lreqn{(ab - 0.0, 0b + 0a)=}{}
	\lreqn{(ab,0+0)=}{}
	\lreqn{(ab,0)=}{}

	si llamamos $ab=c$, entonces $(c,0) \in \C_0$ por definición
\end{prf}

\begin{prf}[Opuesto y recíproco cerrado en $\C_0$]{}
	Sea $z = (a,0), \in \C_0$
	$$-z = (-a,0) \in \C_0$$
	Sea $z = (a,0) \neq (0,0), \in \C_0$
	$$z^{-1} = (a^{-1},0) \in \C_0$$
\end{prf}

\begin{prf}[Cociente cerrado en $\C_0$]{}
	Sean $z = (a,0), w = (b,0) \neq (0,0) \in \C_0$\\

	\lreqn{\frac{z}{w}=}{<Definición de cociente de complejos>}
	\lreqn{zw^{-1}}{}

	Como el producto es cerrado en $\C_0$, $\dfrac{z}{w} \in \C_0$
\end{prf}
Notamos la correspondecia:
$$x \in \R \leftrightarrow (x,0) \in \C_0$$
Ahora definimos:
$$i = (0,1)$$
\begin{prf}[$\forall (a,b) \in \C$, $(a,b)$ puede escribirse $a + bi$]{}
	\lreqn{(a,b)}{<Definición de suma de complejos>}
	\lreqn{(a,0) + (0,b)}{}
	\lreqn{(a,0) + (0 - 0, 1b + 0)}{}
	\lreqn{(a,0) + (0.b - 1.0, 1b + 0.0)}{<Definición de producto de complejos>}
	\lreqn{(a,0) + (b,0)(0,1)}{<Definición de $i$>}
	\lreqn{(a,0) + (b,0)i}{<Notación de elementos de $\C_0$>}
	\lreqn{a + bi}{}
\end{prf}
Entonces usaremos esta notación para referirnos a este tipo de números complejos, $\forall a,b \in \R$:
\begin{align}
	(a,0) &= a\\
	(0,b) &= bi
	(a,b) &= a+bi\\
\end{align}
La última se llama notación binómica de los números complejos.
\begin{prf}[$i^2 = -1$]{}
	\lreqn{i^2 =}{<Definición de cuadrado>}
	\lreqn{i.i =}{<Definición de $i$>}
	\lreqn{(0,1)(0,1) =}{<Definición de producto de complejos>}
	\lreqn{(0.0 - 1.1,1.0 + 0.1)}{}
	De esto resulta el número complejo $(-1,0)$, que representa al número real $-1$.
	Notamos que hacer raíz cuadrada de ambos lados nos deja $\pm i = \sqrt{-1}$.
\end{prf}

Fun Fact
\begin{prf}[$a \in \R, a < 0, z^2 = a \Leftrightarrow z = \pm\sqrt{|a|}i$]{}
	Sabemos que $z^2 = a$, pero por definición de valor absoluto
	$$z^2 = -|a|, z^2 = -1|a|$$
	Al hacer raíz cuadrada de ambos lados
	$$\pm z = \sqrt{-1|a|}$$
	Que reescribiendo sería
	$$z = \mp \sqrt{|a|}\sqrt{-1}$$
	Finalmente por definición de $i$
	$$z = \pm \sqrt{|a|}i$$

\end{prf}
Dado $z = a+bi$, llamamos conjugado de $z$ al número complejo $a-bi$ y lo notamos $\overline{z}$.

\begin{prf}[Distributiva del conjugado respecto de la suma]{}
	Sean $z = a+bi$, $w = c+di \in \C$\\

	\lreqn{\overline{z+w}}{}
	\lreqn{\overline{(a+bi)+(c+di)}}{}
	\lreqn{\overline{(a+c)+(b+d)i}}{<Definición de conjugado>}
	\lreqn{(a+c)-(b+d)i}{<Distribuyendo $-i$>}
	\lreqn{a+c-bi-di}{<Reescribiendo>}
	\lreqn{(a-bi)+(c-di)}{<Definición de conjugado>}
	\lreqn{\overline{z}+\overline{w}}{}
\end{prf}

\begin{prf}[Distributiva del conjugado respecto del producto]{}
	Sean $z = a+bi$, $w = c+di \in \C$\\

	\lreqn{\overline{zw}}{<Hipótesis>}
	\lreqn{\overline{(a+bi)(c+di)}}{<Distribuyendo>}
	\lreqn{\overline{(ac-bd)+(bc+ad)i}}{<Definición de conjugado>}
	\lreqn{(ac-bd)-(bc+ad)i}{<Distribuyendo $-1$>}
	\lreqn{(ac-bd)+((-b)c+a(-d))i}{<$bd = (-d)(-b)$>}
	\lreqn{(ac-(-b)(-d))+((-b)c+a(-d))i}{<Distribuyendo $i$ y reescribiendo>}
	\lreqn{ac+(-b)c.i+a(-d).i-(-b)(-d)}{$-1 = i^2$}
	\lreqn{ac+(-b)c.i+a(-d).i-(-b)d.i^2}{<Factor común $c$ y $-di$>}
	\lreqn{(a-bi)c-(a-bi)di}{<Factor común $a-bi$>}
	\lreqn{(a-bi)(c-di)}{<Definición de conjugado>}
	\lreqn{\overline{z}\overline{w}}{}
\end{prf}

\begin{prf}[$z = \overline{z} \Leftrightarrow z \in \R$]{}
	Sea $z = a+bi = \overline{z} = a-bi, z \in \C$\\

	\lreqn{z = \overline{z}}{<Hipótesis>}
	\lreqn{a+bi = a-bi}{<Propiedad cancelativa de la suma>}
	\lreqn{bi = -bi}{<Sumando $bi$ en ambos lados>}
	\lreqn{2bi = 0}{<Multiplicando $(2i)^{-1}$ en ambos lados>}
	\lreqn{b = 0}{}
	$$z = a+0i = a \in R$$
	Sea $a \in \R$\\

	\lreqn{a = a}{<Elemento neutro de la suma>}
	\lreqn{a+0 = a+0}{$0x = 0$}
	\lreqn{a+0i = a+0(-i)}{<Definición de conjugado>}
	\lreqn{a+0i = \overline{a+0i}}{<Llamemos $z = a+0i$}
	\lreqn{z = \overline{z}}{}
	$$z = a \in \R \Rightarrow z = \overline{z}$$

	$$\therefore z = \overline{z} \Leftrightarrow z \in \R$$
\end{prf}

\begin{prf}[$z\overline{z} = \text{Re}(z)^2+\text{Im}(z)^2$]{}
	Sea $z = a+bi, z \in \C$\\

	\lreqn{z\overline{z}}{<Hipótesis y definición de conjugado>}
	\lreqn{(a+bi)(a-bi)}{<Distribuyendo>}
	\lreqn{a^2+ab.i-ab.i-(bi)^2}{<Resolviendo>}
	\lreqn{a^2-(bi)^2}{<Distribuyendo>}
	\lreqn{a^2-b^2i^2}{<$i^2 = -1$>}
	\lreqn{a^2-b^2.(-1)}{}
	\lreqn{a^2+b^2}{<$a = \text{Re}(z), b = \text{Im}(z)$>}
	\lreqn{\text{Re}(z)^2+\text{Im}(z)^2}{}
\end{prf}
Fun Fact 2. $z = a+bi \in \C z \neq 0$
$$\frac{1}{z} = \frac{1}{z}\frac{\overline{z}}{\overline{z}} = \frac{\overline{z}}{z\overline{z}} = \frac{a-bi}{a^2+b^2} = \frac{a}{a^2+b^2}-\frac{b}{a^2+b^2}i$$
Fun Fact 2.5. $w = a+bi \in \C w \neq 0$
$$\frac{z}{w} = \frac{z}{w}\frac{\overline{w}}{\overline{w}} = \frac{z\overline{w}}{w\overline{w}} = \frac{z\overline{w}}{a^2+b^2}$$

\newpage
\begin{prf}[Formula de Moivre]{}
$$z^n = |z|^n (cos\ n\alpha + i\ sin\ n\alpha), n \in \mathbb{Z}$$
Prueba por induccion:\\
\begin{itemize}

\item Para el caso n = 1 resulta trivial,
$$ z^1 = |z|^1 (cos\ 1\alpha + i\ sin\ 1\alpha) $$
$$ z = |z| (cos\ \alpha + i\ sin\ \alpha)$$

\item Para el caso n > 0, hay que demostrar que $\underbrace{(cos\ \alpha + i\ sin\ \alpha)^n = cos\ n\alpha + i\ sin\ n\alpha}_{HI}$\\
Para $n+1$:
$$ (cos\ \alpha + i\ sin\ \alpha)^{n+1} $$
$$ \underbrace{(cos\ \alpha + i\ sin\ \alpha)^n}_{HI} . (cos\ \alpha + i\ sin\ \alpha) $$
$$ (cos\ n\alpha + i\ sin\ n\alpha) . (cos\ \alpha + i\ sin\ \alpha) $$
$$ (cos(n\alpha).cos(\alpha) - sin(n\alpha).sin(\alpha) + i\ (\cos(n\alpha).sin(\alpha) + sin(n\alpha).cos(\alpha)) $$
$$ cos(n\alpha + \alpha) + i\ sin(n\alpha + \alpha) $$
$$ cos((n + 1)\alpha) + i\ sin((n + 1)\alpha) $$

\item Para el caso n = 0, tambien resulta trivial,
$$ cos(0\alpha) + i\ sin(0\alpha) = 1 + 0i = 1 $$

\item Para el caso n < 0:
$$ (cos\ \alpha + i\ sin\ \alpha)^{-n} $$
$$ \left((cos\ \alpha + i\ sin\ \alpha)^n\right)^{-1} $$
\hfill Aplicando el item 2
$$ (cos\ n\alpha + i\ sin\ n\alpha)^{-1} $$
\hfill Aplicando la identidad $z^{-1} = \dfrac{\overline{z}}{|z|^2}$
$$ cos(-n\alpha) + i\ sin(-n\alpha) $$
\end{itemize}
Con lo que queda demostrado para todos los numeros enteros $n$ que:
$$z^n = |z|^n (cos\ n\alpha + i\ sin\ n\alpha)$$
\end{prf}

\end{document}
