\documentclass[11pt,a4paper]{article}
\usepackage[utf8]{inputenc}
\usepackage[spanish]{babel}
\usepackage{amsmath}
\usepackage{amsfonts}
\usepackage{amssymb}
\usepackage{graphicx}
\usepackage[left=2cm,right=2cm,top=2cm,bottom=2cm]{geometry}
\usepackage{multicol}
\author{Iker M. Canut}
\title{Unidad 3: Conjuntos}
\begin{document}
\maketitle
\newpage
\section{Teor\'ia de Conjuntos}
\textbf{Conjunto}: Colecci\'on bien definida de elementos. Los conjuntos se escriben con letras may\'usculas, los elementos con min\'usculas.
\begin{itemize}
\item $a \in A$: El elemento $a$ \textbf{pertenece} al conjunto $A$.
\item $a \not \in A$: El elemento $a$ \textbf{no pertenece} al conjunto $A$.
\end{itemize}
Definimos un conjunto por \textbf{extensi\'on} si enumeramos todos los elementos que pertenecen. Definimos un conjunto por \textbf{comprensi\'on} si damos una caracteristica, una ley que define si un elemento pertenece o no al conjunto.\\

El universo en el cual estan todos los elementos, se lo denomina \textbf{universal}, $\mathbb{U}$.\\

\noindent \dotfill\\

\begin{itemize}
\item $C$ es un \textbf{subconjunto} de $D$ $\iff C \subseteq D \iff \forall x[x \in C \Rightarrow x \in D]$
\begin{multicols}{2}
\item $C \not \subseteq D \iff \exists x [x \in C \land x \not \in D]$
\item $C \subseteq D \Rightarrow |C| \leq |D|$\\
\end{multicols}

\item $C$ es un \textbf{subconjunto propio} de $D$ $\iff C \subset D \iff C \subseteq D \land C \not = D$
\begin{multicols}{2}
\item $C \not \subset D \iff C \not \subseteq D \lor C=D$
\item $C \subset D \Rightarrow |C| < |D|$\\
\end{multicols}

\item $C$ es \textbf{igual} a $D$ $\iff C = D \iff C \subseteq D \land D \subseteq C \iff \forall x[x \in C \iff x \in D]$
\item $C$ es \textbf{distinto} a $D$ $\iff C \not = D \iff C \not \subseteq D \lor D \not \subseteq C$
\end{itemize}
\noindent \dotfill\\

Sean $A, B, C \subseteq U$
\begin{multicols}{2}
\begin{itemize}
\item Si $A \subseteq B \land B \subseteq C \Rightarrow A \subseteq C$
\item Si $A \subseteq B \land B \subset C \Rightarrow A \subset C$
\item Si $A \subset B \land B \subseteq C \Rightarrow A \subset C$
\item Si $A \subset B \land B \subset C \Rightarrow A \subset C$
\end{itemize}
\end{multicols}
\noindent \dotfill\\

Se llama \textbf{conjunto vacio}, $\emptyset$ o $\{\}$ al conjunto que no tiene elementos. $|\emptyset| = 0$\\

Para cualquier $\mathbb{U}$, $A \subseteq U$ se tiene que $\emptyset \subseteq A$. Y si $a \not = \emptyset \Rightarrow \emptyset \subset A$\\

\noindent \dotfill\\

Dado un conjunto $A$, se llama \textbf{conjunto de partes} de $A$ al conjunto cuyos elementos son todos los subconjuntos de $A$. $P(A) = \{ F . F \subseteq A \}$\\

\noindent \dotfill\\
\newpage

\section{Operaciones de Conjuntos}
\begin{itemize}
\item \textbf{Uni\'on} de $A$ y $B$: Es el conjunto cuyos elemento pertenecen a $A$ o a $B$.\\
$A \cup B = \{x\in \mathbb{U} . x\in A \lor x\in B\}$

\noindent \dotfill
\item \textbf{Intersecci\'on} de $A$ y $B$: Es el conjunto cuyos elementos pertenecen a $A$ y a $B$.\\
$A \cap B = \{x\in \mathbb{U} . x\in A \land x\in B\}$\\
Dos conjuntos son \textbf{disjuntos} si la intersecci\'on es el conjunto vacio.\\
$A$ y $B$ son disjuntos $\iff$ $A \cup B = A \triangle B$

\noindent \dotfill
\item \textbf{Diferencia} de $A$ y $B$: Es el conjunto de elementos que pertenecen a $A$ y no a $B$.\\
$A - B = \{x\in \mathbb{U}.x\in A, x\not \in B\} = A \cap \overline{B}$
\begin{itemize}
\begin{multicols}{3}
\item $A - A = \emptyset$
\item $A - \emptyset = A$
\item $\emptyset - A = \emptyset$
\end{multicols}
\begin{multicols}{2}
\item $(A-B = B-A) \iff A=B$
\item $(A-B)-C = A-(B-C)$
\end{multicols}
\end{itemize}

\noindent \dotfill
\item \textbf{Complemento} de $B$ respecto de $A$: es la diferencia.\\
$\complement_A B = A-B = \{x.x\in A, x\not \in B\}$\\
Si tomamos $A$ = $\mathbb{U}$, notamos $\complement_U B = \complement B = \overline{B}$
\begin{itemize}
\begin{multicols}{3}
\item $\complement \mathbb{U} = \emptyset$
\item $\complement \emptyset = \mathbb{U}$
\item $\complement (\complement A) = A$
\item $\complement (A \cup B) = \complement A \cap \complement B$
\item $\complement (A \cap B) = \complement A \cup \complement B$
\item $A \subseteq B \Rightarrow A \cup \complement_B A = B$
\end{multicols}
\end{itemize}

\noindent \dotfill
\item \textbf{Diferencia Simetrica} de $A$ y $B$: son los elementos que pertenecen a $A$ o a $B$, pero no a ambos.
\begin{align*}
A \triangle B 
&= \{x\in \mathbb{U} . x\in A\ \underline{\lor}\ x\in B\}\\
&= (A \cup B) - (A \cap B) = (A \cap \overline{B}) \cup (\overline{A} \cap B)\\
&= (A \cup B) \cap \overline{(A \cap B)} = (A-B) \cup (B-A)
\end{align*}

\noindent \dotfill
\item \textbf{Producto Cartesiano} de $A$ y $B$: es el conjunto de pares ordenados $(a,b)$ tal que la primer componente pertenece a $A$ y la segunda pertenece a $B$.\\
$A\times B = \{ (a,b) . a\in A, b\in B \}$\\
Si $A = B$ se escribe $A \times A = A^2$\\

\noindent \dotfill
\end{itemize}
\newpage

\section{Generalizaciones}
Sean $E_1, E_2,...E_n \subseteq U$ se llama:
\begin{itemize}
\item \textbf{uni\'on} de $E_1, E_2,...E_n$ al conjunto $E_1 \cup E_2 \cup ... \cup E_n$ = $\displaystyle{\bigcup_{i \in 1}^{n} E_i} = \{ x \in \mathbb{U} . x\in E_i, \text{para algun } i=1..n \}$
\item \textbf{intersecci\'on} de $E_1, E_2,...E_n$ al conjunto $E_1 \cap E_2 \cap ... \cap E_n$ = $\displaystyle{\bigcap_{i \in 1}^{n} E_i} = \{ x \in \mathbb{U} . x\in E_i, \forall i=1..n \}$
\end{itemize}
\noindent \dotfill\\

Sea $I$ un conjunto no vacio, $U$ el conjunto universal,\\
$\forall i \in I$ sea $A_i \subseteq \mathbb{U}$. Cada $i$ es un indice, e $I$ es el conjunto de indices.\\
\begin{itemize}
\item $\displaystyle{\bigcup_{i \in I} A_i} = \{ x \in \mathbb{U} . x\in A_i, \text{ para algun } i\in I \}$
\item $\displaystyle{\bigcap_{i \in I} A_i} = \{ x \in \mathbb{U} . x\in A_i, \text{ para todo } i\in I \}$
\end{itemize}
Equivalentemente,
\begin{itemize}
\item $x \in \displaystyle{\bigcup_{i \in I} A_i} \iff \exists i \in I . (x \in A_i)$
\item $x \in \displaystyle{\bigcap_{i \in I} A_i} \iff \forall i \in I . (x \in A_i)$
\end{itemize}
\noindent \dotfill\\

\begin{itemize}
\begin{multicols}{2}
\item $\overline{\displaystyle{\bigcup_{i \in I} A_i}} = \displaystyle{\bigcap_{i \in I} \overline{A_i}}$
\item $\overline{\displaystyle{\bigcap_{i \in I} A_i}} = \displaystyle{\bigcup_{i \in I} \overline{A_i}}$
\end{multicols}
\end{itemize}
\noindent \dotfill

\newpage
\section{Leyes}
\begin{tabular}{p{0.03\textwidth}p{0.33\textwidth}p{0.33\textwidth}p{0.15\textwidth}}
1. & $\overline{\overline{A}} = A$ & & Doble negaci\'on \\
2. & $\overline{A \cup B} = \overline{A} \cap \overline{B}$ & $\overline{A \cap B} = \overline{A} \cup \overline{B}$ & De Morgan \\
3. & $A \cup B = B \cup A$ & $A \cap B = B \cap A$ & Conmutativa \\
4. & $(A\cup B) \cup C = A \cup (B \cup C)$ & $(A\cap B) \cap C = A \cap (B \cap C)$ & Asociativa \\
5. & $A \cup (B \cap C) = (A \cup B) \cap (A \cup C)$ & $A \cap (B \cup C) = (A \cap B) \cup (A \cap C)$ & Distributiva \\
6. & $A \cup A = A$ & $A \cap A = A$ & Idempotente \\
7. & $A \cup \emptyset = A$ & $A \cap \mathbb{U} = A$ & Neutro \\
8. & $A \cup \overline{A} = \mathbb{U}$ & $A \cap \overline{A} = \emptyset$ & Inverso \\
9. & $A \cup \mathbb{U} = \mathbb{U}$ & $A \cap \emptyset = \emptyset$ & Dominaci\'on \\
10. & $A \cup (A \cap B) = A$ & $A \cap (A \cup B) = A$ & Absorci\'on \\
\end{tabular}

\vspace{1cm}
\noindent \dotfill\\

\section{Cardinalidad}
La cardinalidad de un conjunto finito es la cantidad de elementos que contiene.
\begin{itemize}
\item $|A\cup B| = |A| + |B| - |A \cap B|$
\item $|\overline{A} \cap \overline{B} \cap \overline{C}| = |\overline{A \cup B \cup C}| = |U| = |A \cup B \cup C|\\ = |U| - |A| - |B| - |C| + |A \cap B| + |A \cap C| + |B \cap C| - |A \cap B \cap C|$
\end{itemize}

\section{Dual}
Para conseguir el dual de un conjunto, se reemplazan:
\begin{itemize}
\item $\emptyset$ por $\mathbb{U}$ y $\mathbb{U}$ por $\emptyset$.
\item $\cup$ por $\cap$ y $\cap$ por $\cup$.
\end{itemize}

\end{document}