\documentclass[11pt,a4paper]{article}
\usepackage[utf8]{inputenc}
\usepackage[spanish]{babel}
\usepackage{amsmath}
\usepackage{amsfonts}
\usepackage{amssymb}
\usepackage{graphicx}
\usepackage[left=2cm,right=2cm,top=2cm,bottom=2cm]{geometry}
\usepackage{multicol}
\author{Iker M. Canut}
\title{Unidad 1: N\'umeros Complejos y Polinomios\\\'Algebra y Geometr\'ia Anal\'itica}
\begin{document}
\maketitle
\newpage
\section{N\'umeros Complejos}
\noindent El conjunto de los n\'umeros complejos es $\mathbb{C} = \{ z = a+bi:\ a,b \in \mathbb{R}\}$, donde $i$ es la \textbf{unidad imaginaria} que verifica $i^2 = -1$. Si $z\in\mathbb{C}$, $z=a+bi$ es la \textbf{forma bin\'omica} de $z$.\\

\noindent La \textbf{parte real} de $z$ es $a$, $Re\ z = a$, y la \textbf{parte imaginaria} de $z$ es $b$, $Im\ z = b$.\\
$z = w \iff Re\ z = Re\ w \land Im\ z = Im\ w$.\\

\noindent Sea $z=a+bi$ y $w=c+di$, luego $z+w = (a+c)+(b+d)i$ y tambi\'en $z \cdot w = (ac-bd)+(ad+bc)i$\\

\noindent La suma y el producto son asociativos y conmutativos, y vale la propiedad distributiva.\\

\noindent Sea $z=a+bi \in \mathbb{C}$, llamamos \textbf{conjugado} de $z$ al complejo $\overline{z}=a-bi$. Y llamamos \textbf{m\'odulo} de $z$ al real $|z| = \sqrt{a^2+b^2}$. Adem\'as, $|z|^2 = z\cdot\overline{z}$ y tambi\'en $z^{-1} = \dfrac{\overline{z}}{|z|^2}$. Luego, $\dfrac{z}{w} = \dfrac{ac+bd}{c^2+d^2} + \dfrac{bc-ad}{c^2+d^2}i$.

\subsection{Propiedades}
\begin{itemize}
\begin{multicols}{3}
\item $\overline{\overline{z}} = z$
\item $\overline{z+w} = \overline{z} + \overline{w}$
\item $\overline{z\cdot w} = \overline{z} \cdot \overline{w}$
\item Si $z \not = 0$, $\overline{z^{-1}} = (\overline{z})^{-1}$
\item $z + \overline{z} = 2\cdot Re\ z$
\item $z - \overline{z} = 2\cdot (Im\ z) \cdot i$
\item $z = 0 \iff |z| = 0$
\item $|z\cdot w| = |z|\cdot|w|$
\item $|z|=|\overline{z}|$
\item $|z|=|-z|$
\item Si $z \not = 0$, $|z^{-1}| = |z|^{-1}$
\item Si $z \not = 0$, $\left|\dfrac{z}{w}\right| = \dfrac{|z|}{|w|}$
\end{multicols}
\end{itemize}

\subsection{Otras Formas}
\noindent La \textbf{forma polar} de $z \in \mathbb{C}$ es $z = |z|_{arg\ z}$, donde $arg\ z$ es el \'unico real tal que:
\begin{multicols}{3}
\begin{itemize}
\item $0\leq arg\ z \leq 2\pi$
\item $\cos (arg\ z) = \dfrac{a}{|z|}$
\item $\sin (arg\ z) = \dfrac{b}{|z|}$
\end{itemize}
\end{multicols}

\noindent La \textbf{forma trigonom\'etrica} de $z \in \mathbb{C}$ es $z = |z| (\cos arg\ z + i \sin arg\ z)$.\\
\noindent Sea $z=\rho (\cos \alpha + i \sin \alpha)$ y $w = \tau (\cos \beta + i \sin \beta)$, $z=w \iff (\rho = \tau) \land \alpha = \beta + 2k\pi,\ k \in \mathbb{Z}$.\\

\noindent \textbf{Teorema de Moivre}: Sean $z,w \in \mathbb{C}, z\not=0, w\not=0,\ \ z=|z|(\cos \alpha + i \sin \alpha),\ \ w = |w| (\cos \beta + i \sin \beta)$
$$z \cdot w = |z||w|[\cos (\alpha + \beta) + i\sin(\alpha+\beta)]$$

\begin{itemize}
\begin{multicols}{2}
\item $z^{-1} = |z|^{-1}[\cos (-\alpha) + i \sin (-\alpha)]$
\item $\overline{z} = |z|[\cos (-\alpha) + i \sin (-\alpha)]$
\item $\dfrac{z}{w} = \dfrac{|z|}{|w|} [\cos (\alpha - \beta) + i \sin (\alpha - \beta)]$
\item $z^n = |z|^n [\cos (n\cdot\alpha) + i \sin (n\cdot\alpha)]$, con $n\in\mathbb{N}$
\end{multicols}
\end{itemize}

\noindent Si $w\in\mathbb{C}, w\not=0$, una \textbf{raiz n-\'esima} de $w$, con $n\in\mathbb{N}$, es un n\'umero $z$ tal que $z^n=w$:
$$z = |z|^{\dfrac{1}{n}} \left[\cos \dfrac{arg\ w + 2k\pi}{n} + i \sin \dfrac{arg\ w + 2k\pi}{n}\right], \ \ \ \ 0 \leq k \leq n-1, k \in \mathbb{N}$$

\noindent La \textbf{notaci\'on exponencial} de $z$ es $z=|z|e^{i\alpha}$. Se verifica que $\overline{e^{i\alpha}} = e^{\overline{i\alpha}} = e^{-i\alpha}$ y que $e^{i\alpha} \cdot e^{i\beta} = e^{i(\alpha+\beta)}$

$$\sqrt[n]{z} = \sqrt[n]{|z|} e^{\dfrac{i(\theta + 2k\pi)}{n}}$$

\end{document}