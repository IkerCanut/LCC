\documentclass[11pt,a4paper]{article}
\usepackage[utf8]{inputenc}
\usepackage[spanish]{babel}
\usepackage{amsmath}
\usepackage{amsfonts}
\usepackage{amssymb}
\usepackage{graphicx}
\usepackage[left=2cm,right=2cm,top=2cm,bottom=2cm]{geometry}
\usepackage{multicol}
\author{Iker M. Canut}
\title{Unidad 4: Relaciones\\\'Algebra y Geometr\'ia Anal\'itica I (R-111)\\Licenciatura en Ciencias de la Computaci\'on}
\date{2020}
\newcommand*{\QEDA}{\null\nobreak\hfill\ensuremath{\blacksquare}}
\newcommand*{\QEDB}{\null\nobreak\hfill\ensuremath{\square}}
\begin{document}
\maketitle
\newpage

\section{Definiciones}
\noindent Dados dos conjuntos $A$ y $B$, un \textbf{par ordenado} es un objeto de la forma $(a,b)$ donde $a \in A$ y $b \in B$. Si $(a,b)$ y $(c,d)$ son dos pares ordenados, $(a,b) = (c,d) \iff a=c \land b=d$.\\

\noindent Dados dos conjuntos $A$ y $B$, llamaremos \textbf{producto cartesiano}, $A\times B$, al conjunto formado por los pares ordenados $(a,b)$ tales que $a\in A \land b\in B$. Es decir: $A\times B = \{(a,b):a\in A, b\in B\}$.\\

\noindent Recordando que $(A=B \iff A\subseteq B \land B\subseteq A)$, sean $A,B, C$ conjuntos, entonces:
\begin{itemize}
\item $A \times (B \cap C) = (A \times B) \cap (A \times C)$\\
$\subseteq)$ Sea $(x,y) \in A \times (B \cap C) \Rightarrow x \in A \land y \in B \land y \in C$. Por definici\'on de producto cartesiano, $(x,y) \in A\times B \land (x,y)\in A\times C$. Por lo tanto, $(x,y) \in A \times B \cap A \times C \therefore A \times (B \cap C) \subseteq A \times B \cap A \times C$.\\
$\supseteq)$ Sea $(x,y) \in (A\times B) \cap (A\times C) \Rightarrow x \in A \land y \in B \land y \in C \therefore y \in B\cap C$. Por definici\'on de producto cartesiano, $(x,y) \in A \times (B \cap C) \therefore (A\times B) \cap (A \times C) \subseteq A \times (B \cap C)$. \QEDA
\item $A \times (B \cup C) = (A \times B) \cup (A \times C)$\\
$\subseteq)$ Sea $(x,y) \in A \times (B \cup C) \Rightarrow x \in A \land (y \in B \lor y \in C)$. Por definici\'on de producto cartesiano, $(x,y) \in A\times B \lor (x,y)\in A\times C$. Por lo tanto, $(x,y) \in A \times B \cup A \times C \therefore A \times (B \cup C) \subseteq A \times B \cup A \times C$.\\
$\supseteq)$ Sea $(x,y) \in (A\times B) \cup (A\times C) \Rightarrow x \in A \land (y \in B \lor y \in C) \therefore y \in B\cup C$. Por definici\'on de producto cartesiano, $(x,y) \in A \times (B \cup C) \therefore (A\times B) \cup (A \times C) \subseteq A \times (B \cup C)$. \QEDA
\item $A \times (B - C) = (A \times B) - (A \times C)$\\
$\subseteq) $ Sea $(x,y) \in A \times (B - C) \Rightarrow x \in A \land y \in (B - C) \Rightarrow x \in A \land y \in B \land y \not \in C \Rightarrow\\ (x \in A \land y \in B) \land (x \in A \land y \not \in C)$ que por definici\'on de producto cartesiano, llegamos a que $(x,y) \in A \times B \land (x,y) \not \in A \times C \therefore (x,y) \in (A \times B) - (A \times C) \therefore A \times (B - C) \subseteq (A \times B) - (A \times C)$.\\
$\supseteq) $ Sea $(x,y) \in (A \times B) - (A \times C) \Rightarrow (x,y) \in A\times B \land (x,y) \not \in A \times C$. Es decir, \\$x \in A \land y \in B \land (x \not \in A \lor y \not \in C) \Rightarrow x \in A \land y \in B \land y \not \in C \Rightarrow x \in A \land y \in B - C$. Por definici\'on de producto cartesiano, tenemos que $(x,y) \in A \times (B - C)$. \QEDA
\item $A \times B \subseteq C \times D \iff A \subseteq C \land B \subseteq D $\\
$\Rightarrow)$ Sea $(x,y) \in A \times B \Rightarrow (x,y) \in C \times D$. Es decir, $x \in A \land y \in B \Rightarrow x \in C \land y \in D$. Luego, \\ $[x \in A \Rightarrow x \in C] \land [y \in B \Rightarrow y \in D] \therefore A \times B \subseteq C \times D \Rightarrow A \subseteq C \land B \subseteq D$\\
$\Leftarrow)$ $A \subseteq C \land B \subseteq D$, entonces $x \in A \Rightarrow x \in C \land y \in B \Rightarrow y \in D$. Sea $(x,y) \in A\times B$, $x \in A \land y \in B$ \\ $\therefore x \in C \land y \in D$ y de esta manera $(x,y) \in C\times D$, i.e $A \subseteq C \land B \subseteq D \Rightarrow A \times B \subseteq C \times D$ \QEDA
\end{itemize}


\noindent Una \textbf{relaci\'on} de un conjunto A en un conjunto B es un subconjunto $R$ de $A \times B$. Si $(a,b) \in R$, se dice que $a$ est\'a relacionado con $b$ por $R$, y se nota $aRb$.\\

\noindent Sea $R \subseteq A \times B$, $X \subseteq A$, $Y \subseteq B$
\begin{itemize}
\item El \textbf{dominio} de $R$ es: \hfill $Dom(R) = \{ a \in A:(a,b) \in R, \text{ para alg\'un } b\in B \}$
\item La \textbf{imagen} de $R$ es: \hfill $Im(R) = \{ b\in B : (a,b)\in R, \text{ para alg\'un } a \in A \}$
\item El \textbf{conjunto imagen} de $X$ por $R$ es: \hfill $R(X) = \{ b\in B : (a,b) \in R, \text{ para alg\'un } a \in X \}$
\item El \textbf{conjunto preimagen} de $Y$ por $R$ es: \hfill $R^{-1}(Y) = \{ a\in A : (a,b) \in R, \text{ para alg\'un } b \in Y \}$
\item La \textbf{inversa} de $R$, $R^{-1}: B\rightarrow A$ definida por: \hfill $R^{-1} = \{ (x,y):(y,x) \in R \}$\\
\end{itemize}

\noindent \textbf{Nota}: Sea $R \subseteq A \times B, x\in A, X\subseteq A$, notar que:
\begin{itemize}
\item $R^{-1}(x)$ es la \textit{preimagen del elemento} $x$ por $R$.
\item $R^{-1}(X)$ es la \textit{preimagen del subconjunto} $X$ por $R$.
\item $R^{-1}$ es la \textit{relaci\'on inversa} de R.\\
\end{itemize}

\noindent Sea $R \subseteq A \times B$, entonces $(R^{-1})^{-1} = R$ \\
$(R^{-1})^{-1} = \{(x,y): (y,x) \in \{ (y,x) : (x,y) \in R \} \} = \{(x,y) : (x,y) \in R\} = R$ \QEDA\\

\noindent Sea $R \subseteq A \times B$ y $S \subseteq B \times C$, la relaci\'on \textbf{composici\'on} de $R$ en $S$, notada como $S \circ R$, es una relaci\'on de $A$ en $C$ definida por $x(S \circ R)y \iff \exists u \in B : xRu \land uSy$
$$ S \circ R = \{ (x,y) \in A\times C : (x,u) \in R \land (u,y) \in S, \text{ para algun } u \in B \} $$

\begin{itemize}
\item $T \circ (S \circ R) = (T \circ S) \circ R$\\
$\subseteq)\ \forall (x,w) \in [(T \circ S) \circ R]
\Rightarrow (x,y) \in R \land (y,w) \in (T \circ S)
\Rightarrow (x,y) \in R \land [(y,z) \in S \land (z,w) \in T]\\
\Rightarrow [(x,y) \in R \land (y,z) \in S] \land (z,w) \in T
\Rightarrow (x,z) \in (S \circ R) \land (z,w) \in T
\Rightarrow (x,w) \in T \circ (S \circ R)$\\
$\supseteq)\ \forall (x,w) \in [T \circ (S \circ R)]
\Rightarrow (x,z) \in (S \circ R) \land (z,w) \in T
\Rightarrow [(x,y) \in R \land (y,z) \in S] \land (z,w) \in T\\
\Rightarrow (x,y) \in R \land [(y,z) \in S \land (z,w) \in T]
\Rightarrow (x,y) \in R \land (y,w) \in (T \circ S)
\Rightarrow (x,w) \in (T \circ S) \circ R\ \ 
$\QEDA\\
\item $(S \circ R)^{-1} = R^{-1} \circ S^{-1}$\\
$ \subseteq)\ \forall (x,z) \in (S \circ R)^{-1}
\Rightarrow (z,x) \in (S \circ R)
\Rightarrow (z,y) \in R \land (y,x) \in S
\Rightarrow (y,z) \in R^{-1} \land (x,y) \in S^{-1}\\
\Rightarrow (x,y) \in S^{-1} \land (y,z) \in R^{-1}
\Rightarrow (x,z) \in R^{-1} \circ S^{-1}$\\
$ \supseteq)\ \forall (x,z) \in R^{-1} \circ S^{-1}
\Rightarrow (x,y) \in S^{-1} \land (y,z) \in R^{-1}
\Rightarrow (y,x) \in S \land (z,y) \in R\\
\Rightarrow (z,y) \in R \land (y,x) \in S
\Rightarrow (z,x) \in S \circ R
\Rightarrow (x,z) \in (S \circ R)^{-1}
$
\QEDA\\
\end{itemize}

Cabe aclarar que la composici\'on de funciones no es conmutativa.

\newpage
\section{Relaciones en un conjunto}
\noindent Sea $R \subseteq A \times A$, y $a,b,c \in A$, se dice que $R$ es:
\begin{itemize}
\item \textbf{Reflexiva}: si $(a,a) \in R\ \forall a \in A$
\item \textbf{Sim\'etrica}: si $(a,b) \in R \Rightarrow (b,a \in A)$
\item \textbf{Antisim\'etrica}: $(a,b) \in R \land a \not = b \Rightarrow (b,a) \not \in R$, equivalentemente, $(a,b) \in R \land (b,a) \in R \Rightarrow a = b$
\item \textbf{Transitiva}: si $(a,b) \in R \land (b,c) \in R \Rightarrow (a,c) \in R$
\end{itemize}

\section{Relaciones de Orden}
\noindent Una relaci\'on $R$ en $A$ es una relaci\'on de orden si es \textbf{reflexiva}, \textbf{antisim\'etrica} y \textbf{transitiva} (R.A.T.)\\

\noindent Si $(a,b) \in R$, se dice que $a$ es \textbf{anterior} a $b$ y se nota $a \prec b$.\\

\noindent Al par $(A,R)$ o $(A, \prec)$ se lo llama \textbf{conjunto ordenado}.\\

\noindent Sea $(A, \prec)$, dos elementos distintos $x, y \in A$ son \textbf{comparables} si $x \prec y$ o si $y \prec x$. Un conjunto ordenado es \textbf{totalmente ordenado} si todo par de elementos es comparable, y se dice que es un \textbf{orden total}.\\

\noindent Sea $(A,R)$ un conjunto parcialmente ordenado, decimos que es un \textbf{ret\'iculo} si dados $x,y\in A$, existen en $A$ el $\sup\{x,y\}$ y el $\inf\{x,y\}$.\\

\noindent Sea $(A, \prec)$, y $B \subseteq A$, el \textbf{orden inducido} por $R$ en $B$ es $R_B = R \cap (B\times B)$, es decir, sea $x,y \in B$, $xR_By \iff xRy$. $(B,S)$ es un \textbf{subconjunto ordenado} de $(A,R)$ si $B \subseteq A$ y $S = R_B$. \\

\noindent Ademas, si $R_B$ es un orden total en $B$, $(B, R_B)$ se llama subconjunto ordenado de $(A, R)$ o \textbf{cadena}.\\

\noindent \textbf{Diagrama de Hasse}: se dibuja como un grafo, y convenimos que no se dibujan las flechas correspondientes a $(a,a)$, ni la flecha $(a,c)$ cuando $a \prec b$ y $b \prec c$.\\

\noindent Sea $(A, \prec)$ y $B \subseteq A$:
\begin{itemize}
\item $a \in A$ es \textbf{minimal} si $\forall x \in A : x\prec a$, se tiene que $x = a$.
\item $a \in A$ es \textbf{maximal} si $\forall x \in A : a\prec x$, se tiene que $x = a$.\\

\item $a \in A$ es \textbf{m\'inimo} si $a \prec x \forall x \in A$
\item $a \in A$ es \textbf{m\'aximo} si $x \prec a \forall x \in A$\\

\item $a \in A$ es \textbf{cota inferior} para $B$ si $a \prec x$ $\forall x \in B$. Una cota inferior $'a$ es el \textbf{\'infimo} de $B$ si $a \prec a'$ para toda cota inferior de $B$.
\item $a \in A$ es \textbf{cota superior} para $B$ si $x \prec a$ $\forall x \in B$. Una cota superior $'a$ es el \textbf{supremo} de $B$ si $a' \prec a$ para toda cota inferior de $B$.\\
\end{itemize}
\noindent Un conjunto puede tener m\'as de un minimal o maximal, pero si tiene m\'aximo, m\'inimo, supremo o \'infimo, estos es \'unico. Adem\'as, si tiene alguna cota se dice que est\'a \textbf{acotado}.

\newpage
\section{Relaciones de Equivalencia}
\noindent Una relaci\'on $R$ en $A$ es de equivalencia si es \textbf{reflexiva}, \textbf{sim\'etrica} y \textbf{transitiva} (R.S.T.)\\

\noindent Si $(a,b) \in R$, se dice que $a$ es equivalente a $b$ y se nota $a \sim b$.\\

\noindent Dada una relaci\'on de equivalencia $R$ en un conjunto $A$ y $a \in A$, el conjunto $R(a)$ se llama \textbf{clase de equivalencia} de $a$ y se nota $[a]$.

$$[a] = \{ x \in A : (a,x) \in \mathbb{R} \}$$

\noindent Observemos que como es sim\'etrica, $[a] = \{ x \in A : (x,a) \in \mathbb{R} \}$ tambien vale. Todo elemento $x \in [a]$ se dice que es un \textbf{representante} de esa clase de equivalencia.

\begin{itemize}
\item $[a] \not = \emptyset$\\
Sabemos que $a \in [a]$ ya que es reflexiva, luego $[a] \not = \emptyset$ \QEDA
\item $(a,b) \in R \iff [a]=[b]$\\
$\Rightarrow)\ \subseteq)\ x \in [a] \Rightarrow (a,x) \in R \Rightarrow (x,a) \in R$ (por simetr\'ia). Por H) $(a,b) \in R \Rightarrow (x,b) \in R$ (por \\ \indent $\ \ \ \ \ \ $ transitividad) $\therefore x \in [b]$.\\
\indent $\ \ \ $ $ \ \ \supseteq)\ x \in [b] \Rightarrow (b,x) \in R \Rightarrow (x,b) \in R$. Como $(b,a) \in R \Rightarrow (x,a) \in R \therefore x \in [a]$\\
$\Leftarrow)$ Sean $a,b \in A : [a]=[b]$ luego $a \in [b] \therefore (a,b) \in R$ \QEDA
\item $(a,b) \not \in R \iff [a]\cap[b] = \emptyset$\\
$\Rightarrow)$ $(a,b) \not \in R \Rightarrow [a]\cap[b] = \emptyset$ es equivalente a $[a]\cap[b] \not = \emptyset \Rightarrow (a,b) \in R$.\\
\indent $\ \ \ \ $ Si existe $x \in [a]\cap[b]$, luego $(a,x) \in R \land (b,x) \in R \Rightarrow (a,x) \in R \land (x,b) \in R \therefore (a,b) \in R$\\
$\Leftarrow)$ $[a]\cap[b] = \emptyset \Rightarrow (a,b) \not \in R$ es equivalente a $(a,b) \in R \Rightarrow [a]\cap[b] \not = \emptyset$. Y por el punto anterior,\\ \indent $\ \ \ \ $ sabemos que $[a]=[b]$. Y particularmente $a \in [a]\cap[b]$ \QEDA
\end{itemize}

\noindent Es decir, todo elemento de $A$ pertenece a alguna clase y dos clases de equivalencia, o bien son iguales o son conjuntos disjuntos.\\

\noindent Una \textbf{partici\'on} $P$ de un conjunto $A$ es una colecci\'on de conjuntos no vacios $\{ X_1, X_2, ...\}$ tales que:
\begin{itemize}
\item $i \not = j \Rightarrow X_i \cap X_j = \emptyset$,
\item $\forall a \in A, \exists X_i \in P : a \in X_i$\\
\end{itemize}

\noindent Sea $P$ una partici\'on de $A \not = \emptyset$, existe una \'unica relaci\'on de equivalencia en $A$ cuyas clases de equivalencia son los elementos de $P$.\\
\noindent\textbf{Dem/} Se define una relaci\'on $R$ en $A$ tal que $(x,y) \in R \iff$ existe $X_i \in P$ tal que $x,y \in X_i$, i.e:\\
Sea $P = \{A_1, A_2, ..., A_n\}$ una partici\'on de $A$, se define $R = \{(x,y) : x \in A_i, y \in A_i, A_i \in P\}$. Luego,
\begin{itemize}
\item $R$ es reflexiva, como $A \not = \emptyset$, existe $x \in A$. Y como $P$ es una partici\'on de $A$, $x \in A_i, A_i \in P$
\item $R$ es sim\'etrica pues $xRy$, $x \in A_i \land y \in A_i$, y por lo tanto $yRx$.
\item $R$ es transitiva, si $xRy \land yRz$, entonces $x \in A_i \land y \in A_i$ y tambi\'en $y \in A_j \land z \in A_j$. Y como $y \in A_i \land y \in A_j \Rightarrow A_i = A_j \therefore x \in A_i \land z \in A_i$, de donde $xRz$. \QEDA\\
\end{itemize}

\noindent Sea $R$ una relaci\'on de equivalencia en $A$, llamamos \textbf{conjunto cociente} de $A$ por $R$, y lo notamos $A|_R$ al conjunto cuyos elementos son las clases de equivalencia de $A$ definidadas por $R$: $$A|_R = \{ [a] : a \in A \}$$
\end{document}