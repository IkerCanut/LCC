\documentclass[11pt,a4paper]{article}
\usepackage[utf8]{inputenc}
\usepackage[spanish]{babel}
\usepackage{amsmath}
\usepackage{amsfonts}
\usepackage{amssymb}
\usepackage{graphicx}
\usepackage[left=2cm,right=2cm,top=2cm,bottom=2cm]{geometry}
\usepackage{multicol}
\author{Iker M. Canut}
\title{Unidad 5: Funciones\\\'Algebra y Geometr\'ia Anal\'itica}
\begin{document}
\maketitle
\newpage

\section{Funciones}
Dados $A$ y $B$ conjuntos no vacios, una \textbf{funci\'on} de $A$ en $B$ es una relaci\'on de $A$ en $B$ que verifica que cada elemento de $A$ es exactamente una vez primera componente de un par ordenado de la relaci\'on. Lo notamos $f : A \rightarrow B$. En otras palabras, se tiene que cumplir:
\begin{itemize}
\item Para cada $a \in A$ existe $b \in B : (a,b)$ est\'a en la relaci\'on.
\item No puede haber dos pares $(a, b_1)$ y $(a,b_2)$ con $b_1 \not = b_2$ en la relaci\'on.
\end{itemize}
Podemos escribir $f(a) = b$ para indicar que la \textbf{imagen} de $a\in A$ es $b\in B$\\

\noindent Si la relaci\'on que es funci\'on es un subconjunto $A\times B$, diremos que:\\
\indent El \textbf{dominio} de la funci\'on es $A$ y el \textbf{codominio} de la funci\'on es $B$.\\

Si $f : A \rightarrow B$ y $A_1 \subseteq A$,
$f(A_1) = \{ b \in B : f(a) = b, a \in A_1 \}$
y decimos que es la imagen de $A_1$ por medio de $f$. Si $A_1 = A$, notamos $f(A) = Im(f)$ y ese es el \textbf{conjunto imagen} de $f$.

\noindent \dotfill

Decimos que $f : A \rightarrow B$ es \textbf{inyectiva} si cada elemento de $B$ aparece a lo sumo una vez como segunda componente de los pares ordenados de la relaci\'on: $$\forall a_1, a_2 \in A, f(a_1) = f(a_2) \Rightarrow a_1 = a_2$$
\noindent \dotfill

Sea $f:A\rightarrow B, A_1, A_2 \subseteq A$:
\begin{multicols}{2}
\begin{itemize}
\item $f(A_1 \cup A_2) = f(A_1) \cup f(A_2)$
\item $f(A_1 \cap A_2) \subseteq f(A_1) \cap f(A_2)$
\end{itemize}
\end{multicols}
\noindent \dotfill

\noindent Sea $f:A\rightarrow B, A_1 \subseteq A$:
\begin{itemize}
\item $f|_{A_1} : A \rightarrow B : f|_{A_1}(a) = f(a)$ si $a \in A_1$, es la \textbf{restricci\'on} de $f$ a $A_1$.
\end{itemize}
Sea $f:A\rightarrow B, A \subseteq A_2$
\begin{itemize}
\item $g : A_2 \rightarrow B : g(a) = f(a)$ si $a \in A$ es una \textbf{extensi\'on} de $f$ a $A_2$.
\end{itemize}
\noindent \dotfill

Sea $f: A \rightarrow B, B_1 \subseteq B$, la \textbf{preimagen} de $B_1$ por medio de $f$, notada como $f^{-1}(B_1)$ es:
$$f^{-1}(B_1) = \{ x \in A : f(x) \in B_1 \}$$
Notar que la preimagen es otro conjunto.\\

\begin{itemize}
\item $f^{-1}(B_1 \cup B_2) = f^{-1}(B_1) \cup f^{-1}(B_2)$
\item $f^{-1}(B_1 \cap B_2) = f^{-1}(B_1) \cap f^{-1}(B_2)$
\item $f^{-1}(\overline{B_1}) = \overline{f^{-1}(B_1)}$
\end{itemize}
\noindent \dotfill

Diremos que $f : A \rightarrow B$ es \textbf{suryectiva} si cada elemento de $B$ aparece una vez como segunda componente de los pares ordenados de la relaci\'on: $f(A) = Im(f) = B$. Es decir:$$ \text{Dado } y \in B, \exists x \in A : f(x) = y $$
\noindent \dotfill

Luego, una funci\'on es \textbf{biyectiva} si es inyectiva y suryectiva.

\noindent \dotfill
\newpage

Sean $f$ y $g$ dos funciones, tales que $Im(f) \cap Dom(g) \not = \emptyset$, se define la \textbf{composici\'on} de $g$ con $f$, y se lo nota $g \circ f$, a la funci\'on con dominio: $Dom (g \circ f) = \{ x \in Dom(f) : f(x) \in Dom(g) \}$ y tal que $$(g \circ f) (x) = g(f(x)), \forall x \in Dom(g \circ f)$$

Bajo la condici\'on $Im(f) \cap Dom(g) \not = \emptyset$ decimos que la composici\'on de $g$ con $f$ es posible ya que su dominio es no vacio. Adem\'as, hay funciones para las cuales $(g\circ f)$ esta bien definida, pero $(f\circ g)$ no lo est\'a. Tambi\'en pueden existir y ser distintas. Por lo tanto, no es conmutativa.\\

Pero la composici\'on de funciones si es asociativa, es decir, $(h \circ g) \circ f = h \circ (g \circ f)$\\

Sea $f: A \rightarrow A$, la composici\'on $(f \circ f)$ es posible y se nota $f^2$. Recursivamente, $f^n = (f \circ f^{n-1})$\\

Si $f: A \rightarrow B$ y $g: B \rightarrow C$ son inyectivas (resp. suryectivas), entonces $(g \circ f) : A \rightarrow C$ es inyectiva (resp. suryectiva.)

\noindent \dotfill

Una funci\'on $f : A \rightarrow B$ es \textbf{inversible} si existe $g : B \rightarrow A :$ $$(g\circ f) = id_A \text{  y  } (f\circ g) = id_B$$
Luego, si $f$ es inversible, entonces $g$ tambi\'en lo es.\\
Tambi\'en, si $f : A \rightarrow B$ es inversible y $g : B \rightarrow A$ es una inversa de $f$, entonces es unica.\\

\begin{itemize}
\item $f$ es inversible $\iff$ $f$ es biyectiva.
\item $f : A \rightarrow B, g: B \rightarrow C$ son inversibles, entonces $(g \circ f)$ es inversible y $(g \circ f)^{-1} = f^{-1} \circ g^{-1}$.
\end{itemize}

La preimagen SIEMPRE existe y es un conjunto. La funci\'on inversa (si existe) es una funci\'on.\\

Sea $f:A \rightarrow B$, $A$ y $B$ finitos, $|A| = |B|$, son equivalentes:
\begin{multicols}{3}
\begin{itemize}
\item $f$ es inyectiva
\item $f$ es suryectiva
\item $f$ es inversible
\end{itemize}
\end{multicols}

\noindent \dotfill

\newpage
\section{Operaciones}
Dados $A$ y $B$ no vacios, una funci\'on $f : A \times A \rightarrow B$ es una \textbf{operaci\'on binaria} en $A$. Si adem\'as, $Im(f) \subseteq A$, la operaci\'on es \textbf{cerrada} en $A$.\\

Una funci\'on $g : A \rightarrow A$ es una \textbf{operaci\'on monaria} (unaria) en $A$.\\

\noindent \dotfill


Dada $f: A \times A \rightarrow B$, operaci\'on binaria en $A$, 
\begin{itemize}
\item $f$ es \textbf{conmutativa} si $f(a_1, a_2) = f(a_2, a_1), \forall (a_1, a_2) \in A \times A$.
\item Si $f$ es cerrada, entonces $f$ es \textbf{asociativa} si $f(f(a,b), c) = f(a,f(b,c)), \forall a,b,c \in A$.
\end{itemize}
Podemos notar $f(a,b) = a \otimes b$, y estas propiedadades son mas amigables: Por ejemplo la asociatividad ser\'ia: $(a \otimes b) \otimes c = a \otimes (b \otimes c)$.\\

\noindent \dotfill

Luego, dado $f : A \times A \rightarrow A$, decimos que tiene \textbf{neutro} si existe $a_0 \in A$ tal que $$f(a, a_0) = f(a_0, a) = a, \forall a \in A$$
Es decir, $$a \otimes a_0 = a_0 \otimes a = a$$

Adem\'as, si $f : A \times A \rightarrow A$ tiene neutro, \'este es \'unico.\\

\noindent \dotfill

Dada $f : A \times A \rightarrow A$, si $f$ posee neutro $x \in A$, decimos que la operaci\'on posee inversos si $$\forall a \in A \exists a' : f(a, a') = f(a', a) = x$$

Luego, si $f : A \times A \rightarrow A$ es una operaci\'on asociativa, con elemento neutro $x \in A$ que posee \textbf{inversos}, entonces cada elemento posee un \'unico inverso: Supongamos que tiene 2 inversos $a_1$ y $a_2$, $$a_1 = a_1 \otimes x = a_1 \otimes (a \otimes a_2) = (a_1 \otimes a) \otimes a_2 = x \otimes\ a_2 = a_2$$

\noindent \dotfill
\end{document}