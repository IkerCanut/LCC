\documentclass[11pt,a4paper]{article}
\usepackage[utf8]{inputenc}
\usepackage[spanish]{babel}
\usepackage{amsmath}
\usepackage{amsfonts}
\usepackage{amssymb}
\usepackage{graphicx}
\usepackage[left=2cm,right=2cm,top=2cm,bottom=2cm]{geometry}
\usepackage{multicol}
\author{Iker M. Canut}
\title{Unidad 1: Presentaci\'on Axiomatica de los N\'umeros Reales\\ Analisis Matem\'atico I}
\begin{document}
\maketitle
\newpage

\section{Axiomas de Cuerpo}
\begin{itemize}
\item Propiedad \textbf{Conmutativa}: $a+b = b+a$ y $a \cdot b = b \cdot a$
\item Propiedad \textbf{Asociativa}: $a+(b+c) = (a+b)+c$ y $a \cdot (b \cdot c) = (a \cdot b) \cdot c$
\item Propiedad \textbf{Distributiva}: $a\cdot (b+c) = a\cdot b + a\cdot c$
\item Existencia de \textbf{Elementos Neutros}: $\forall a \in \mathbb{R}, a+0 = a$ y $a\cdot 1 = a$
\item Existencia de \textbf{Elementos Opuestos}: $\forall a \in \mathbb{R}\ \exists b : a+b = b+a = 0$
\item Existencia de \textbf{Elementos Rec\'iprocos}: $\forall a \not = 0,\ \exists b : a\cdot b = b\cdot a = 1$
\end{itemize}
\noindent \dotfill
\begin{multicols}{3}
$a = a$\\
$a = b \Rightarrow b = a$\\
$a = b \land b = c \Rightarrow a = c$
\end{multicols}

\noindent \textbf{Teorema 1}: Propiedad Cancelativa de la Suma: $a+b = a+c \Rightarrow b = c$\\
\noindent \textbf{Corolario 1}: Unicidad del Elemento Neutro de la Suma. $a+0' = 0'+a = a \Rightarrow 0' = 0$\\
\noindent \textbf{Corolario 2}: Unicidad del Elemento Opuesto. $a+b = a+b' = 0 \Rightarrow b=b'$\\
\vspace{-0.9cm}
\begin{multicols}{3}
\noindent \textbf{Teorema 2}:\\ \\

\begin{itemize}
\item $-(-a)=a$
\item $-0 = 0$
\item $0 \cdot a = 0$
\item $a \cdot (-b) = (-a) \cdot b = -(a \cdot b)$
\item $(-a) \cdot (-b) = a \cdot b$
\item $a \cdot (b-c) = a \cdot c$
\end{itemize}
\end{multicols}
\noindent \textbf{Teorema 3}: Propiedad Cancelativa del Producto: $a \cdot b = a \cdot c \land a \not = 0 \Rightarrow b = c$\\
\noindent \textbf{Corolario 3}: Unicidad del Elemento Neutro del Producto. $a+0' = 0'+a = a \Rightarrow 0' = 0$\\
\noindent \textbf{Corolario 4}: Unicidad del Rec\'iproco. $\forall a \not = 0,$ existe un \'unico $b : a\cdot b = b\cdot a = 1$
\vspace{-0.4cm}
\begin{multicols}{3}
\noindent \textbf{Teorema 4}:\\ \\ \\ \\ \\ \\ \\ \\

\begin{itemize}
\item $0$ no tiene rec\'iproco.
\item $1^{-1} = 1$
\item $\dfrac{a}{1} = a$, si $a \not = 0$, $\dfrac{1}{a} = a^{-1}$
\item $a\cdot b = 0 \Rightarrow a=0 \lor b=0$
\item $-a = (-1) \cdot a$
\item $a\not = 0, b\not = 0, \left(\dfrac{a}{b}\right)^{-1} = \dfrac{a^{-1}}{b^{-1}}$
\item $b \not = 0 \land d \not = 0$:
\begin{itemize}
\item $(b\cdot d)^{-1} = b^{-1} \cdot d^{-1}$
\item $\dfrac{a}{b}+\dfrac{c}{d} = \dfrac{a\cdot d + b\cdot c}{b\cdot d}$
\item $\dfrac{a}{b} \cdot \dfrac{c}{d} = \dfrac{a\cdot c}{b \cdot d}$
\end{itemize}
\end{itemize}
\end{multicols}

\section{Axiomas de Orden}
\begin{itemize}
\item Si $a,b \in \mathbb{R}^+_0 \Rightarrow\ a+b \in \mathbb{R}^+_0$ y $a \cdot b \in \mathbb{R}^+_0$
\item $\forall a \in \mathbb{R} : a \not = 0 \Rightarrow$ o bien $a \in \mathbb{R}^+$ o $-a \in \mathbb{R}^+$
\item $0 \not \in \mathbb{R}$
\end{itemize}
\noindent \dotfill
\begin{itemize}
\item $a < b \Rightarrow b - a \in \mathbb{R}^+$
\item $a > b \Rightarrow a - b \in \mathbb{R}^+$
\item $a \leq b \Rightarrow$ o bien $b - a \in \mathbb{R}^+$ o $a = b$
\item $a \geq b \Rightarrow$ o bien $a - b \in \mathbb{R}^+$ o $a = b$
\item $a > 0 \iff a \in \mathbb{R}^+$
\end{itemize}

\noindent \textbf{Teorema 5}: Propiedad de Tricotom\'ia: Dados $a,b \in \mathbb{R}$ sucede solo una de las siguientes proposiciones:
\begin{multicols}{3}
$a < b$\\
$a > b$\\
$a = b$
\end{multicols}
\noindent \textbf{Teorema 6}: Propiedad Transitiva de la Relaci\'on Menor: Si $a < b \land b < c \Rightarrow a < c$
\vspace{-0.4cm}
\begin{multicols}{3}
\noindent \textbf{Teorema 7}:\\ \\ \\ \\ \\ \\ \\ \\ \\

\begin{itemize}
\item $a < b \Rightarrow a + c < b + c$
\item $a < b \land c < d \Rightarrow a + c < b + d$
\item $a < b \land c > 0 \Rightarrow a \cdot c < b \cdot c$
\item $a < b \land c < 0 \Rightarrow a \cdot c > b \cdot c$
\item $a \not = 0 \Rightarrow a^2 > 0$
\item $1 > 0$
\item $a < b \Rightarrow -b < -a$
\item $a \cdot b > 0 \iff a$ y $b$ son los dos positivos o los dos negativos.
\item $a \cdot b < 0 \iff a$ positivo y $b$ negativo, o $a$ negativo y $b$ positivo.
\item $a > 0 \iff \dfrac{1}{a} > 0$
\item $0 < a < b \Rightarrow 0 < \dfrac{1}{b} < \dfrac{1}{a}$
\end{itemize}
\end{multicols}

\subsection{N\'umeros Naturales, Enteros, Racionales e Irracionales}
\noindent \textbf{N\'umeros Naturales}: $\mathbb{N}$. El conjuntos inductivo m\'as peque\~{n}o:
\begin{enumerate}
\item El n\'umero $1$ pertenece al conjunto.
\item Si $a$ pertenece al conjunto, $a+1$ tambi\'en pertenece.
\end{enumerate}
\indent Destacamos que $1$ es el primer elemento de $\mathbb{N}$, i.e es el menor. Ergo, si $a < 1 \Rightarrow a \not \in \mathbb{N}$ \\

\noindent \textbf{N\'umeros Enteros}: $\mathbb{Z} = \{ x \in \mathbb{R} : x \in \mathbb{N} \lor -x \in \mathbb{N} \lor x = 0 \}$\\
\indent La suma, la diferencia y el producto son operaciones cerradas en $\mathbb{Z}$.\\

\noindent \textbf{N\'umeros Racionales}: $\mathbb{Q} = \left\{ x \in \mathbb{R} : \exists p, q \in \mathbb{Z}, q \not = 0 : x = \dfrac{p}{q} \right\}$\\

\noindent \dotfill\\
Observaciones: • $\mathbb{Z} \subset \mathbb{Q}$ • Dados $a,b \in \mathbb{R}, c,d \in \mathbb{R}-\{0\}, \dfrac{a}{c} = \dfrac{b}{d} \iff ad=bc$
\subsection{Representaci\'on Geometrica de los numeros reales: la recta real}
En una recta se elige un punto para representar al $0$ y otro punto distinto para representar al $1$ (esta elecci\'on fija la escala). Cada punto de la recta representa a un \'unico n\'umero real y cada n\'umero real est\'a representado por un \'unico punto de la recta.
\begin{enumerate}
\item Si los puntos $A$ y $B$ representan los n\'umeros reales $a$ y $b$, $A$ est\'a a la izquierda de $B$ $\iff$ $a<b$.
\item Si los puntos $A,B,C,D$ representan a los n\'umeros reales $a,b,c,d$. con $a<b$ y $c<d$, entonces $\overline{AB}$ y $\overline{CD}$ son congruentes $\iff$ $b-a = d - c$.
\end{enumerate}
Adem\'as, los n\'umeros positivos quedan a la derecha del $0$, y los negativos a la izquierda del mismo.

\subsection{Intervalos Reales}
\begin{multicols}{2}
\begin{itemize}
\item $(a,b) = \{ x \in \mathbb{R} : a < x < b \}$
\item $[a,b) = \{ x \in \mathbb{R} : a \leq x < b \}$
\item $(a,b] = \{ x \in \mathbb{R} : a < x \leq b \}$
\item $[a,b] = \{ x \in \mathbb{R} : a \leq x \leq b \}$

\item $(a,+\infty) = \{ x \in \mathbb{R} : a < x \}$
\item $[a,+\infty) = \{ x \in \mathbb{R} : a \leq x \}$
\item $(-\infty,b) = \{ x \in \mathbb{R} : x < b \}$
\item $(-\infty,b] = \{ x \in \mathbb{R} : x \leq b \}$
\end{itemize}
\end{multicols}

\subsection{Valor absoluto de un n\'umero}
Dado $x \in \mathbb{R}$, su valor absoluto es el n\'umero real $|x| : $
\[|x| = \left\{
\begin{array}{ll}
x  & \text{, si } x \geq 0\\
-x & \text{, si } x < 0\\
\end{array} \right.
\]
Geom\'etricamente, $|x|$ es la distancia en la recta real entre los puntos $0$ y $x$. Tambi\'en puede verse que la distancia entre dos puntos cualesquiera $x,y \in \mathbb{R}$ est\'a dada por el valor $|x - y| = |y - x|$.\\

\noindent \textbf{Proposici\'on}:
\begin{multicols}{2}
\begin{itemize}
\item $|x| \geq 0$
\item $|x| = 0 \iff x = 0$
\item $|x| = |-x|$
\item $-|x| \leq x \leq |x|$
\item Sea $a>0$: $|x| < a \iff -a < x < a$
\item Sea $a>0$: $|x| > a \iff x < -a\ \underline{\lor}\ a < x$
\item $|x+y| \leq |x| + |y|$
\item $|x \cdot y| = |x| \cdot |y|$
\item Sea $y \not = 0$, $\left|\dfrac{x}{y}\right| = \dfrac{|x|}{|y|}$
\end{itemize}
\end{multicols}

\section{Introducci\'on A10}
Sea $A$ un subconjunto no vacio de $\mathbb{R}$
\begin{itemize}
\item \textbf{Cota Superior}: Sea $b \in \mathbb{R}$, $b$ es una cota superior de $A$ si $a \leq b\ \forall a \in A$.
\item \textbf{Cota Inferior}: Sea $b \in \mathbb{R}$, $b$ es una cota inferior de $A$ si $a \geq b\ \forall a \in A$.\\

\item \textbf{Supremo}: $b$ es supremo de $A \iff (a \leq b\ \forall a \in A) \land (c < b \Rightarrow c$ no es una cota superior de $A)$.
\item \textbf{\'Infimo}: $b$ es \'infimo de $A \iff (b \leq a\ \forall a \in A) \land (b < c \Rightarrow c$ no es una cota inferior de $A)$.\\

\item \textbf{M\'aximo}: $b$ es m\'aximo de $A$ si $a \leq b\ \forall a \in A$ $\land$ $b \in A$.
\item \textbf{M\'inimo}: $b$ es m\'inimo de $A$ si $b \leq a\ \forall a \in A$ $\land$ $b \in A$.
\end{itemize}

\noindent \dotfill

\noindent \textbf{Teorema 8}: Unicidad del supremo: Dos n\'umeros distintos no pueden ser supremos de un mismo conjunto. Por esto tenemos una notaci\'on: $b = sup(A)$.\\
\noindent \textbf{Teorema 9}: Caracterizaci\'on del Supremo: $b = sup(A) \iff b$ es una cota superior de $A$ tal que $\forall \epsilon > 0$ existe algun elemento $a \in A$ tal que $b - \epsilon < a$.\\
Demostraci\'on:\\
$\Rightarrow )$ Supongamos que no ocurre, entonces $a \leq b - \epsilon$ y es cota superior de $A$, pero contradice que $b$ es supremo de $A$, porque $a \leq b - \epsilon < b$.\\
$\Leftarrow )$ Queremos demostrar que $c < b$ no es cota superior de $A$. Sea $\epsilon_c = b - c > 0$ y como $\exists a \in A : b - \epsilon_c < a$, entonces $a > b - \epsilon_c = b - (b - c) = c$ i.e c no es cota superior de A. Luego, $b = sup(A)$.\\
\noindent \textbf{Proposici\'on 3}: $b = max(A) \iff b \in A \land b = sup(A)$.\\
\noindent \textbf{Proposici\'on 4}: $b = min(A) \iff b \in A \land b = inf(A)$.

\newpage

\subsection{Axioma del Supremo}
Todo conjunto no vac\'io de n\'umeros reales que sea acotado superiormente tiene un supremo.\\

\noindent \textbf{Teorema 10}: Existencia de Raices Cuadradas: Dado $a \geq 0$, existe un \'unico $x \in \mathbb{R}$ : $x \geq 0$ y $x^2 = a$. Si $a=0$ es trivial. Si $a > 0$, sabemos que tiene dos soluciones (solo una es positiva). Se define el conjunto $S_a = \{ x \in \mathbb{R} : x^2 \leq a \}$. Vemos que $S_a \not = \emptyset$ y que est\'a acotado superiormente. Luego existe $b = sup(A)$. Luego, por tricotom\'ia sacamos que $b^2 = a$.\\

\noindent \textbf{Teorema 11}: Propiedad Arquimediana de los Reales: Sean $x,y \in \mathbb{R}, x > 0 \Rightarrow \exists n \in \mathbb{N} : y < n \cdot x$. Va por absurdo, suponiendo $n\cdot x \leq y\ \forall n \in \mathbb{N}$. Definimos $S = \{ n\cdot x : n \in \mathbb{N} \}$. S no es vacio, ergo existe $b=sup(S)$. Luego $\exists a \in S : b - x < a$ (Caracterizaci\'on). Y se podria escribir como $a = m \cdot x$, $m \in \mathbb{N}$. Es decir, $b < mx + x = (m+1) \cdot x$. Pero $(m+1) \cdot x \in S$, y $b$ no es cota superior de $S$, lo que contradice que $b = sup(S)$. Se contradice por suponer $S$ acotado superiormente. Luego $\exists n \in \mathbb{N} : y < n\cdot x$.\\

\noindent \textbf{Corolario 5}:
\begin{itemize}
\item $\forall y \in \mathbb{R}, \exists n \in \mathbb{N} : y < N$.
\item $\mathbb{N}$ no est\'a acotado superiormente.
\item Sea $x > 0$, $\exists n \in \mathbb{N} : \dfrac{1}{n} < x$
\item $x,y,z \in \mathbb{R}, z > 0$, si $x \leq y < x + \dfrac{z}{n}$ $\forall n \in \mathbb{N}$ entonces $x = y$.\\

\item Si $|x| < \dfrac{1}{n}\ \forall n \in \mathbb{N}$, entonces $x = 0$.
\item Si $|x| < \epsilon\ \forall \epsilon > 0$ entonces $x = 0$.
\end{itemize}

\noindent \textbf{Teorema 12}: Si $A$ est\'a acotado inferiormente, entonces posee \'infimo.

\noindent \dotfill

Dado $x \in \mathbb{R}$, existe un \'unico n\'umero $p$ entero tal que $p \leq x < p + 1$ . Demostracion:
\begin{itemize}
\item Si $x \in \mathbb{Z}$, $p=x$ verifica.
\item Sino, si $0 < x < 1$, entonces $p = 0$ verifica.
\item Sino, sea $S = \{ n \in \mathbb{N} : x < n \}$ es distinto de $\emptyset$. Est\'a acotado inferiormente por $x$, y por la propiedad arquimediana, existe $n_0 > x$ y $n_0 \in S$. Luego existe un minimo $m$ y $m-1 \leq x < m$ $ \not \in S$. Luego, llamando $p = m-1$, tenemos que $p \leq x < p+1$, siendo $p$ \'unico.
\item Si $x < 0 \Rightarrow -x > 0$ y es an\'alogo.
\end{itemize}
Y queda demostrado que cuaquiera sea $x \in \mathbb{R}$, existe un unico $p \in \mathbb{Z} : $
$$p \leq x < p+1$$
que suele notarse como $[x]$ y se denomina \textbf{parte entera} de x:
$$[x] \leq x < [x] + 1$$
\end{document}