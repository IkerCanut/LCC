\documentclass[10pt]{article}
\usepackage{hyperref}
\hypersetup{
    colorlinks=true,
    linkcolor=black,
    filecolor=magenta,      
    urlcolor=cyan,
}

\usepackage{import}
\usepackage{siunitx}
\usepackage{esvect}
\usepackage{fourier}
\usepackage{amssymb}
\usepackage{amsmath}
\usepackage{multicol}
\usepackage{geometry}
\usepackage[framemethod=TikZ]{mdframed}
\geometry{
 a4paper,
 total={160mm,237mm},
 left=30mm,
 top=30mm,
}

\usepackage{tikz}
\usetikzlibrary{automata, positioning, calc, through, angles, quotes, intersections}
\usepackage{multirow}

\subimport{.}{environment}

\author{Luciano N. Barletta \& Iker M. Canut}
\begin{document}
\title{Práctica de Álgebra y Geometría 1}
\maketitle
\date
\newpage

\tableofcontents
\newpage

\section{Ejercicio 2}

\subsection{Inciso c}
Sean $P(x) = 3x^4 - x^2 + ix - 2$ y $Q(x) = 5x - 4 = 5(x-\dfrac{4}{5})$.\\
Por algoritmo de la división, existen únicos $C, R \in \C[x]$ tal que:
$$P = C.Q + R$$
$$P = C.5(x-\dfrac{4}{5}) + \dfrac{5R}{5}$$
$$\dfrac{P}{5} = C.(x-\dfrac{4}{5}) + \dfrac{R}{5}$$
Llamemos $P' = \dfrac{P}{5}$, $Q' = x-\dfrac{4}{5}$, $R' = \dfrac{R}{5}$. Por regla de Ruffini podemos encontrar $C$ y $R$ con la definición del teorema porque $Q'$ es de la forma $x-\alpha$. $C$ se define recursivamente:
\begin{align}
	C(x) =& \dfrac{3}{5}x^3 + \\
	& (\dfrac{0}{5} + \dfrac{4}{5}.\dfrac{3}{5})x^2 = \dfrac{12}{5^2}x^2 + \\
	& (-\dfrac{1}{5} + \dfrac{4}{5}.\dfrac{12}{5^2})x = \dfrac{23}{5^3}x + \\
	& \dfrac{i}{5} + \dfrac{4}{5}.\dfrac{23}{5^3} = \dfrac{4.23}{5^4} + \dfrac{1}{5}i
\end{align}
Y $R'$ siempre queda definido como $a_0 + \alpha.b_0$.
\begin{align}
	R' &= -\dfrac{2}{5} + \dfrac{4}{5}.(\dfrac{4.23}{5^4} + \dfrac{1}{5}i)\\
	&= \dfrac{4.4.23}{5^5} + \dfrac{4}{5^2}i - \dfrac{2}{5}\\
	&= -\dfrac{4.4.23 - 2.5^4}{5^5} + \dfrac{4}{5^2}i
\end{align}
Pero $R' = \dfrac{R}{5}$
$$R = -\dfrac{4.4.23-2.5^4}{5^4} + \dfrac{4}{5}i$$

\subsection{Inciso d}
Sean $P(x) = 3x^6 - x^4 + ix^3 - 2x^2$ y $Q(x) = 5x^3 - 4x^2 = 5x^2(x-\dfrac{4}{5})$.\\
Por algoritmo de la división, existen únicos $C, R \in \C[x]$ tal que:
$$P = C.Q + R$$
$$P = C.5x^2(x-\dfrac{4}{5}) + \dfrac{5x^2.R}{5x^2}$$
$$\dfrac{P}{5x^2} = C.(x-\dfrac{4}{5}) + \dfrac{R}{5x^2}$$
Llamemos $P' = \dfrac{P}{5x^2}$, $Q' = x-\dfrac{4}{5}$, $R' = \dfrac{R}{5x^2}$. Notamos que $P'$ y $Q'$ son iguales al inciso anterior, entonces $C$ y $R$ también son iguales.

\section{Ejercicio 3}
En los incisos $2c$ y $2d$ el resultado es el mismo, pues la división entre $P$ y $Q$ en ambos ejercicios es equivalente. En otras palabras, los polinomios están multiplicados por la misma expresión.

\section{Ejercicio 4}
\subsection{Inciso e}
Sea $P(x) = x^4 - ix^3 - ix + 1+i$.
Calcular $P(i+1)$ es lo mismo que averiguar el resto de dividir por $Q(x) = (x - i+1)$, por teorema del resto.

Aplicando el teorema de la regla de ruffini, podemos definir $C$ recursivamente:
\begin{align}
	C(x) =& x^3 +
	& (-i + (1+i).1)x^2 = x^2 +
	& (0 + (1+i).1)x = (1+i)x +
	& -i + (1+i).(1+i) = i
\end{align}

Entonces el resto se calcula de la siguiente manera
$$r = 1+i + (1+i).i$$
$$r = (1+i)(1+i)$$
$$r = 2i$$

Por lo tanto $P(i+1) = 2i$.

\subsection{Inciso h}
Calcular $P(2-i)$ es lo mismo que averiguar el resto de dividir por $Q(x) = (x - 2-i)$, por teorema del resto.

Aplicando el teorema de la regla de ruffini, podemos definir $C$ recursivamente:
\begin{align}
	C(x) =& x^3 +
	& (-i + (2-i).1)x^2 = (2-2i)x^2 +
	& (0 + (2-i).(2-2i))x = (2-6i)x +
	& -i + (2-i).(2-6i) = -2-15i
\end{align}

Entonces el resto se calcula de la siguiente manera
\begin{align}
	r &= 1+i + (2-i)(-2-15i)
	r &= 1+i + -19-28i
	r &= -18-27i
\end{align}

Por lo tanto $P(2-i) = -18-27i$

\subsection{Ejercicio 5}
Sea $P(x) = kx^4 + kx^3 - 33x^2 + 17x - 10$, calcular $P(4)$ sabiendo que $P(5) = 0$.\\
Que $P(5) = 0$ significa que
\begin{align}
	k.5^4 + k.5^3 - 33.5^2 + 17.5 - 10 &= 0
	k.5^4 + k.5^3 &= 10 - 17.5 + 33.5^2
	k.(5^4 + 5^3) &= 10 - 17.5 + 33.5^2
	k &= \dfrac{10 - 17.5 + 33.5^2}{5^4 + 5^3} 
	k &= \dfrac{(2 - 17 + 33.5)5}{(5^3 + 5^2).5} 
	k &= \dfrac{-15 + 33.5}{5^3 + 5^2} 
	k &= \dfrac{(-3 + 33).5}{(5+1).5^2} 
	k &= \dfrac{30}{6.5} 
	k &= \dfrac{30}{30} 
	k &= 1 
\end{align}

Entonces, por teorema del resto, $P(4)$ se obitene de calcular el resto de dividir $P$ por $Q(x) = x - 4$. Aplicamos Ruffini.
\begin{align}
	C =& 1.x^3 +\\
	& (1 + 4.1)x^2 = 5x^2 +\\
	& (-33 + 4.5)x = -12x +\\
	& (17 + 4.(-12)) = -31
\end{align}
Bajo este contexto
$$r = P(4) = -31 + 4.(-10) = -71$$

\section{Ejercicio 7}
\subsection{Inciso c}

Para que un polinomio $P$ tenga
\begin{itemize}
	\item $2$, raíz simple
	\item $i$, raíz triple
	\item $gr(P) = 4$
	\item $P(1) = 3i$
\end{itemize}

Por teorema de descomposición factorial, puedo llamar $Q(x) = (x-2)^1.(x-i)^3$ y este debe dividir a $P$ con resto 0. Observamos que $gr(Q) = 4$. Por lo tanto $Q$ y $P$ difieren por el producto de una constante $k \in \C$.
$$P(x) = k(x-2)(x-i)^3$$
Si además sabemos $P(1) = 3i$, deducimos $k$.
$$P(1) = k(1-2)(1-i)^3$$
$$3i = k.(-1).(-2-2i)$$
$$k = -3i.(-2-2i)^{-1}$$
$$k = -3i.(-2-2i)^{-1}$$
$$k = \dfrac{3}{4} + \dfrac{3}{4}i$$
Se puede demostrar que este polinomio es único.

\section{Ejercicio 8}
\subsection{Inciso h}
Sea $P(x) = (x^7 + x^4 - 9x^3 - 9)(x^3 + 1)$, factorizarlo.\\
Primero tratamos de reescribir el polinomio como producto de polinomios más fáciles de tratar.
$$(x^7 + x^4 - 9x^3 - 9)(x^3 + 1)$$
$$(x^4(x^3 + 1) - 9(x^3 + 1))(x^3 + 1)$$
$$(x^3 + 1)(x^4 - 9)(x^3 + 1)$$
$$(x^4 - 9)(x^3 + 1)^2$$
Luego simplemennte encontramos las raíces de $A(x) = x^4 - 9$ y $B = (x^3 + 1)$\\
Comenzando con $A(x) = 0$:
$$x^4 - 9 = 0$$
$$x^4 = 9$$
$$x = \sqrt[\leftroot{-2}\uproot{2}4]{9}$$
Luego $x$ puede adquirir 4 valores, por teorema de De Moivre. $\pm\sqrt{3}, \pm\sqrt{3}i$\\
Siguiendo con $B(x) = 0$:
$$x^3 + 1 = 0$$
$$x^3 = -1$$
$$x = \sqrt[\leftroot{-2}\uproot{2}3]{-1}$$
Luego $x$ puede adquirir 3 valores, por teorema de De Moivre. $-1, \left(\frac{1}{2}\pm\frac{\sqrt{3}}{2}i\right)$\\
Finalmente expresamos el resultado final. Donde si $P(\alpha) = 0$, entonces $P$ = $C.(x - \alpha)$
$$P(x) = \left(x+\sqrt{3}\right)\left(x-\sqrt{3}\right)\left(x+\sqrt{3}i\right)\left(x-\sqrt{3}i\right)\left(x+1\right)^2\left(x-\left(\frac{1}{2}+\frac{\sqrt{3}}{2}i\right)\right)^2\left(x-\left(\frac{1}{2}-\frac{\sqrt{3}}{2}i\right)\right)^2$$

\section{Ejercicio 9}
\subsection{Inciso a}
\begin{prf}[$P(\alpha) = 0 \Leftrightarrow Q(x) = P(-x), Q(-\alpha) = 0$]{}
	Sea $P(\alpha) = 0$.\\
	Llamemos $Q(x) = P(-x)$
	$$Q(-\alpha) = P(-(-\alpha)) = P(\alpha) = 0$$
	Sea $Q(x) = P(-x), Q(-\alpha) = 0$
	$$P(\alpha)$$
	Llamemos $-\alpha' = \alpha$
	$$P(-\alpha') = Q(\alpha')$$
	Pero $\alpha' = -\alpha$
	$$Q(\alpha') = Q(-\alpha) = 0$$
\end{prf}

\subsection{Inciso b}
\begin{prf}[$P(\alpha) = 0 \Rightarrow (P.Q)(\alpha) = 0$]{}
	Sea $P(\alpha) = 0, Q \in \C[x]$.\\
	$$(P.Q)(\alpha)$$
	Pero por teorema del algoritmo de la división. $(P.Q)(x)$ puede ser escrito como $P(x).Q(x)$
	$$P(\alpha).Q(\alpha) = 0.Q(\alpha) = 0$$
\end{prf}

\subsection{Inciso c}
\begin{prf}[Todo polinomio de grado impar admite al menos una raíz real]{}
	Sea $P \in \C[x], gr(P) = 2k+1, k \in \N$.\\
	Construyo $P(x) = x - i$. $gr(P) = 2.0+1 = 1$\\
	Por teorema fundamental del álgebra, $P$ tiene al menos una raíz compleja, y por corolario tiene exactamente una. La raíz es $i \notin \R$. Contradicción.
\end{prf}

\subsection{Inciso d}
\begin{prf}[Todo polinomio a coeficientes reales de grado impar admite al menos una raíz real]{}
	Sea $P \in \R[x], gr(P) = 2k+1, k \in \N$.\\
	Supongamos que todas las raíces son no reales. O sea, todas son de la forma $a + bi$ con $b \neq 0$. Pero por teorema
	$$P \in \R[x], P(\alpha) = 0 \Rightarrow P(\overline{\alpha}) = 0, \alpha \in \C$$
	Si $P(a+bi) = 0 \Rightarrow P(a-bi) = 0$.\\
	Por teorema fundamental del álgebra y su corolario, $P$ tiene $2k+1$ raíces. Pero por cada raíz $a+bi$, también es raíz $a-bi$, y vale $a+bi \neq a-bi$. Entonces la cantidad de raíces es dos veces la cantidad de raíces de la forma $a+bi$, digamos que la cantidad de raíces $a+bi$ es igual a $k' \in \N$. Por lo tanto
	$$2k+1 = 2k'$$
	Pero esto no tiene sentido en los naturales. Esto es una contradicción a la suposición de que no existe raíz de la forma $a+bi$ con $b \neq 0$. Por lo tanto existe al menos una.
\end{prf}

\subsection{Inciso e}
\begin{prf}[$P = Q \Leftrightarrow \forall \alpha P(\alpha) = 0 \Rightarrow Q(\alpha) = 0$]{}
	Sean $P,Q \in \C[x], P = Q$.\\
	Como son iguales, todo valor $P(\alpha) = Q(\alpha)$, eso incluye cuando $P(\alpha) = 0$. Por lo tanto tienen las mismas raíces.

	Sean $P,Q \in \C[x], \forall \alpha P(\alpha) = 0 \Rightarrow Q(\alpha) = 0$.\\
	Si construyo $P = x$ y $Q = x^2$. Son polinomios distintos pero aún así cumplen la condición, pues $0$ es raíz de $P$ y de $Q$. Esto es una contradicción con la hipótesis. Por lo tanto es falsa.
\end{prf}

\subsection{Inciso f}
\begin{prf}[Sean P y Q de grado n, son iguales si coinciden en n evaluaciones distintas]{}
	Sean $P, Q$ de grado $n = 1$. $P(x) = a_1(x - k), Q(x) = b_1.(x - k)$.\\
	Coinciden en $n = 1$ evaluaciones, pues $P(k) = Q(k)$. Pero $P \neq Q$, pues sus coeficientes son distintos.\\
	Esto es una contradicción con la hipótesis.
\end{prf}

\end{document}
