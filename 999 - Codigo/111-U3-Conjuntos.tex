\documentclass[11pt,a4paper]{article}
\usepackage[utf8]{inputenc}
\usepackage[spanish]{babel}
\usepackage{amsmath}
\usepackage{amsfonts}
\usepackage{amssymb}
\usepackage{graphicx}
\usepackage[left=2cm,right=2cm,top=2cm,bottom=2cm]{geometry}
\usepackage{multicol}
\author{Iker M. Canut}
\title{Unidad 3: Conjuntos\\\'Algebra y Geometr\'ia Anal\'itica I (R-111)\\Licenciatura en Ciencias de la Computaci\'on}
\date{2020}
\newcommand*{\QEDA}{\null\nobreak\hfill\ensuremath{\blacksquare}}
\newcommand*{\QEDB}{\null\nobreak\hfill\ensuremath{\square}}
\begin{document}
\maketitle
\newpage
\section{Teor\'ia de Conjuntos}
\noindent \textbf{Conjunto}: Colecci\'on bien definida de elementos. Los conjuntos se escriben con letras may\'usculas, los elementos con min\'usculas.
\begin{itemize}
\item $a \in A$: El elemento $a$ \textbf{pertenece} al conjunto $A$.
\item $a \not \in A$: El elemento $a$ \textbf{no pertenece} al conjunto $A$.
\end{itemize}
\noindent Definimos un conjunto por \textbf{extensi\'on} si enumeramos todos los elementos que pertenecen, o podemos definirlo por \textbf{comprensi\'on} si damos una caracter\'istica, una ley que define si un elemento pertenece o no al conjunto. \\ \noindent El universo en el cual estan todos los elementos, se lo denomina \textbf{universal}, $\mathbb{U}$.\\

\begin{itemize}
\item $C$ es un \textbf{subconjunto} de $D$ $\iff C \subseteq D \iff \forall x[x \in C \Rightarrow x \in D]$
\begin{multicols}{2}
\item $C \not \subseteq D \iff \exists x [x \in C \land x \not \in D]$
\item $C \subseteq D \Rightarrow |C| \leq |D|$
\end{multicols}

\item $C$ es un \textbf{subconjunto propio} de $D$ $\iff C \subset D \iff C \subseteq D \land C \not = D$
\begin{multicols}{2}
\item $C \not \subset D \iff C \not \subseteq D \lor C=D$
\item $C \subset D \Rightarrow |C| < |D|$
\end{multicols}

\item $C$ es \textbf{igual} a $D$ $\iff C = D \iff C \subseteq D \land D \subseteq C \iff \forall x[x \in C \iff x \in D]$
\item $C$ es \textbf{distinto} a $D$ $\iff C \not = D \iff C \not \subseteq D \lor D \not \subseteq C$\\
\end{itemize}

\noindent Sean $A, B, C \subseteq U$
\begin{multicols}{2}
\begin{itemize}
\item Si $A \subseteq B \land B \subseteq C \Rightarrow A \subseteq C$
\item Si $A \subseteq B \land B \subset C \Rightarrow A \subset C$
\item Si $A \subset B \land B \subseteq C \Rightarrow A \subset C$
\item Si $A \subset B \land B \subset C \Rightarrow A \subset C$
\end{itemize}
\end{multicols}

\noindent \textbf{Demostraci\'on}: $A \subset B \land B \subseteq C \Rightarrow A \subset C$.\\ Como $A \subset B$, entonces $\forall x \in A \Rightarrow x \in B \land \exists y \in B : y \not \in A$ y adem\'as como $\land B \subseteq C \forall x \in B \Rightarrow x \in C$. Para probar que $A \subset C$, hay que demostrar que $x\in A \Rightarrow x\in C$ y existe un $y \in C : y \not \in A$. Sea $x \in A \Rightarrow x \in B \Rightarrow x \in C$ y adem\'as $\exists y \in B : y \not \in A$, pero ese $y$ pertenece a $C$, ergo, $\exists y \in C : y \not \in A$. \QEDA \\

\noindent Se llama \textbf{conjunto vacio}, $\emptyset$ o $\{\}$ al conjunto que no tiene elementos. $|\emptyset| = 0$\\

\noindent Para cualquier $\mathbb{U}$, $A \subseteq U$ se tiene que $\emptyset \subseteq A$. Y si $a \not = \emptyset \Rightarrow \emptyset \subset A$\\
\noindent \textbf{Demostraci\'on}: Por absurdo, si $\emptyset \not \subseteq A$ entonces $\exists x \in \emptyset : x \not \in A$. Pero es absurdo. Luego, $\emptyset \subseteq A$. Finalmente, si $A \not = \emptyset$, entonces $\exists a[a \in A \land a \not \in \emptyset]\therefore \ \emptyset \subset A$ \QEDA \\

\noindent Dado un conjunto $A$, se llama \textbf{conjunto de partes} de $A$ al conjunto cuyos elementos son todos los subconjuntos de $A$. $P(A) = \{ F . F \subseteq A \}$\\

\noindent \textbf{Paradoja de Russel}: Sea $S$ el conjunto de todos los conjuntos $A$ que no son elemenos de si mismos, es decir $S = \{A: A \not \in A\}$, entonces $S \in S \iff S \not \in S$. Es decir, 
\begin{quote}
“El conjunto cuyos elementos son todos los conjuntos que no son elementos de sí mismos, ¿es o no es elemento de sí mismo?”
\end{quote}
Tambi\'en se lo conoce como \textit{"Paradoja del barbero"}. 
\newpage

\section{Operaciones de Conjuntos}
\begin{itemize}
\item \textbf{Uni\'on} de $A$ y $B$: Conjunto cuyos elemento pertenecen a $A$ o a $B$. $\ A \cup B = \{x\in \mathbb{U} . x\in A \lor x\in B\}$
\begin{itemize}
\item $A \cup B = B \cup A$\\
\indent \textbf{Dem/} $x \in (A \cup B) \iff x \in A \lor x \in B \iff x \in B \lor x \in A \iff x \in (B \cup A)$ \QEDA
\item $B \subseteq A \iff B \cup A = A$\\
\indent \textbf{Dem/} $\Rightarrow)\ x \in A \Rightarrow $ (por amp. disy.) $x \in A \lor x \in B \Rightarrow x \in (A \cup B) $ i.e $A \subseteq (A \cup B)$.\\
Sea $x \in (A \cup B) \Rightarrow x \in A \lor x \in B \Rightarrow x \in A \lor x \in A \Rightarrow x \in A$ i.e $(A \cup B) \subseteq A$ $\therefore A = B \cup A$\\
\indent $\Leftarrow)$ $x \in B \Rightarrow x \in B \lor x \in A \iff x \in B \cup A = A \iff x \in A \QEDA$
\item $(A \cup B) \cup C = A \cup (B \cup C)$\\
\indent \textbf{Dem/} $x \in (A \cup B) \cup C \iff x \in (A \cup B) \lor x \in C \iff (x \in A \lor x \in B) \lor x \in C \iff x \in A \lor (x \in B \lor x \in C) \iff x \in A \lor x \in (B \cup C) \iff x \in A \cup (B \cup C)$ \QEDA
\end{itemize}

\item \textbf{Intersecci\'on} de $A$ y $B$: Elementos que pertenecen a $A$ y a $B$. $\ A \cap B = \{x\in \mathbb{U} . x\in A \land x\in B\}$
\begin{itemize}
\item $A \cap B = B \cap A$\\
\indent \textbf{Dem/} $x \in A \cap B \iff x \in A \land x \in B \iff x \in B \land x \in A \iff x \in B \cap A$ \QEDA
\item $B \subseteq A \iff B \cap A = B$\\
\indent \textbf{Dem/} $\Rightarrow)\ x \in B \cap A \Rightarrow x \in B \land x \in A \Rightarrow x \in B$ i.e $B\cap A \subseteq B$ \\
Sea $x \in B \Rightarrow x \in B \land x \in B \Rightarrow x \in A \land x \in B \Rightarrow x \in A\cap B$ i.e $B \subseteq A\cap B \therefore B \cap A = B$\\
$\Leftarrow) $ Sea $x \in B \Rightarrow x \in A \cap B \Rightarrow x \in B \land x \in A \Rightarrow x \in A$ i.e $B \subseteq A$ \QEDA
\item $(A \cap B) \cap C = A \cap (B \cap C)$\\
\indent \textbf{Dem/} $x \in (A \cap B) \cap C \iff x \in (A \cap B) \land x \in C \iff (x \in A \land x \in B) \land x \in C \iff x \in A \land (x \in B \land x \in C) \iff x \in A \land x \in (B \cap C) \iff x \in A \cap (B \cap C)$ \QEDA
\item Dos conjuntos son \textbf{disjuntos} si la intersecci\'on es el conjunto vacio.
\item $X \cap Y = \emptyset \Rightarrow P(X) \cap P(Y) = \{\emptyset\}$\\
\textbf{Dem/} Suponemos que $P(X) \cap P(Y) \not = \{\emptyset\}$. Luego existe $Z \not = \emptyset : Z \in P(X) \cap P(Y)$. Entonces $Z \in P(X) \land Z \in P(Y) \Rightarrow Z \subseteq X \land Z \subseteq Y$. Y al no ser vacio, existe $x \in Z \subset X$, y para el mismo $x$, tenemos que $x \in Y$. Luego $X \cap Y \not = \emptyset$, contradiciendo la hipotesis. \QEDA
\item $A$ y $B$ son disjuntos $\iff$ $A \cup B = A \triangle B$\\
\textbf{Dem/} $\Rightarrow)\ x \in A \cup B \iff x \in A \cup B - x \in \emptyset \iff x \in A \cup B - x \in A \cap B \iff \\ x \in A \triangle B \therefore A \cup B = A \triangle B$\\
$\Leftarrow) A \cup B = A \triangle B \iff (A \cup B) = (A \cup B) - (A \cap B) \Rightarrow A \cap B = \emptyset$ i.e $A$ y $B$ son disjuntos. \QEDA

\end{itemize}

\item \textbf{Diferencia} de $A$ y $B$: Elem. que pertenecen a $A$ y no a $B$. $\ A - B = \{x\in \mathbb{U}.x\in A, x\not \in B\} = A \cap \overline{B}$
\begin{itemize}
\item $A - A = \emptyset$\\
\textbf{Dem/} $A - A = \{x : x \in A \land x \not \in A\} = \{x : F_0\} = \emptyset$ \QEDA
\item $A - \emptyset = A$\\
\textbf{Dem/} $A - \emptyset = \{x : x \in A \land x \not \in \emptyset\} = \{x : x \in A\} = A$ \QEDA
\item $\emptyset - A = \emptyset$\\
\textbf{Dem/} $\emptyset - A = \{x : x \in \emptyset \land x \not \in A\} = \emptyset$\QEDA
\item $(A-B = B-A) \iff A=B$\\
\textbf{Dem/} $\Rightarrow)$ Suponemos $A \not = B$. Luego $A-B = B-A \not = \emptyset$. Dado $x \in A-B$, para ese mismo $x$, $x \in B-A$. Por lo tanto $x \in A \land x \in B$, pero entonces no podria estar en $A-B$ o en $B-A$ llegando asi a una contradicci\'on.\\
$\Leftarrow) $ Sea $A = B \Rightarrow A - B = \emptyset \land B - A = \emptyset$ $\therefore A - B = B - A$ \QEDA
\item $(A-B)-C \subseteq A-(B-C)$\\
\textbf{Dem/} $x \in (A-B)-C \Rightarrow (x \in A \land x \not \in B) \land x \not \in C \Rightarrow (x \in A \land x \not \in B) \Rightarrow \\ x \in A \land (x \not \in B \lor x \in C) \Rightarrow x \in A \land \lnot (x \in B \land x \not \in C) \Rightarrow \\ x \in A \land \lnot (x \in B - C) \Rightarrow x \in A \land x \not \in B - C \Rightarrow x \in A-(B-C)$\QEDA
\end{itemize}

\newpage

\item \textbf{Complemento} de $B$ respecto de $A$: es la diferencia. $\ \complement_A B = A-B = \{x.x\in A, x\not \in B\}$\\
Si tomamos $A$ = $\mathbb{U}$, notamos $\complement_U B = \complement B = \overline{B}$
\begin{itemize}
\item $\complement \mathbb{U} = \emptyset$\\
\textbf{Dem/} $\complement\mathbb{U} = \complement_\mathbb{U} \mathbb{U} = \mathbb{U} - \mathbb{U} = \emptyset $\QEDA
\item $\complement \emptyset = \mathbb{U}$\\
\textbf{Dem/} $\complement \emptyset = \complement_\mathbb{U} \emptyset = \mathbb{U} - \emptyset = \mathbb{U} $\QEDA
\item $\complement (\complement A) = A$\\
\textbf{Dem/} $\complement (\complement A) = U - (U - A) = \{ x \in \mathbb{U} \land x \not \in (\mathbb{U} - A)\} = \{x \in \mathbb{U} \land \lnot (x \in \mathbb{U} \land x \not \in A)\} = \\ = \{x \in \mathbb{U} \land (x \not \in \mathbb{U} \lor x \in A)\} = \{ x \in \mathbb{U} \land x \in A\} = A$\QEDA
\item $\complement (A \cup B) = \complement A \cap \complement B$\\
\textbf{Dem/} $x \in \complement (A \cup B) = x \in \mathbb{U} \land x \not \in (A \cup B) = x \in \mathbb{U} \land \lnot (x \in A \lor x \in B)) = \\ = x \in \mathbb{U} \land x \not \in A \land x \not \in B$ \QEDA
\item $\complement (A \cap B) = \complement A \cup \complement B$\\
\textbf{Dem/} $\subseteq)\ x \in \overline{A \cap B} \Rightarrow \lnot(x \in A \land x \in B) \Rightarrow x \not \in A \lor x \not \in B \Rightarrow x \in \overline{A} \lor x \in \overline{B}$ \\
$\supseteq)\ x \in \overline{A} \cup \overline{B}$. Suponemos que $x \in A\cap B$, luego $x \in A \land x \in B \Rightarrow x \not \in \overline{A} \land x \not \in \overline{B} \Rightarrow \\ \lnot (x \in \overline{A} \land x \in \overline{B})$, contradiciendo asi la hip\'otesis. Luego, $x \in \overline{A \cap B}$ i.e $\overline{A} \cup \overline{B} \subseteq \overline{A \cap B}$\QEDA
\item $A \subseteq B \Rightarrow A \cup \complement_B A = B$\\
\textbf{Dem/} $\subseteq)\ x \in (A \cup \complement_B A) \Rightarrow x \in A \lor (x \in B \land x \not \in A) \Rightarrow x \in A \lor x \in B \Rightarrow \\ x \in B \lor x \in B \Rightarrow x \in B$ i.e $A \cup \complement_B A \subseteq B$\\
$\supseteq)\ x \in B \Rightarrow x \in A \lor x \in B \Rightarrow x \in A \lor (x \in B \land x \not \in A) \Rightarrow x \in A \cup (B - A) \Rightarrow \\ x \in A \cup \complement_B A$ i.e $A \cup \complement_B A \supseteq B\ \therefore\ A \cup \complement_B A = B$\QEDA
\end{itemize}

\item \textbf{Diferencia Simetrica} de $A$ y $B$: son los elementos que pertenecen a $A$ o a $B$, pero no a ambos.
\begin{align*}
A \triangle B 
&= \{x\in \mathbb{U} . x\in A\ \underline{\lor}\ x\in B\}\\
&= (A \cup B) - (A \cap B) = (A \cap \overline{B}) \cup (\overline{A} \cap B)\\
&= (A \cup B) \cap \overline{(A \cap B)} = (A-B) \cup (B-A)
\end{align*}

\begin{itemize}
\item $A \triangle B = (A \cup B) \cap \complement (A \cap B)$\\
\textbf{Dem/} $x \in A \triangle B \iff x \in A \cup B \land x \not \in (A \cap B) \iff \\
(x \in A \lor x \in B) \land \lnot (x \in A \land x \in B) \iff
x \in (A \cup B) \cap \overline{(A \cap B)}$ \QEDA
\item $A \triangle B = (A \cap \complement B) \cup (\complement A \cap B)$\\
\textbf{Dem/} $x \in A \triangle B \iff (x \in A \lor x \in B) \land \lnot (x \in A \land x \in B) \iff \\
((x \in A \lor x \in B) \land x \not \in A) \lor ((x \in A \lor x \in B) \land x \not \in B) \iff \\
(x \not \in A \land x \in B) \lor (x \in A \land x \not \in B) \iff x \in (A \cap \overline{B}) \cup (\overline{A} \cap B)$ \QEDA
\item $A \triangle B = (A-B) \cup (B-A)$\\
\textbf{Dem/} $x \in A \triangle B \iff x \in (A \cap \overline{B}) \cup (\overline{A} \cap B) \iff x \in (A - B) \cup (B - A)
$ \QEDA
\end{itemize}

\item \textbf{Producto Cartesiano} de $A$ y $B$: es el conjunto de pares ordenados $(a,b)$ tal que la primer componente pertenece a $A$ y la segunda pertenece a $B$.\\
$A\times B = \{ (a,b) . a\in A, b\in B \}$\\
Si $A = B$ se escribe $A \times A = A^2$\\

$[a,b] = \{x \in \mathbb{R} : a \leq x \leq b\}$\\
$[c,d] = \{x \in \mathbb{R} : c \leq x \leq d\}$\\
$[a,b]\times[c,d] = \{(x,y) \in \mathbb{R}^2 : a \leq x \leq b \land c \leq y \leq d\}$
\end{itemize}
\newpage

\section{Generalizaciones}
\noindent Sean $E_1, E_2,...E_n \subseteq U$ se llama:
\begin{itemize}
\item \textbf{uni\'on} de $E_1, E_2,...E_n$ al conjunto $E_1 \cup E_2 \cup ... \cup E_n$ = $\displaystyle{\bigcup_{i \in 1}^{n} E_i} = \{ x \in \mathbb{U} . x\in E_i, \text{para algun } i=1..n \}$
\item \textbf{intersecci\'on} de $E_1, E_2,...E_n$ al conjunto $E_1 \cap E_2 \cap ... \cap E_n$ = $\displaystyle{\bigcap_{i \in 1}^{n} E_i} = \{ x \in \mathbb{U} . x\in E_i, \forall i=1..n \}$
\end{itemize}

\noindent Sea $I$ un conjunto no vacio, $U$ el conjunto universal,\\
$\forall i \in I$ sea $A_i \subseteq \mathbb{U}$. Cada $i$ es un indice, e $I$ es el conjunto de indices.\\
\begin{itemize}
\begin{multicols}{2}
\item $\displaystyle{\bigcup_{i \in I} A_i} = \{ x \in \mathbb{U} . x\in A_i, \text{ para algun } i\in I \}$
\item $\displaystyle{\bigcap_{i \in I} A_i} = \{ x \in \mathbb{U} . x\in A_i, \text{ para todo } i\in I \}$
\end{multicols}
\end{itemize}
Equivalentemente,
\begin{itemize}
\begin{multicols}{2}
\item $x \in \displaystyle{\bigcup_{i \in I} A_i} \iff \exists i \in I . (x \in A_i)$
\item $x \in \displaystyle{\bigcap_{i \in I} A_i} \iff \forall i \in I . (x \in A_i)$
\end{multicols}
\end{itemize}

\begin{itemize}
\begin{multicols}{2}
\item $\overline{\displaystyle{\bigcup_{i \in I} A_i}} = \displaystyle{\bigcap_{i \in I} \overline{A_i}}$
\item $\overline{\displaystyle{\bigcap_{i \in I} A_i}} = \displaystyle{\bigcup_{i \in I} \overline{A_i}}$
\end{multicols}
\end{itemize}

\section{Leyes}
\begin{tabular}{p{0.03\textwidth}p{0.33\textwidth}p{0.33\textwidth}p{0.15\textwidth}}
1. & $\overline{\overline{A}} = A$ & & Doble negaci\'on \\
2. & $\overline{A \cup B} = \overline{A} \cap \overline{B}$ & $\overline{A \cap B} = \overline{A} \cup \overline{B}$ & De Morgan \\
3. & $A \cup B = B \cup A$ & $A \cap B = B \cap A$ & Conmutativa \\
4. & $(A\cup B) \cup C = A \cup (B \cup C)$ & $(A\cap B) \cap C = A \cap (B \cap C)$ & Asociativa \\
5. & $A \cup (B \cap C) = (A \cup B) \cap (A \cup C)$ & $A \cap (B \cup C) = (A \cap B) \cup (A \cap C)$ & Distributiva \\
6. & $A \cup A = A$ & $A \cap A = A$ & Idempotente \\
7. & $A \cup \emptyset = A$ & $A \cap \mathbb{U} = A$ & Neutro \\
8. & $A \cup \overline{A} = \mathbb{U}$ & $A \cap \overline{A} = \emptyset$ & Inverso \\
9. & $A \cup \mathbb{U} = \mathbb{U}$ & $A \cap \emptyset = \emptyset$ & Dominaci\'on \\
10. & $A \cup (A \cap B) = A$ & $A \cap (A \cup B) = A$ & Absorci\'on\\
    & $A \cup (\overline{A} \cap B) = A \cup B$ & $A \cap (\overline{A} \cup B) = A \cap B$
\end{tabular}


\section{Cardinalidad}
\noindent La cardinalidad de un conjunto finito es la cantidad de elementos que contiene.
\begin{itemize}
\item $|A\cup B| = |A| + |B| - |A \cap B|$
\item $|\overline{A} \cap \overline{B} \cap \overline{C}| = |\overline{A \cup B \cup C}| = |U| = |A \cup B \cup C|\\ = |U| - |A| - |B| - |C| + |A \cap B| + |A \cap C| + |B \cap C| - |A \cap B \cap C|$
\end{itemize}

\section{Dualidad}
\noindent Para conseguir el dual de un conjunto, se reemplazan:
\begin{itemize}
\item $\emptyset$ por $\mathbb{U}$ y $\mathbb{U}$ por $\emptyset$.
\item $\cup$ por $\cap$ y $\cap$ por $\cup$.
\end{itemize}

\end{document}