\documentclass[11pt,a4paper]{article}
\usepackage[utf8]{inputenc}
\usepackage[spanish]{babel}
\usepackage[left=2cm,right=2cm,top=2cm,bottom=2cm]{geometry}
\usepackage{multicol}
\usepackage{listings}

\usepackage{xcolor}
\definecolor{codegreen}{rgb}{0,0.6,0}
\definecolor{codegray}{rgb}{0.7,0.7,0.7}
\definecolor{codepurple}{rgb}{0.58,0,0.82}
\definecolor{backcolour}{rgb}{0,0,0.2}
\lstdefinestyle{mystyle}{
	basicstyle=\color{codewhite},
    backgroundcolor=\color{backcolour},
    commentstyle=\color{codegreen},
    keywordstyle=\color{magenta},
    numberstyle=\tiny\color{codegray},
    stringstyle=\color{codepurple},
    basicstyle=\ttfamily\footnotesize,
    breakatwhitespace=false,
    breaklines=true,
    captionpos=b,
    keepspaces=true,
    numbers=left,
    numbersep=5pt,
    showspaces=false,
    showstringspaces=false,
    showtabs=false,
    tabsize=2}
\lstdefinestyle{customc}{
	belowcaptionskip=1\baselineskip,
	breaklines=true,
	frame=L,
	xleftmargin=\parindent,
	language=C,
	showstringspaces=false,
	basicstyle=\footnotesize\ttfamily,
	keywordstyle=\bfseries\color{green!40!black},
	commentstyle=\itshape\color{purple!40!black},
	identifierstyle=\color{blue},
	stringstyle=\color{orange}}
\lstset{style=customc}

\setlength{\parindent}{0pt}

\author{Iker M. Canut}
\title{Listas Enlazadas}
\date{2021}
\begin{document}
\maketitle
\newpage

\section{Lista Simplemente Enlazada}
Estructura dinámica formada por una serie de nodos, conectados entre sí a través de una referencia. Se compone de informaci\'on y un puntero.
\begin{lstlisting}[language=C]
typedef struct _SNodo {
	int dato;
	struct _SNodo *sig;
} SNodo;

SNodo* slint_agregar_inicio(SNodo* lista, int dato) {
	SNodo* nuevoNodo = malloc(sizeof(SNodo));
	nuevoNodo->dato = dato;
	nuevoNodo->sig = lista;
	return nuevoNodo;
}

SNodo* slint_agregar_final(SNodo* lista, int dato) {
	SNodo* nuevoNodo = malloc(sizeof(SNodo));
	nuevoNodo->dato = dato;
	nuevoNodo->sig = NULL;
	
	if (lista == NULL) return nuevoNodo;
	else {
		SNodo* temp = lista;
		for(;temp->sig != NULL; temp=temp->sig);
		temp->sig = nuevoNodo;
	}	
	return lista;
}

SNodo* slint_agregar_finalR(SNodo* lista, int dato){
	if (lista == NULL) {
		lista = malloc(sizeof(SNodo));
		lista->dato = dato;
		lista->sig = NULL;
	}
	else {
		lista->sig = slist_agregar_finalR(lista->sig, dato);
	}
	return lista;
}
\end{lstlisting}

Una posible mejora puede ser directamente sobre el tipo de dato:
\begin{lstlisting}[language=C]
typedef struct _SNodo {
	int dato;
	struct _SNodo *sig;
} SNodo;

typedef struct SList {
	SNodo *primero;
	SNodo *ultimo;
} SList;
\end{lstlisting}
De esta manera, se puede acceder tanto al inicio como al final de una lista sin necesidad de recorrerla por completo.

\newpage

\section{Lista Doblemente Enlazada}
Estructura que nos permite trabajar con listas pero con la posibilidad de recorrerlas en ambos sentidos.
\begin{lstlisting}[language=C]
typedef struct _DNodo {
	int dato;
	struct _DNodo* sig;
	struct _DNodo* ant;
} DNodo;

typedef struct {
	DNodo* primero;
	DNodo* ultimo;
} DList;

DList* dlist_agregar_inicio(DList* lista, int dato) {
	DNodo* nuevoNodo = malloc(sizeof(DNodo));
	nuevoNodo->dato = dato;
	nuevoNodo->sig = lista->primero;
	nuevoNodo->ant = NULL;
	if (lista->primero != NULL)
		lista->primero->ant = nuevoNodo;
	if (lista->ultimo == NULL)
		lista->ultimo = nuevoNodo;
	lista->primero = nuevoNodo;
	return lista;
}

DList* dlist_agregar_final(DList* lista, int dato) {
	DNodo* nuevoNodo = malloc(sizeof(DNodo));
	nuevoNodo->dato = dato;
	nuevoNodo->sig = NULL;
	nuevoNodo->ant = lista->ultimo;
	if (lista->ultimo = NULL)
		lista->ultimo->sig = nuevoNodo;
	if (lista->primero == NULL)
		lista->primero = nuevoNodo;
	lista->ultimo = nuevoNodo;
	return lista;
}
\end{lstlisting}

Otra variante es cuando el siguiente del \'ultimo elemento apunta al primero, y en ese caso se dice que es una lista circular. Notamos que no necesitamos otra estructura.

\end{document}
