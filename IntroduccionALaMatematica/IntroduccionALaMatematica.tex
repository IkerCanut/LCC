\documentclass[10pt]{article}
\usepackage{hyperref}
\hypersetup{
    colorlinks=true,
    linkcolor=black,
    filecolor=magenta,      
    urlcolor=cyan,
}

\usepackage{geometry}
\geometry{
 a4paper,
 total={160mm,237mm},
 left=30mm,
 top=30mm,
}

\usepackage{multirow}

\author{Iker M. Canut}
\begin{document}
\title{Introducción a la Matemática}
\maketitle
\date
\newpage

\tableofcontents
\newpage

\section{Conjuntos}
\subsection{Definiciones Básicas}
Un Conjunto es una colección de objetos. Los conjuntos se denominan con letras mayúsculas. Y los elementos que lo forman con letras minúsculas. El conjunto vacio se denomina $\emptyset$.
\subsection{Representación de conjuntos}
\begin{itemize}
\item \textbf{Por Extensión}: Se lista todo entre llaves. $\{a,b,c,d,...\}$
\item \textbf{Por Comprension}: Se dicen las propiedades. $\{x/x...\}$
\end{itemize}
\subsection{Subconjuntos}
El conjunto B es subconjunto de A si y sólo si todo elemento de B, es también de A.
\begin{center}
\framebox[6cm][c]{$B \subset A \iff (x \in B \Rightarrow x \in A) $}
\end{center}
Dos conjuntos serán iguales cuando posean los mismos elementos.
\begin{center}
\framebox[6cm][c]{$B = A \iff (A \subset B \land B \subset A) $}
\end{center}
Al conjunto que contiene a todos los datos en un contexto específico lo denominaremos \textbf{Conjunto\\ Universal} y se denota con la letra \textbf{U}.
\subsection{Operaciones}
\begin{itemize}
\item \textbf{Intersección de Conjuntos}: $A \cap B = \{x/x \in A \land x \in B\}$
\item \textbf{Unión de Conjuntos}: $A \cup B = \{x/x \in A \lor x \in B\}$
\end{itemize}
Si dos conjuntos no tienen elementos en comun, entonces son \textbf{disjuntos}. A y B disjuntos $\iff A \cap B = \emptyset$
\vspace{-.2cm}
\begin{table}[h]
\begin{center}
\begin{tabular}{|c|c|c|}
\hline
Propiedades&UNIÓN&INTERSECCIÓN\\
\hline
\textit{Conmutativa}&$A \cup B = B \cup A$&$A \cap B = B \cap A$\\
\hline
\textit{Asociativa}&$(A \cup B) \cup C = A \cup (B \cup C)$&$(A \cap B) \cap C = A \cap (B \cap C)$\\
\hline
\textit{Distributiva}&$A \cup (B \cap C) = (A \cup B) \cap (A \cup C)$&$A \cap (B \cup C) = (A \cap B) \cup (A \cap C)$\\
\hline
\textit{Idempotencia}&$A \cup A = A$&$A \cap A = A$\\
\hline
\end{tabular}
\end{center}
\end{table}
\vspace{-.2cm}
\begin{itemize}
\item \textbf{Diferencia}: $A - B = \{x/x \in A \land x \not \in B\}$
\item \textbf{Complemento}: $C_A = \overline{A} = U - A$. Se cumple que $A - B = A \cap \overline{B}$
\end{itemize}
\begin{table}[h]
\begin{center}
\begin{tabular}{|c|c|}
\hline
\multicolumn{2}{|c|}{Propiedades}\\
\hline
&\\
\multirow{4}{*}{\textit{Complemento}}&$\overline{\overline{A}} = A$\\&$A \cup \overline{A} = U$\\&$A \cap \overline{A} = \emptyset$\\&$\overline{\emptyset} = U \land \overline{U} = \emptyset$\\
&\\
\hline
&\\
\multirow{2}{*}{\textit{Leyes de Morgan}}&$\overline{A \cap B} = \overline{A} \cup \overline{B}$\\&$\overline{A \cup B} = \overline{A} \cap \overline{B}$\\
&\\
\hline
\end{tabular}
\end{center}
\end{table}
\begin{itemize}
\item \textbf{Cardinal de un conjunto}: Es el número de elementos. $|A| = card(A)$
\end{itemize}
\newpage


\end{document}