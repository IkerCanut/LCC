\documentclass[10pt]{article}
\usepackage{hyperref}
\hypersetup{
    colorlinks=true,
    linkcolor=black,
    filecolor=magenta,      
    urlcolor=cyan,
}

\usepackage{import}
\usepackage{siunitx}
\usepackage{esvect}
\usepackage{fourier}
\usepackage{amssymb}
\usepackage{amsmath}
\usepackage{multicol}
\usepackage{geometry}
\usepackage[framemethod=TikZ]{mdframed}
\geometry{
 a4paper,
 total={160mm,237mm},
 left=30mm,
 top=30mm,
}

\usepackage{tikz}
\usetikzlibrary{automata, positioning, calc, through, angles, quotes, intersections}
\usepackage{multirow}

\subimport{.}{environment}

\author{Luciano N. Barletta \& Iker M. Canut}
\begin{document}
\title{Práctica de Álgebra y Geometría 1}
\maketitle
\date
\newpage

\tableofcontents
\newpage

\section{Unidad 1: Números Complejos}

\subsection{Preámbulo}

Definimos el conjunto de los números complejos de la siguiente manera:
$$\C = \{(a,b) \mid a,b \in \R \}$$
O sea que $\C = \R \times \R$.\\
Dado un $z = (a,b) \in \C$, llamamos parte real de $z$ al número real $a$ y la notamos $\textrm{Re}(z) = a$. Análogamente llamamos parte imaginaria a $b$ y la notamos $\textrm{Im}(z) = b$.\\
Definimos para todo $z=(a,b) \in \C$ y $w=(c,d) \in \C$:
$$z=w \Leftrightarrow a=c \land b=d$$
$$z+w = (a+c,b+d)$$
$$zw = (ac-bd,bc+ad)$$
Identificamos al conjunto:
$$\C_0 = \{ z \mid z = (a,0) \in \C, \forall a \in \R \}$$
que tiene una correspondencia
$$x \in \R \leftrightarrow (x,0) \in \C_0$$
Observamos:
$$(a,b) = (a,0)+(0,b) = (a,0)+(b,0)(0,1)$$
Ahora definimos:
$$i = (0,1)$$
Y usaremos esta notación para referirnos a este tipo de números complejos, $\forall a,b \in \R$:
$$
(a,b) = a+bi\\
(a,0) = a\\
(0,b) = bi\\
$$
\begin{prf}[$i^2 = -1$]{}
\lreqn{i^2 =}{<Definición de cuadrado>}
\lreqn{i.i =}{<Definición de $i$>}
\lreqn{(0,1)(0,1) =}{<Definición de producto de complejos>}
\lreqn{(0.0 - 1.1,1.0 + 0.1)}{}
De esto resulta el número complejo (-1,0), que representa al número real -1.
\end{prf}

\subsection{Demostraciones}

\begin{prf}[Conmutatividad de la suma]{}
\lreqn{z+w}{<Definición de suma de complejos>}
\end{prf}

\end{document}