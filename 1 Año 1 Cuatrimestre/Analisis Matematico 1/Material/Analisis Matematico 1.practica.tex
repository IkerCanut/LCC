\documentclass[10pt]{article}
\usepackage{hyperref}
\hypersetup{
    colorlinks=true,
    linkcolor=black,
    filecolor=magenta,      
    urlcolor=cyan,
}

\usepackage{import}
\usepackage{siunitx}
\usepackage{esvect}
\usepackage{fourier}
\usepackage{amssymb}
\usepackage{amsmath}
\usepackage{multicol}
\usepackage{geometry}
\usepackage[framemethod=TikZ]{mdframed}
\geometry{
 a4paper,
 total={160mm,237mm},
 left=30mm,
 top=30mm,
}

\usepackage{tikz}
\usetikzlibrary{automata, positioning, calc, through, angles, quotes, intersections}
\usepackage{multirow}

\subimport{.}{environment}

\author{Iker M. Canut}
\begin{document}
\title{Introducción a la Matemática}
\maketitle
\date
\newpage

\tableofcontents
\newpage

\section{Unidad 1: Numeros Reales}

\begin{prf}[Demostracion de la Propiedad Cancelativa de la Suma]{}
Sea $d=a+b$, y por ende, $d=b+c$, por el Axioma 5, existe $y$ que es opuesto  a $a$, entonces:
$$y+d=y+(a+b) \overset{A2}{=} (y+a)+b = 0 + b \overset{A4}{=}b$$
$$y+d=y+(a+c) \overset{A2}{=} (y+a)+c = 0 + c \overset{A4}{=}c$$
$$b=c$$
\end{prf}

\begin{prf}[Demostracion de la Unicidad del Elemento Neutro de la suma]{}
Supongamos que 0' es un numero que tambien funciona como neutro de la suma, entonces
$$a+0=a \land a+0'=a$$
$$a+0=a+0'$$
Y por propiedad cancelativa de la suma
$$0=0'$$
\end{prf}

\begin{prf}[Demostracion de la Unicidad del Elemento Opuesto]{}
La existencia de un numero b esta dada por el axioma 5, hay que demostrar que es unico. Suponiendo que existe $b'$ / $a+b'=b'+a=0$, tenemos que
$$a+b=0 \land a+b'=0$$
$$a+b = a+b'$$
Y por propiedad cancelativa de la suma
$$b=b'$$
\end{prf}

\begin{prf}[Demostracion de que el opuesto al opuesto de $a$ es $a$]{}
Sea $b$ el opuesto de $a$, se puede concluir que $a+b=0 \land b=(-a) \land a=(-b)$\hfill$(1)\land(2)\land(3)$
$$-(-a)\equals{(2)}-b\equals{(3)}a$$
\end{prf}

\begin{prf}[Demostracion de que el opuesto de 0 es 0]{}
Por el axioma 5, todo numero real tiene su opuesto. Llamemos 0' al opuesto de 0, siendo $0+0'=0$ y\\
Del axioma 3 se concluye que $0+0=0$
$$si\ 0+0'=0 \land 0+0=0 \Rightarrow 0'=0$$
\end{prf}

\begin{prf}[Demostracion de que el producto de 0 con cualquier otro numero es 0]{}
$$a.0 \equals{A4} a.0+0 \equals{A5} a.0+(a+(-a)) \equals{A2} (a.0+a)+(-a) \equals{A4} (a.0+a.1)+(-a) \equals{A3}$$
$$a(0+1)+(-a) \equals{A4} a.1+(-a) \equals{A4} a+(-a) \equals{A5} 0$$
\end{prf}

\begin{prf}[$a(-b)=-(ab)=(-a)b$]{}
$$
a(-b) \equals{A4}
a(-b)+0 \equals{A5}
a(-b)+(ab+-(ab)) \equals{A2}
(a(-b)+ab)+-(ab) \equals{A3}$$$$
(a((-b)+b)+-(ab)) \equals{A5}
a.0+-(ab) \equals{T2.3}
0+-(ab) \equals{A4}
-(ab)
$$
\end{prf}

\begin{prf}[$(-a)(-b)=ab$]{}
$$
(-a)(-b) \equals{T2.4}
-((-a)(-(-b))) \equals{T2.1}
-((-a)b) \equals{T2.4}
-(-(ab)) \equals{T2.1}
ab
$$
\end{prf}

\begin{prf}[$a(b-c)=ab-ac$]{}
Por la definicion de diferencia, se puede reescribir como:
$$
a(b+(-c)) \equals{A3}
ab+a(-c) \equals{T2.4}
ab+-(ac)
$$
Que por la definicion de diferencia, se puede reescribir como: $ab-ac$
\end{prf}


\begin{prf}[$a<b \implies a+c<b+c$]{}
$$
a < b \impliesbecause{Def <}
b - a \in \mathbb{R}^{+} \impliesbecause{A4}
(b - a) + 0 \in \mathbb{R}^{+} \impliesbecause{A5}
(b - a) + (c + -c) \in \mathbb{R}^{+} \impliesbecause{Def -}
(b + -a) + (c + -c) \in \mathbb{R}^{+}
$$

Reescribiendo usando $A1$ y $A2$:

$$
(b + c) + (-a + -c) \in \mathbb{R}^{+} \impliesbecause{T?}
(b + c) + -(a + c) \in \mathbb{R}^{+} \impliesbecause{Def -}
(b + c) - (a + c) \in \mathbb{R}^{+} \impliesbecause{Def <}
a + c < b + c
$$
\end{prf}

\begin{prf}[$a<b \land c>0 \implies ac<bc$]{}
Tenemos que:
$c > 0 \implies c \in \mathbb{R}^{+}$

Analizamos $a < b$:
$$
a < b \impliesbecause{Def <}
b - a \in \mathbb{R}^{+} \impliesbecause{A7 y c > 0}
(b - a)c \in \mathbb{R}^{+} \impliesbecause{A?}
bc - ac \in \mathbb{R}^{+} \impliesbecause{Def <}
ac < bc
$$
\end{prf}

\begin{prf}[$a \neq 0 => a^{2} > 0$]{}

Por la propiedad tricotómica, $a \neq 0 => a > 0 xor a < 0$. Analicemos los dos casos.

Analizemos a > 0:
$$
a > 0 \impliesbecause{Def >}
a - 0 \in \mathbb{R}^{+} \impliesbecause{Def -}
a + -0 \in \mathbb{R}^{+} \impliesbecause{T?}
a \in \mathbb{R}^{+} \impliesbecause{A?}
a * a \in \mathbb{R}^{+} \impliesbecause{Def > y Def x^{2}}
a^{2} > 0
$$
Analizamos a < 0:
$$
a < 0 \impliesbecause{Def <}
0 - a \in \mathbb{R}^{+} \impliesbecause{Def -}
0 + -a \in \mathbb{R}^{+} \impliesbecause{A?}
-a \in \mathbb{R}^{+} \impliesbecause{A?}
-a * -a \in \mathbb{R}^{+} \impliesbecause{T?}
$$

$$
(-1 * a) * (-1 * a) \in \mathbb{R}^{+} \impliesbecause{A?}
(-1 * -1) * (a * a) \in \mathbb{R}^{+} \impliesbecause{A?}
1 * (a * a) \in \mathbb{R}^{+} \impliesbecause{A?}
a * a \in \mathbb{R}^{+} \impliesbecause{Def > y Def x^{2}}
a^{2} > 0 
$$
\end{prf}

\begin{prf}[$1 \in \mathbb{R^{+}}$]{}

Existen neutros, 0 y 1, $0 \neq 1$

Ax 8, $1 \in \mathbb{R^{+}} xor -1 \in \mathbb{R^{+}}$

Supongo $-1 \in \mathbb{R^{+}}$, entonces $1 \notin \mathbb{R^{+}}$

Por Ax 7 $-1 * -1 \in \mathbb{R^{+}}$

$-1 * -1 = 1 \in \mathbb{R^{+}}$, pero esto es una contradicción con lo supuesto
\end{prf}

\end{document}