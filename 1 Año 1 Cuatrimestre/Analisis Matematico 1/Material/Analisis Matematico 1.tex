\documentclass[10pt]{article}
\usepackage{hyperref}
\hypersetup{
    colorlinks=true,
    linkcolor=black,
    filecolor=magenta,      
    urlcolor=cyan,
}

\usepackage{import}
\usepackage{siunitx}
\usepackage{esvect}
\usepackage{fourier}
\usepackage{amssymb}
\usepackage{amsmath}
\usepackage{multicol}
\usepackage{geometry}
\usepackage[framemethod=TikZ]{mdframed}
\geometry{
 a4paper,
 total={160mm,237mm},
 left=30mm,
 top=30mm,
}

\usepackage{tikz}
\usetikzlibrary{automata, positioning, calc, through, angles, quotes, intersections}
\usepackage{multirow}

\subimport{.}{environment}

\author{Iker M. Canut}
\begin{document}
\title{Introducción a la Matemática}
\maketitle
\date
\newpage

\tableofcontents
\newpage

\section{Unidad 1: Numeros Reales}
Los números reales son elementos de un conjunto denominado $R$ entre los que existen dos operaciones que, por definición, satisfacen ciertas propiedades específicas llamadas axiomas.\\ Las operaciones son la suma y el producto. Si $a,b \in R$
\begin{itemize}
\item Y la operación suma les asigna el elemento $c \in R$, escribimos: $a+b=c$
\item Y la operación producto les asigna el elemento $d \in R$, escribimos $a.b=d$
\end{itemize}


\subsection{Axiomas de los Numeros Reales}
\begin{axiom}[Axioma de Cuerpo 1: Conmutativa]{}
$$a+b=b+a \land a.b=b.a$$
\end{axiom}

\begin{axiom}[Axioma de Cuerpo 2: Asociativa]{}
$$(a+b)+c=a+(b+c) \land (a.b).c=a.(b.c)$$
\end{axiom}

\begin{axiom}[Axioma de Cuerpo 3: Distributiva de la Multiplicacion respecto a la Suma]{}
$$a.(b+c)=a.b+a.c$$
\end{axiom}

\begin{axiom}[Axioma de Cuerpo 4: Existencia de Elementos Neutros]{}
Existen dos numeros reales, notados 0 y 1 $/\ \forall a \in R$
$$0+a = a+0 = a \land 1.a = a.1 = a$$
\end{axiom}

\begin{axiom}[Axioma de Cuerpo 5: Existencia de Elementos Opuestos]{}
$$\forall a \in R, \exists\ b \in R\ / a+b = b+a = 0$$
\end{axiom}

\begin{axiom}[Axioma de Cuerpo 6: Existencia de Elementos Reciprocos]{}
$$\forall a \in R-\{0\}, \exists\ b \in R\ / a.b = b.a = 1$$
\end{axiom}

\begin{theo}[Propiedad Cancelativa de la Suma]{}
$$a,b,c \in R, si\ a+b = a+c,\ entonces\ b=c $$
\begin{prf}[Demostracion de la Propiedad Cancelativa de la Suma]{}
Sea $d=a+b$, y por ende, $d=b+c$, por el Axioma 5, existe $y$ que es opuesto  a $a$, entonces:
$$y+d=y+(a+b) \overset{A2}{=} (y+a)+b = 0 + b \overset{A4}{=}b$$
$$y+d=y+(a+c) \overset{A2}{=} (y+a)+c = 0 + c \overset{A4}{=}c$$
$$b=c$$
\end{prf}
\end{theo}

\begin{data}{}
Junto con los axiomas, se presupone la validez de las siguientes propiedades de la igualdad:
\begin{itemize}
\item Propiedad de Reflexibidad: $\forall a, a=a$
\item Propiedad de Simetria: $si\ a=b \Rightarrow b=a$
\item Propiedad de Transitividad: $si\ a=b \land b=c \Rightarrow a=c$
\end{itemize}
\end{data}

\begin{cor}[Unicidad del Elemento Neutro de la suma]{}
Si $0'$ es un numero que verifica que $a+0' = 0'+a = a$, $\forall a \in R$, entonces $0' = 0$
\begin{prf}[Demostracion de la Unicidad del Elemento Neutro de la suma]{}
Supongamos que 0' es un numero que tambien funciona como neutro de la suma, entonces
$$a+0=a \land a+0'=a$$
$$a+0=a+0'$$
Y por propiedad cancelativa de la suma
$$0=0'$$
\end{prf}
\end{cor}

\begin{cor}[Unicidad del Elemento Opuesto]{}
$\forall a \in R, \exists$ un unico numero $b$ / $a+b=b+a =0$
\begin{prf}[Demostracion de la Unicidad del Elemento Opuesto]{}
La existencia de un numero b esta dada por el axioma 5, hay que demostrar que es unico. Suponiendo que existe $b'$ / $a+b'=b'+a=0$, tenemos que
$$a+b=0 \land a+b'=0$$
$$a+b = a+b'$$
Y por propiedad cancelativa de la suma
$$b=b'$$
\end{prf}
\end{cor}

Para cualquier numero $a$, denotamos con $-a$ al unico elemento opuesto de $a$.

\begin{data}{}
Llamamos \textit{diferencia} entre dos numeros reales $a$ y $b$, y lo denotamos como $a-b$, al numero dado por la suma de $a$ y el opuesto de $b$.
$$a-b = a+ (-b)$$
\end{data}

\begin{theo}[]{}
\begin{multicols}{2}
$$-(-a)=a$$
$$-0 = 0$$
$$0.a = 0$$
$$a(-b) = -(ab) = (-a)b$$
$$(-a)(-b)=ab$$
$$a(b-c)=ab-ac$$
\end{multicols}
\end{theo}

\begin{prf}[Demostracion de que el opuesto al opuesto de $a$ es $a$]{}
Sea $b$ el opuesto de $a$, se puede concluir que $a+b=0 \land b=(-a) \land a=(-b)$\hfill$(1)\land(2)\land(3)$
$$-(-a)\equals{(2)}-b\equals{(3)}a$$
\end{prf}

\begin{prf}[Demostracion de que el opuesto de 0 es 0]{}
Por el axioma 5, todo numero real tiene su opuesto. Llamemos 0' al opuesto de 0, siendo $0+0'=0$ y\\
Del axioma 3 se concluye que $0+0=0$
$$si\ 0+0'=0 \land 0+0=0 \Rightarrow 0'=0$$
\end{prf}

\begin{prf}[Demostracion de que el producto de 0 con cualquier otro numero es 0]{}
$$a.0 \equals{A4} a.0+0 \equals{A5} a.0+(a+(-a)) \equals{A2} (a.0+a)+(-a) \equals{A4} (a.0+a.1)+(-a) \equals{A3}$$
$$a(0+1)+(-a) \equals{A4} a.1+(-a) \equals{A4} a+(-a) \equals{A5} 0$$
\end{prf}

\begin{prf}[$a(-b)=-(ab)=(-a)b$]{}
$$
a(-b) \equals{A4}
a(-b)+0 \equals{A5}
a(-b)+(ab+-(ab)) \equals{A2}
(a(-b)+ab)+-(ab) \equals{A3}$$$$
(a((-b)+b)+-(ab)) \equals{A5}
a.0+-(ab) \equals{T2.3}
0+-(ab) \equals{A4}
-(ab)
$$
\end{prf}

\begin{prf}[$(-a)(-b)=ab$]{}
$$
(-a)(-b) \equals{T2.4}
-((-a)(-(-b))) \equals{T2.1}
-((-a)b) \equals{T2.4}
-(-(ab)) \equals{T2.1}
ab
$$
\end{prf}

\begin{prf}[$a(b-c)=ab-ac$]{}
Por la definicion de diferencia, se puede reescribir como:
$$
a(b+(-c)) \equals{A3}
ab+a(-c) \equals{T2.4}
ab+-(ac)
$$
Que por la definicion de diferencia, se puede reescribir como: $ab-ac$
\end{prf}
\end{document}
